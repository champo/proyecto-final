\chapter{Conclusiones}
\label{chap-conclusion}

El Capítulo \ref{chap-problems} muestra cómo el problema en cuestión es de gran
complejidad y ha sido objeto de interés y de estudio con anterioridad. Diversos
estudios han estudiado soluciones a este problema, y en general las soluciones
existentes requieren de muchos recursos para lograr buenos resultados.

Teniendo en cuenta estas dificultades, sumadas a las restricciones del
problema, se concluye que la solución propuesta realiza un buen trabajo al
lograr el seguimiento de los jugadores con el material disponible. Puede
observarse como la solución propuesta resulta flexible y capaz de recuperarse
de forma semi-supervisada, algo que otras técnicas no contemplan, como por
ejemplo la técnica de \textit{IFTrace}.

Uno de los grandes impedimentos y fuente de problemas es el material utilizado,
de mala calidad debido a la baja resolución de la imagen y la elección para la
ubicación de la cámara. Conseguir filmaciones de mejor calidad es dificultoso
debido a los contratos y derechos de transmisión que actualmente rigen el
fútbol argentino. Para utilizar videos televisados de fútbol, se debe resolver
el problema de manejar los movimientos de cámara, lo que escapa al objetivo del
trabajo. Una desventaja de este tipo de material es que sólo ofrecen una visión
parcial del campo de juego (no se tiene información de todos los jugadores en
la cancha).

Un enfoque multi-cámara podría resolver el problema de la baja resolución de la
imagen y ofrecer mejores resultados (el ejemplo más sencillo serían 2 cámaras
que tomen, cada una, una mitad del campo de juego). Además, en el caso
de contar con suficientes cámaras, se podría disponer de mayor información a la
hora de manejar oclusiones, al tener disponible más de un punto de vista.

\section{Trabajo Futuro}

Este estudio se vería beneficiado de utilizar videos con mayor resolución, con
una cámara ubicada sobre el campo del juego apuntando perpendicularmente al
plano del campo de juego, o múltiples cámaras para poder hacer una comparación
justa del método de contornos activos contra los métodos utilizados por
\citeauthor{xu-8cams} y \citeauthor{papers-tanos} en los trabajos vistos en el
Capítulo 1.

Uno de los objetivos de investigación de este trabajo es lograr un seguimiento
en tiempo real, es decir procesar por lo menos 24 cuadros por segundo. La
implementación de referencia no cumple con este objetivo, pero es posible
avanzar. Por ejemplo, el trabajo de preprocesamiento de un cuadro y de
aplicación de \textit{Contornos Activos} puede distribuirse en 2 equipos
distintos. Esto permite tener una menor latencia entre cuadros, pudiendo
duplicar la velocidad de procesamiento.

Existe también la posibilidad de optimizar el algoritmo de \textit{Contornos
Activos}. El algoritmo original no está diseñado para seguir 24 contornos en
tiempo real. La implementación procesa cada contorno de manera secuencial,
desperdiciando la capacidad de multiprocesamiento de las computadoras modernas.
Se podría adaptar el algoritmo para aprovechar esta capacidad y mejorar el
tiempo de ejecución notablemente.

