\chapter{Conclusiones}
\label{chap-conclusion}

El Capítulo \ref{chap-problems} muestra cómo el problema en cuestión es de gran
complejidad y ha sido objeto de interés y de estudio con anterioridad. Diversos
estudios han estudiado soluciones a este problema, y en general las soluciones
existentes requieren de muchos recursos para lograr buenos resultados.

Teniendo en cuenta estas dificultades, sumadas a las restricciones del
problema, se concluye que la solución propuesta realiza un buen trabajo al
lograr el seguimiento de los jugadores con el material disponible. Puede
observarse como la solución propuesta resulta flexible y capaz de recuperarse
de forma semi-supervisada, algo que otras técnicas no contemplan, como por
ejemplo la técnica de \textit{IFTrace}.

Uno de los grandes impedimentos y fuente de problemas es el material utilizado,
de mala calidad debido a la baja resolución de la imagen y la elección para la
ubicación de la cámara. Conseguir filmaciones de mejor calidad es dificultoso
debido a los contratos y derechos de transmisión que actualmente rigen el fútbol
argentino. % TODO: Desde aca
Aún así, se insiste con este tipo de material ya que otro tipo de
material, como puede ser material televisivo, de fácil acceso por ser público,
no permite un análisis tan directo ya que complica la tarea de la ubicación de
los jugadores en un plano bidimensional. Sin mencionar que si la cámara no
ofrece una visión total del campo de juego se requieren más cámaras para
complementar la información.
% TODO: Hasta acá me gustaría revisar este párrafo

Un enfoque multi-cámara podría resolver el problema de la baja resolución de la
imagen y ofrecer mejores resultados (el ejemplo más sencillo serían 2 cámaras
que tomen, cada una, una mitad del campo de juego). Además, en el caso
de contar con suficientes cámaras, se podría disponer de mayor información a la
hora de manejar oclusiones, al tener disponible más de un punto de vista.
 De todas % TODO: Yo sacaría a partir de ahora
formas, el enfoque escogido utilizando una única cámara resulta más rentable en
el sentido de que los recursos utilizados se aprovechan al máximo de forma
eficiente, y estos recursos no requieren una gran inversión.

\section{Trabajo Futuro}

% TODO: Yo sacaría esto
Queda como posible investigación futura la utilización de otras características
para representar a los objetos, como podrían ser otros espacios de colores, o
el uso de descriptores de texturas.

Cabe destacar que trabajos futuros se verían beneficiados de contar con
material de mejor calidad. Lo ideal sería poder escoger la ubicación de la
cámara de modo de utilizar eficientemente todo el cuadro de la imagen para la
cancha, y contar con cámaras de mayor resolución.

Uno de los objetivos de investigación de este trabajo es lograr un seguimiento
en tiempo real, es decir procesar por lo menos 24 cuadros por segundo. La
implementación de referencia no cumple con este objetivo, pero es posible
avanzar. Por ejemplo, el trabajo de preprocesamiento de un cuadro y de
aplicación de \textit{Contornos Activos} puede distribuirse en 2 equipos
distintos. Esto permite tener una menor latencia entre cuadro y cuadro, y
podría duplicar la velocidad percibida.

Existe también la posibilidad de optimizar el algoritmo de \textit{Contornos
Activos}. El algoritmo original no está diseñado para seguir 24 contornos en
tiempo real. La implementación procesa cada contorno de manera secuencial,
desperdiciando la capacidad de multiprocesamiento de las computadoras modernas.
Se podría adaptar el algoritmo para aprovechar esta capacidad y mejorar el
tiempo de ejecución notablemente.

