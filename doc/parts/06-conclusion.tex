\chapter{Conclusiones}
\label{chap-conclusion}

Como se explica en la Capítulo \ref{chap-problems}, el problema en cuestión
tiene una gran complejidad, y ha sido objeto de interés y de estudio con
anterioridad. Se han dedicado muchos recursos y esfuerzo en la búsqueda de una
solución.

Teniendo en cuenta las numerosas y complejas dificultades, sumadas a las
restricciones del problema, se puede concluir que la solución propuesta en este
trabajo realiza un buen trabajo en el seguimiento de los jugadores en los
videos de fútbol. Resulta díficil obtener mejores resultados con los recursos
disponibles y las restricciones planteadas. Puede observarse como la solución
propuesta resulta flexible y capaz de recuperarse de forma semi-supervisada,
algo que otras técnicas complejas y robustas no contemplan, como por ejemplo la
técnica de \textit{IFTrace}.

%%TODO podemos decir que el material era malo? Y que era dificil
%% conseguir material de las mismas caracteristicas??

\section{Trabajo Futuro}

Queda como posible investigación futura la utilización de otras características
para representar a los objetos, como podrían ser otros espacios de colores, o
el uso de descriptores de texturas.

Cabe destacar que trabajos futuros se verían muy beneficiados de contar con más
material y de mejor calidad. Lo ideal sería poder escoger la ubicación de la
cámara de modo de utilizar eficientemente todo el cuadro de la imagen para la
cancha, y contar con cámaras de mayor resolución.

\subsection{Mejoras de performance}

Uno de los objetivos de investigación de este trabajo es lograr un seguimiento en
tiempo real, es decir procesar por lo menos 24 cuadros por segundo. La implementación
de referencia no cumple con este objetivo, pero tiene mucho lugar a optimización. 
Por ejemplo, El trabajo de preprocesamiento de un cuadro y de aplicación de Contornos
Activos puede distribuirse en 2 computadoras distintas. Esto permite tener una menor
latencia entre cuadro y cuadro, y podria duplicar la velocidad percibida. 

Existe tambien lugar a optimización en Contornos Activos. El algoritmo original no
esta diseñado para seguir 24 contornos en tiempo real. La implementación procesa
cada contorno de manera secuencial, desperdiciando la capacidad de multiprocesamiento
de las computadoras modernas. Se podria adaptar el algoritmo para aprovechar esta
capacidad y mejorar el tiempo de ejecución notablemente.

%- Mal material, imposible agarrar la pelota y difícil los jugadores
%- Resultados aceptables
