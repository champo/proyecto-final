\chapter{Conclusiones}
\label{chap-conclusion}

Como se explica en la Capítulo \ref{chap-problems}, el problema en cuestión
tiene una gran complejidad, y ha sido objeto de interés y de estudio con
anterioridad. Se han dedicado muchos recursos y esfuerzo en la búsqueda de una
solución.

Teniendo en cuenta las numerosas y complejas dificultades, sumadas a las
restricciones del problema, se puede concluir que la solución propuesta en este
trabajo realiza un buen trabajo en el seguimiento de los jugadores en los
videos de fútbol. Resulta díficil obtener mejores resultados con los recursos
disponibles y las restricciones planteadas. Puede observarse como la solución
propuesta resulta flexible y capaz de recuperarse de forma semi-supervisada,
algo que otras técnicas complejas y robustas no contemplan, como por ejemplo la
técnica de \textit{IFTrace}.

Uno de los grandes impedimentos y fuente de problemas es el material utilizado,
su baja calidad, medida en la baja resolución de la imagen y la pobre elección
para la ubicación de la cámara, y la dificultad de obtención del mismo. Esto
último se debe a los contratos y derechos de transmisión que actualmente rigen
el fútbol argentino. Aún así, se insiste con este tipo de material ya que otro
tipo de material, como puede ser material televisivo, de fácil acceso por ser
público, no permite un análisis tan directo ya que complica la tarea de la
ubicación de los jugadores en un plano bidimensional. Sin mencionar que si la
cámara no ofrece una visión total del campo de juego se requieren más cámaras
para complementar la información.

Un enfoque multi-cámara, el ejemplo más sencillo sería 2 cámaras que tomen,
cada una, una mitad del campo en su totalidad, podría resolver el problema
de la baja resolución de la imagen y ofrecer mejores resultados. Además, en el caso
de contar con suficientes cámaras, podría proporcionar información útil a
la hora de manejar oclusiones, ofreciendo más de un punto de vista. De todas
formas, el enfoque escogido de utilizar una única cámara resulta más rentable
en el sentido de que los recursos utilizados se aprovechan al máximo de forma
eficiente, y estos recursos no requieren una gran inversión.

\section{Trabajo Futuro}

Queda como posible investigación futura la utilización de otras características
para representar a los objetos, como podrían ser otros espacios de colores, o
el uso de descriptores de texturas.

Cabe destacar que trabajos futuros se verían muy beneficiados de contar con más
material y de mejor calidad. Lo ideal sería poder escoger la ubicación de la
cámara de modo de utilizar eficientemente todo el cuadro de la imagen para la
cancha, y contar con cámaras de mayor resolución.

\subsection{Mejoras de performance}

Uno de los objetivos de investigación de este trabajo es lograr un seguimiento en
tiempo real, es decir procesar por lo menos 24 cuadros por segundo. La implementación
de referencia no cumple con este objetivo, pero tiene mucho lugar a optimización.
Por ejemplo, el trabajo de preprocesamiento de un cuadro y de aplicación de \textit{Contornos
Activos} puede distribuirse en 2 computadoras distintas. Esto permite tener una menor
latencia entre cuadro y cuadro, y podría duplicar la velocidad percibida.

Existe también la posibilidad de optimizar el algoritmo de \textit{Contornos
Activos}. El algoritmo original no está diseñado para seguir 24 contornos en
tiempo real. La implementación procesa cada contorno de manera secuencial,
desperdiciando la capacidad de multiprocesamiento de las computadoras modernas.
Se podría adaptar el algoritmo para aprovechar esta capacidad y mejorar el
tiempo de ejecución notablemente.

%- Mal material, imposible agarrar la pelota y difícil los jugadores
%- Resultados aceptables
