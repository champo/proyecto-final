\section{Solución propuesta}
\label{sec:solution}

En esta sección describimos las distintas tecnicas aplicadas. Primero describimos algoritmos para
ignorar los elementos del fondo de la imagen en \ref{sec:background-elimination}. Luego, explicamos
como utilizamos el algoritmo contornos activos\cite{fast-level-set} y las modificaciónes necesarias
en \ref{sec:ac}. Finalmente, detallamos la integración final entre las distintas tecnicas, que es
la que analizamos en la siguiente seccion \ref{sec:results}.

\subsection{Eliminación de fondo por varios métodos}
\label{sec:background-elimination}
Problemas y soluciones (tal vez como subsubsection)

\subsubsection{Tribuna y publicidades}
% Simplemente como ponemos en negro para que no interfiera

\subsubsection{Energia}
  * Método de la energía

\subsubsection{Eliminación de lineas}
  * Eliminación de lineas por Detector de borde + umbral + morfología

\subsubsection{Eliminación del pasto}
  * Eliminación de colores de cancha por histograma de colores


\subsection{Contornos activos}
\label{sec:ac}

- Características

  * Colores del jugador
  * Aprendizaje
  * Múltiples características
  * Selección de máximos en histograma con bfs

- Descriptores

\subsection{Algoritmo final}
\label{sec:alg-final}
% Que onda esta seccion? Me parece medio tirada de los pelos la idea

- Eliminación de lineas por detector de borde + umbral + morfología
- Múltiples características
- Descriptores
- Imágenes de cómo resolvemos algunos casos extremos

- Opcional para equipos difíciles de detectar: seguimiento por complemento
  - eliminación de colores de cancha por histograma + ronda de contornos

- Que lastima que no tuvimos trackeo de pelota porque hubiese dado mejores datos
