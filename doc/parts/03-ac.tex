\chapter{Descripción del Método}
\label{chap-ac}

En este Capítulo se describe el método de seguimiento seleccionado para este
trabajo. En primer lugar, la Sección \ref{sec:ac} explica el algoritmo
utilizado (\emph{Contornos Activos}), y en la Sección \ref{sec:impl} su
implementación. Luego, en la Sección \ref{sec:eleccion} se justifica la
elección del algoritmo. Finalmente, en la Sección \ref{sec:ac-problemas} se
analizan las limitaciones del algoritmo para esta aplicación.

\section{Contornos Activos}
\label{sec:ac}
La segmentación basada en contornos activos se basa en definir una región en
base a su contorno. El contorno o borde de una región $\Omega_i$ se representa
por una curva paramétrica dada por:

\begin{equation}
    C_i(s) = (x_i(s), y_i(s))
\end{equation}
donde $0 \leq s \leq S_i$, y $C_i(0) = C_i(S_i)$, siendo $S_i$ el total de
puntos que conforman la curva. Es decir, $C_i$ es una curva cerrada.

Se definen tantos contornos como objetos de interés haya en la secuencia ($i$
es un entero entre 1 y el número de objetos). Adicionalmente, se considera una
región $\Omega_0$ que contiene a todo punto que no forma parte de ninguna otra
región. Se la denomina \textit{región de fondo}.

Se declara una función de probabilidad $p$ que permite saber que tan probable
es que un píxel forme parte de una dada región. Para esto, es necesaria una
función $v$ que dado un píxel devuelve un vector $v(x)$ de características,
por ejemplo los valores RGB del píxel. Esto permite calcular $p(v(x) \vert
\Omega_i)$, la probabilidad de que un píxel $x$ forme parte de una región
$\Omega_i$. Estas funciones dependen de la secuencia particular a ser
analizada, por lo que varían dependiendo del caso de estudio. Como ejemplo, se
toman los valores RGB como vector de características y $p(v(x) \vert \Omega_i)
= \| v(x) - v_i \| $, donde $v_i$ es el valor promedio RGB de todo píxel en la
región $\Omega_i$.

Se define la función de energía de los contornos:

\begin{equation}
    \label{eq:ac-energy}
    E = - \sum_{m=0}^{M}{\int_{\Omega_m}{\log{p(v(x) \vert \Omega_m)} dx} + \lambda \int_{C_m}{ds}}
\end{equation}

Los algoritmos basados en contornos activos buscan minimizar esta ecuación. Si
el valor de $E$ es mínimo para un cuadro, se considera que se encontró la mejor
aproximación de la región de los objetos interés. De \ref{eq:ac-energy} se
deriva la ecuación de evolución de cualquier curva $C_m$:

\begin{eqnarray}
    \frac{dC_m}{dt} &=& (F_d + F_s) \overrightarrow{N}_{C_m} \\
    F_d &=& \log{p(v(x) \vert \Omega_m) / p(v(x) \vert \Omega_0)} \\
    F_s &=& \lambda \kappa_m \label{eq:ac-formal}
\end{eqnarray}

$F_d$ y $F_s$ son fuerzas derivadas de la ecuación de energía. $F_d$ representa
la competencia entre regiones y $F_s$ una función que produce el efecto de
suavizado. $\kappa_m$ es la curvatura de la curva $C_m$.

\section{Implementación Numérica}
\label{sec:impl}

Se toma de referencia la implementación según \citeauthor{fast-level-set} (ver
\cite{fast-level-set}). En la implementación, el borde de un contorno $C_m$ se
representa usando dos conjuntos de píxeles correspondientes a los bordes
interno y externo, $L_{in}$ y $L_{out}$ respectivamente. Entonces, la evolución
se realiza intercambiando píxeles entre estos dos conjuntos.

A continuación se detalla la técnica para la segmentación de un solo objeto de
interés que se puede fácilmente extrapolar y utilizar para la segmentación de
múltiples objetos
\footnote{Para la segmentación de múltiples objetos basta con marcar las
distintas regiones con un identificador distinto y seguirlas por separado.}.

El objeto de interés $\Omega_{1}$ y el fondo $\Omega_{0}$ cumplen
$\Omega_{1}\cup\Omega_{0} = I_{k}$, donde $I_{k}$ es la imagen del cuadro $k$
de la secuencia de imágenes, y $\Omega_{1}\cap\Omega_{0} = \emptyset$. Cada una
de las regiones está caracterizada por su vector característico $v_{m}, m =
\{0,1\}$.

Se define una función $\phi(x)$ que indica si un píxel $x$ pertenece a una
región o al fondo de la siguiente manera:

\begin{equation}
\phi(x) =
\left\{
    \begin{array}{ll}
        3  & \mbox{si } x \in \Omega_{0} \mbox{  y  } x \notin L_{out} \\
        1  & \mbox{si } x \in L_{out}\\
        -1  & \mbox{si } x \in L_{in}\\
        -3 & \mbox{si } x \in \Omega_{1} \mbox{  y  } x \notin L_{in} \\
    \end{array}
\right.
\end{equation}

Los conjuntos $L_{in}$ y $L_{out}$ se definen como

\begin{equation}
    L_{in} = \{ x \mbox{ es un píxel } \vert \mbox{    }  \phi(x) < 0 \mbox{ y } \exists y \in N_{4}(x) \mbox{ de modo que } \phi(y) > 0 \}
\end{equation}

\begin{equation}
    L_{out} = \{ x \mbox{ es un píxel } \vert \mbox{    } \phi(x) > 0 \mbox{ y } \exists y \in N_{4}(x) \mbox{ de modo que } \phi(y) < 0 \}
\end{equation}

donde $N_{4}(x) = \{ y \mbox{ es un píxel } \vert \mbox{   } |x-y| = 1 \}$, son
los píxeles vecinos del píxel $x$.

El algoritmo de segmentación se compone de dos etapas, ya que luego de la
especificación inicial de la curva en forma supervisada
\footnote{Podría no ser supervisada. Existen variantes con determinaciones
semi-supervisadas del objeto de interés, así como también detección automática
basada en ciertas características predefinidas.}
se intercambian los píxeles de $L_{in}$ y $L_{out}$ en dos ciclos. En el
primero, se aplica la fuerza $F_{d}(x)$, y en el segundo se aplica $F_{s}(x)$
para la regularización.

En el primer ciclo, se ejecutan los siguientes pasos $N_{a}$ veces, donde $ 0 <
N_{a} < max(filas, columnas)$.

\begin{enumerate}

    \item Para cada $x \in L_{out}$, si $F_{d}(x) > 0$ entonces borrar $x$ de $L_{out}$ y agregarlo a $L_{in}$. \\
    Luego, $\forall y \in N_{4}(x)$, con $\phi(y) = 3$, agregar $y$ to $L_{out}$ y hacer $\phi(y) = 1$.

    \item Después del paso 1 algunos de los píxeles $x$ en $L_{in}$ pasan a ser píxeles internos. \\
    Por lo tanto, se sacan de $L_{in}$ y se hace $\phi(x) = -3$.

    \item Para cada $x \in L_{in}$ , si $F_{d}(x) < 0$ entonces, borrar $x$ de $L_{in}$ y agregarlo a $L_{out}$. \\
    Luego, $\forall y \in N_{4}(x)$, con $\phi(y) = -3$, agregar $y$ a $L_{in}$ y hacer $\phi(y) = -1$.


    \item Después del paso 3 algunos de los píxeles $x$ en $L_{out}$ pasan a ser píxeles externos. \\
    Por lo tanto, se sacan de $L_{out}$ y se hace $\phi(x) = 3$.

\end{enumerate}

En el segundo ciclo, la curva se suaviza utilizando un filtro Gaussiano, de tal
forma que la fuerza de evolución es $F_{s}(x) = G \otimes \phi(x)$. Para
aplicar $F_s$ se usan los mismos pasos que para $F_d$. El resultado final es
análogo a modificar la curva de acuerdo a la definición formal dada
anteriormente (Ecuación \ref{eq:ac-formal}).

En cada cuadro, el borde del contorno del objeto es actualizado de acuerdo al
resultado obtenido por el algoritmo en el cuadro anterior. En el caso de la
primera imagen, se puede dar de forma supervisada o semi-supervisada por el
usuario, o bien puede ser obtenida de forma automática mediante algoritmos de
aprendizaje complejos basados en características predefinidas.

El algoritmo termina cuando se alcanza la condición de corte, dada por las
ecuaciones \ref{eq:active-contours-stoppingCondition} o cuando se alcanza el
numero de iteraciones $N_a$

\begin{equation}
\label{eq:active-contours-stoppingCondition}
    \begin{array}{ll}
        F_{d}(x) \leq 0 & \forall x \in L_{out}\\
        F_{d}(x) \leq 0 & \forall x \in L_{in}
    \end{array}
\end{equation}

\section{Elección}
\label{sec:eleccion}

Se escogió el algoritmo \emph{Contornos Activos} por varios motivos:
\begin{itemize}
    \item No depende de disponer de bloques de $n \times n$ píxeles para su
        correcto funcionamiento. Una de las principales restricciones de este
        trabajo es utilizar una sola cámara para obtener el video, y, como se
        explica en la Sección \ref{sub-sec:camaras}, un jugador está formado
        por un número muy limitado de píxeles, dificultando el uso de
        algoritmos que no tengan esta característica.

    \item Su tiempo de ejecución no depende del tamaño del cuadro, sino del
        de los jugadores. Esto hace que el tiempo de procesamiento de un cuadro
        sea muy pequeño, posibilitando el análisis en tiempo real.

    \item Se cuenta con métodos de manejo de oclusiones para
        \emph{Contornos Activos}\cite{paper-juliana}.

\end{itemize}

\subsection{Parámetros}

Se seleccionaron experimentalmente los siguientes valores para las constantes
del método:

\begin{itemize}
\item $N_a = 500$ 
\item Para el filtro $G$, se usa una máscara de $7 \times 7$ con un $\sigma = 0.7$ 
\end{itemize}

\section{Limitaciones}
\label{sec:ac-problemas}

El correcto funcionamiento del algoritmo de contornos activos depende de una
buena selección de la función característica, para poder distinguir claramente
a un jugador respecto a otros objetos o respecto del fondo (ver
\cite{fast-level-set}). En casos como el ilustrado por la Figura
\ref{fig:camiseta}, elegir una función característica resulta sencillo, ya que
con elegir el color de la camiseta se asegura una buena descripción del
contorno a seguir. Sin embargo, la imagen de la Figura
\ref{fig:camiseta-rayada} introduce uno de los problemas en la selección de
esta función. Se puede ver que las diferencias entre ambos colores del objeto
hacen difícil identificarlos a ambos con una misma característica. Si se toma
el promedio del valor \textit{RGB}, considerando que el color negro tiene valor
$(0, 0, 0)$ y el rojo $(255, 0, 0)$, el promedio de ambos seria $(127, 0, 0)$
(un marrón oscuro). Este valor muestra distinto a ambos colores, y no describe
bien a ninguno. Es por eso, que en la Figura \ref{fig:camiseta-rayada} el 
algoritmo de contornos activos podría determinar que uno de los dos colores de
la camiseta es más similar al fondo que al promedio, y en consecuencia
actualizar el contorno siguiendo únicamente uno de los dos colores.

\begin{figure}[H]
    \centering
    \begin{minipage}[t]{.5\textwidth}
        \centering
        \includegraphics[width=.4\linewidth]{./images/rect2995.png}
        \captionof{figure}{Camiseta de color lisa. El color de la camiseta es
          claramente distinguible del fondo.
          \label{fig:camiseta}
        }
    \end{minipage}%
    \begin{minipage}[t]{.5\textwidth}
        \centering
        \includegraphics[width=.4\linewidth]{./images/rect2996.png}
        \captionof{figure}{Camiseta de 2 colores a rayas. Uno de los dos
          colores es más similar al color de fondo que al promedio.
          \label{fig:camiseta-rayada}
        }
    \end{minipage}
\end{figure}

Esto es dificultoso debido a dos problemas:
\begin{itemize}

\item Selección de colores: si la cancha tiene un color muy similar a la
  camiseta de un equipo, ¿cómo será posible distinguirlos? En este trabajo se
  exploran distintas alternativas, como utilizar distintas codificaciones de
  color.

\item Textura de los jugadores: esto representa un gran problema por varias
  razones. La primera es que la técnica de contornos activos depende de la
  selección de una o varias características del objeto a seguir. Como la
  característica más distintiva es el color, la técnica se basa en ella. Pero
  para camisetas con más de un color, encontrar un único color característico
  no es posible. Esta dificultad se ilustra en la imágenes de las figuras
  \ref{fig:barsa1} y \ref{fig:barsa2}. Por otro lado, dada la escasa cantidad
  de píxeles que representan a un jugador, debido al enfoque de única
  cámara de este trabajo, cualquier tipo de análisis se torna complejo cuando
  sus características no están lo suficientemente definidas como para
  diferenciarlas (ya sea de otros jugadores o del fondo).

\end{itemize}

\begin{figure}[H]
    \centering
    \begin{minipage}{.5\textwidth}
        \centering
        \includegraphics[width=.4\linewidth]{./images/resize_barcelona1.png}
        \captionof{figure}{Desde este punto de vista, se puede observar al
        menos 3 fuertes características (colores) en la camiseta del jugador,
        rojo, azul y amarillo.}
        \label{fig:barsa1}
    \end{minipage}%
    \begin{minipage}{.5\textwidth}
        \centering
        \includegraphics[width=.4\linewidth]{./images/resize_barcelona4.png}
        \captionof{figure}{Incluso con mayor resolución, las distintas
        características (como ser los colores azul, amarillo, y rojo)
        dificultan el seguimiento con contornos activos.}
        \label{fig:barsa2}
    \end{minipage}
\end{figure}

%TODO aca mepa que pueden ir varias imagenes.
% 1 de un tracking de de un jugador de remera blanca (podría ser PRE y POST, osea sin pintar y pintado)
% 1 de un tracking de un jugador de boca (again PRE y POST)
% 1 de la pelota? Para mostrar la cantidad infima de píxeles?

% eordano says: Me parece que este TODO quedó viejo, lo borro?
