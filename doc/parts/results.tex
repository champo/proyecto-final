\section{Resultados}
\label{sec:resultados}

% TODO: Discutir sobre diferencias semanticas entre algoritmo, implementación, programa... etc. Qué usamos y que cosas las tomamos como sinonimos. Tal vez hacer un glosario...
Esta sección presenta un análisis de los resultados obtenidos de la ejecución
del algoritmo sobre distintos videos. % [Llenar más]

Se presentan las salidas de la implementación, métricas utilizadas para evaluar la performance del algoritmo. A lo largo del capítulo se discuten alternativas propuestas y descartadas, resultados parciales, y otras métricas que no se calcularon.

\subsection{Aplicación}

La aplicación implementada incorpora una interfaz gráfica utilizada tanto para dar instrucciones como para recibir información acerca del partido y del funcionamiento del algoritmo. La misma cuenta con una interfaz donde se muestra a los jugadores, una lista de jugadores siendo seguidos, un mapa de calor de espacios ocupados por los jugadores, y una imágen que permite al operador evaluar el correcto funcionamiento del algoritmo.

Para comenzar la ejecución, el algoritmo requiere que el operador identifique la posición de los jugadores en la imagen inicial de la secuencia. % [ TODO: Completar con chamuyo ]

Un paso adicional es el de determinar puntos a utilizar para el cálculo de la homografía. Los datos necesarios son al menos cuatro parejas de puntos. De estas parejas, el primer punto es una coordenada la imagen en perspectiva y el otro punto es la coordenada
en un plano bidimensional. Con cuatro de estas relaciones, se puede calcular la matriz que resuelve la homografía para cualquier otro punto. % [ TODO: esto no queda claro ]

Una vez que el programa tiene estos datos, el algoritmo de contornos activos puede comenzar a correr. Existen dos modos, uno que avanza sólo ante la indicación del operador (mecánica ``cuadro por cuadro'') y otro modo que avanza automáticamente cada vez que se computa un cuadro (mecánica ``tiempo real''). % [ TODO: algo de amor a este párrafo ]

Cada cuadro, una ventana muestra las posiciones actualizadas de los jugadores. El usuario puede seleccionar un jugador en particular y ver en detalle información sobre su posición, velocidad promedio, actual, y máxima desde el comienzo del seguimiento, y un mapa de calor que muestra con colores próximos al rojo los lugares más visitados por el jugador. Además, en un archivo se escribe la posición de cada jugador por cada frame.

De manera opcional, en cada cuadro se guarda en disco duro una copia del estado actual del seguimiento para referencia futura.

\subsection{Material utilizado}

Se utilizaron principalmente dos videos, uno correspondiente a un partido entre los equipos argentinos de los clubes Boca Juniors e Independiente; y un segundo video en el cual se enfrentan los equipos Independiente y San Lorenzo. Se detalla a continuación las características de las imágenes extraídas de esos videos.

\begin{itemize}
  \item \textbf{Boca vs Independiente:}
  \item \textbf{Independiente vs San Lorenzo:}
\end{itemize}

Además, se tomaron pequeños cortes de tres videos de fútbol televisado en los cuales la cámara se encuentra relativamente estática y se analizó la correctitud del seguimiento en estos casos, contando con mayor resolución pero sin poder efectuar exitosamente la eliminación de fondo o la correlación a un plano bidimencional por homografía % [ TODO: ESTO ESTÁ ESCRITO CON LA ** ]

\begin{itemize}
  \item \textbf{Manchester City vs Barcelona:}
  \item \textbf{Real Madrid vs Borussia Dortmound:}
  \item \textbf{Argentina vs Suiza:}
\end{itemize}

\subsection{Evaluación del método}

Se evaluó cualitativamente por un operador que los contornos de los jugadores según el algoritmo modificado de contornos activos no correspondían a los jugadores, como es el caso de trabajos citados en el estado del arte (ver \cite{tanos}). Se atribuye esto a la baja calidad de los videos que se pudieron obtener, y la poca capacidad de procesamiento en comparación. (TODO: Completar esto justificando mejor).

Dado esto, se procedió a evaluar cuántas veces sucedía esta discrepancia por unidad de tiempo. Esta métrica, resumida como la cantidad de errores del algoritmo por cada cien cuadros, se intentó minimizar durante el estudio de las variantes de la aplicación.

Otra métrica que se buscó minimizar es el tiempo de demora por cuadro procesado, hasta intentar alcanzar una velocidad de $24$ cuadros por segundo (equivalente a $42$ milisegundos por cuadro), la velocidad de los videos utilizados.
