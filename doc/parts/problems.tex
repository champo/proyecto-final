\chapter{Descripción del Problema}
\label{chap-problems}

Este trabajo busca ofrecer una solución para obtener de manera semi-supervisada
información sobre del estado del juego de fútbol en tiempo real. Esto involucra
conocer la posición de cada jugador para cada cuadro del video, la pelota,
puntaje actual, etcétera. Se acota el problema a la determinación de la
posición de los jugadores únicamente.

% TODO: Les copa este párrafo? Eliminarlo sino
Soluciones más avanzadas podrían otorgar información de mayor nivel,
conociendo datos como la posición de la pelota. Por ejemplo, se pueden
determinar automáticamente posiciones de jugadores fuera de juego, lo cual
requiere, al menos, determinar tres cosas: la posición de cada jugador (y a qué
equipo pertenece), la posición de la pelota, y detección de cuándo un jugador
realiza un pase. Este último punto requiere un análisis de alto nivel que es
descripto en \cite{papers-tanos} y se discuten en el Capítulo \ref{chap-solution}
las dificultades encontradas para replicar estos resultados.

\section{Análisis en Tiempo Real}

Se plantea que el algoritmo debe poder realizar el seguimiento en tiempo
real, tomando ventaja de que contornos activos (la técnica utilizada para
obtener la posición de los jugadores) es lo suficientemente eficiente para
funcionar en tiempo real.

Realizar el seguimiento en tiempo real agrega restricciones fuertes al
problema. Toda técnica tiene un costo de procesamiento, por lo tanto se debe
tener cautela al momento de seleccionar qué procesos de análisis de imagen
pueden ser realizados en los aproximadamente 40 milisegundos que separan un
cuadro de otro al mostrar un video de 24 cuadros por segundo.

\section{Supervisión Humana}

Respecto a la supervición necesaria, el objetivo es igualar y complementar la
información sobre un partido que un operador (o grupo de operadores) podría
obtener de un video.

En el presente trabajo, un supervisor selecciona inicialmente las posiciones
de los jugadores en la cancha. El algoritmo de contornos activos es luego
ejecutado para obtener la posición de cada jugador para los cuadros siguientes.
El supervisor debe corregir eventuales incorrectas actualizaciones de contornos
activos. Es deseable que estas falsas detecciones sean mínimas o inexistentes.

\section{Sistema de Cámaras}

El trabajo se enfoca en imágenes obtenidas de un sistema de una única cámara
fija, posicionada en la cancha de forma tal que todo el campo de juego entra
dentro del cuadro de la cámara. Al utilizar una única cámara, la resolución
adquiere un rol determinante, dificultando o impidiendo el uso de muchas de las
técnicas de seguimiento.

\section{Dificultades}

% TODO: Describir esta sección

\subsection{Corrección de la Perspectiva de la Cámara}

La imagen capturada por la cámara es una representación en 2 dimensiones de la
realidad. El modelo de datos de este trabajo representa cada jugador y la
pelota como un punto en un campo de dos dimensiones, es decir, se descarta el
valor de la altura de cada objeto seguido, ya que no interesa esa información.
Para esto, se aplica una homografía para convertir las coordenadas de un punto
de la imagen a coordenadas en el plano donde se encuentra la cancha.

El cálculo de una homografía involucra reconocer por lo menos 4 puntos de la
cancha en la imagen (ver \cite{homography-estimation}). Esto puede hacerse de
manera supervisada, seleccionando en la imagen proveniente del video distintos
puntos y luego correspondiéndolos con su posición en la cancha en dos
dimensiones. También puede lograrse de manera automática mediante un algoritmo
de deteccion de líneas que permita comparar las líneas en la imagen con las que
se encuentran en la cancha.

Ignorar el valor de altura de los objetos seguidos es una buena aproximación
para los jugadores, pero podría eventualmente causar errores en la medición de
la posición y velocidad de la pelota (ver \cite{Liu20061146}).

\subsection{Sistema de cámaras}
\label{sub-sec:camaras}

Un sistema de múltiples cámaras se ve beneficiado por una mejor resolución y
por lo tanto una mejor precisión al determinar la posición de objetos, pero
requiere un sistema de sincronización que coordine la obtención de información.

Por otro lado, un sistema constituído por una única cámara no tiene la
complejidad extra que implica la sincronización de la información de las
diferentes cámaras, pero sufre de una menor resolución y menor precisión.

Esta reducción de resolución puede tornarse prohibitiva para algunas técnicas o
algoritmos de seguimiento ya que algunos objetos (como por ejemplo la pelota)
tendrán unos pocos píxeles en cada imagen, lo cual dificulta la tarea de
segmentación y seguimiento al contar con una imagen de peor calidad.  Esto
puede observarse en las imágenes de las figuras \ref{fig:barsa3} y
\ref{fig:barsa4}.

\begin{figure}[H]
    \begin{minipage}[t]{.5\textwidth}
        \centering
        \includegraphics[width=.4\linewidth]{./images/resize_barcelona2.png}
        \captionof{figure}{En la imagen se observa la falta de resolución. Los
        bordes se ven difusos.}
        \label{fig:barsa3}
    \end{minipage}%
    \begin{minipage}[t]{.5\textwidth}
        \centering
        \includegraphics[width=.4\linewidth]{./images/resize_barcelona3.png}
        \captionof{figure}{La imagen muestra varias de las dificultades del
          problema: la baja resolución y el
          diminuto tamaño de la pelota con respecto al jugador.  }
        \label{fig:barsa4}
    \end{minipage}
\end{figure}

\subsection{Complejidad del Análisis en Tiempo Real}

Al tener una resolución de \textit{1080p} (aproximadamente dos millones de
píxeles por cuadro), el procesamiento de cada píxel debe tomar a lo sumo 20
nanosegundos.  Para lidiar con esta restricción se puede utilizar información
adicional de la que se disponga respecto al video con el objeto de evitar
procesar píxeles de poca o nula utilidad para el seguimiento. Un ejemplo de esto
es descartar píxeles que estén fuera de la cancha, ya que es probable que la
cámara encuadre más que el campo de juego, abarcando las gradas, el público
espectador, publicidades alrededor del campo de juego, entre otros.

Muchos autores han desarrollado algoritmos automáticos de seguimiento de
objetos en secuencias de imágenes\cite{IFTrace, alp, local-learning, MHT-2}.
Todos ellos están basados en soluciones de ecuaciones diferenciales en
derivadas parciales y resultan aceptablemente robustos, pero tienen severas
restricciones que impiden que se utilicen para aplicaciones en tiempo real.

Se utiliza el algoritmo de contornos activos\cite{fast-level-set}, el cual no
utiliza ecuaciones diferenciales (haciéndolo apto para aplicaciones en tiempo
real) y además hace un análisis local de los objetos seguidos en la imagen, lo
cual hace que el tiempo de análisis de un cuadro sea dependiente de la
resolución de los jugadores e independiente de la resolución del video.

\subsection{Distorsión de la lente}

Al utilizar una única cámara para captar la cancha entera se corre el riesgo de
tener distorción en los puntos de la imagen más alejados al foco de la cámara.
Este es el llamado ``efecto de ojo de buey'' e introduce mucho error, por
ejemplo en la aplicación de la homografía, por lo tanto se debe aplicar una
corrección. Una lente apropiada y bien calibrada puede reducir este error, pero
nunca puede ser eliminado totalmente. Sólo puede evitarse utilizando una mejor
técnica de filmación.

\subsection{Oclusiones entre jugadores}

En un partido es muy común que ocurran oclusiones entre los jugadores. El
sistema debe poder tolerar la oclusión parcial o total de los jugadores.
Esto puede llevar a situaciones muy difíciles de automatizar. Una situación
difícilmente automatizable sucede cuando dos jugadores del mismo equipo (con
vestimenta muy similar) se encuentren alineados con respecto a la cámara.
Se pueden agregar reglas para intentar que el método no se confunda, cuando
los jugadores se separen, quién es quién, pero no hay una solución evidente.

Por ejemplo, se puede utilizar información de cuadros anteriores para
estimar la velocidad de cada uno y estimar sus nuevas posiciones, pero esto
es poco efectivo si los jugadores cambian de velocidad mientras uno ocluye
al otro, o si la velocidad era muy similar al momento de generarse la oclusión.

Otra situación problemática similar es una jugada de córner, donde las
oclusiones entre varios jugadores son muy numerosas, lo que agrega a la
restricción de tiempo real mayor complejidad, ya que la resolución de
oclusiones debe ser muy eficiente en tiempo.

