\documentclass[a4paper,11pt]{report}

\usepackage{fullpage}
\usepackage[utf8]{inputenc}
\usepackage{t1enc}
\usepackage[spanish]{babel}
\usepackage[pdftex,usenames,dvipsnames]{color}
\usepackage[pdftex]{graphicx}
\usepackage{enumerate}
\usepackage{url}
\usepackage{amsmath}
\usepackage{amsfonts}
\usepackage{amssymb}
\usepackage{comment}
\usepackage[table]{xcolor}
\usepackage[small,bf]{caption}
\usepackage{float}
\usepackage{subfig}
\usepackage{bm}
\usepackage{fancyhdr}
\usepackage{times}
\usepackage{titlesec}
\usepackage{csquotes}
\usepackage[backend=bibtex,sorting=none]{biblatex}
\usepackage{titling}
% \usepackage{algorithmicx}
\usepackage{algpseudocode}
\usepackage{algorithm}
\usepackage{letltxmacro}
\usepackage[margin=1cm]{caption}
\usepackage{setspace}

%%%%% BEGIN ALGPSEUDOCODE STUFF %%%%%%
\algdef{SxnE}[FOREACH]{ForEach}{EndFor}[1]{\algorithmicfor\ #1\ \algorithmicdo}
% LEAVES BLANK LINE AT END \algblockdefx[FOREACH]{ForEach}{EndFor}{\textbf{for each }}{}
\algdef{SxnE}[FOR]{For}{EndFor}[1]{\algorithmicfor\ #1\ \algorithmicdo}
\algdef{SxnE}[WHILE]{While}{EndWhile}[1]{\algorithmicwhile\ #1\ \algorithmicdo}
\algdef{SxnE}[IF]{If}{EndIf}[1]{\algorithmicif\ #1\ \algorithmicthen}
\algdef{cxnE}{IF}{Else}{EndIf}

\renewcommand{\algorithmicrequire}{\textbf{Input:}}
\renewcommand{\algorithmicensure}{\textbf{Output:}}
\algnewcommand\algorithmicauxiliary{\textbf{Auxiliary:}}
\algnewcommand\Auxiliary{\item[\algorithmicauxiliary]}

\DefineBibliographyStrings{spanish}{andothers = {et\addabbrvspace al\adddot}}
\renewbibmacro{in:}{}

\floatname{algorithm}{Algoritmo}

\DeclareMathOperator*{\argmin}{arg\,min}
\addbibresource{references}
\DeclareFieldFormat[inbook]{citetitle}{#1}

\newcommand{\norm}[1]{\left\lVert#1\right\rVert}

% \titleformat{\section}{\small\center\bfseries}{\thesection.}{0.5em}{\normalsize\uppercase}
% \titleformat{\subsection}{\small\center\bfseries}{}{0.5em}{\small\uppercase}

\def\customabstract{\vspace{.5em}
    {\Large\center{\textbf{RESUMEN}} \\[0.5em] \relax%
    }}
\def\endkeywords{\par}

\def\keywords{\vspace{.5em}
    {\textit{Palabras clave: }
    }}
\def\endkeywords{\par}

% TITLE Configuration
\setlength{\droptitle}{-30pt}
\pretitle{\begin{center}\Huge\begin{rmfamily}}
\posttitle{\par\end{rmfamily}\end{center}\vskip 0.5em}
\preauthor{\begin{center}
        \large \lineskip 0.5em%
\begin{tabular}[t]{c}}
\postauthor{\end{tabular}\normalsize
    \\[1em] Instituto Tecnológico de Buenos Aires
\par\end{center}}
\predate{\begin{center}\small}
\postdate{\par\end{center}}

% Headers
\addtolength{\voffset}{-40pt}
\addtolength{\textheight}{80pt}
\renewcommand{\headrulewidth}{0pt}
\fancyhead{}
\fancyfoot{}
\lhead{\small } % No publicado
\rhead{\small \thepage}
\cfoot{\small Copyright \copyright 2013 ITBA}

% Next 4 lines allow for parts with independet sections counters
% and correct referencing between parts

% Metadata
\title{Análisis Automático de Jugadas en Partidos de Fútbol}
\date{1 de agosto de 2014}
\author{Civile, Juan Pablo \and Crespo, Álvaro \and Ordano, Esteban }


\begin{document}
\pagestyle{fancy}
\maketitle
\thispagestyle{fancy}

\begin{customabstract}
\begin{doublespace}
El seguimiento de jugadores en una secuencia de imágenes (video) es un caso
particular del seguimiento de objetos, aplicado a partidos de fútbol. El
objetivo es detectar y seguir la posición de los jugadores y la pelota a medida
que se desarrolla el juego, obteniendo información útil para distintas
aplicaciones, como puede ser soporte informático para los árbitros (detección
automática de pases, goles, posiciones adelantadas, etc...), adaptación del
entrenamiento de los jugadores, análisis y estudio de las tácticas de un
contricante, entre otros.

En el presente trabajo, se analizan las dificultades de realizar el seguimiento
de jugadores en tiempo real con reducida o nula colaboración humana a partir de
una sola fuente de video que consiste en una cámara de alta resolución (HD)
fija capaz de encuadrar todo el campo de juego. 

En base al analisis de dificultades, se selecciona una tecnica de seguimiento,
y se detalla un algoritmo construido sobre esta para la tarea especifica de 
detección y seguimiento de jugadores en futbol. Finalmente, se presentan 
resultados de aplicar este algoritmo, y que información podria proveer para el
analisis automatico de jugadas.
\end{doublespace}
\end{customabstract}

\tableofcontents

\chapter{Introducción}

Dentro del campo de análisis y tratamiento de imágenes un problema que se
estudia es el del reconocimiento y seguimiento de objetos en secuencias de
imágenes. Identificar correctamente un objeto en un video es útil para un
amplio rango de aplicaciones, desde aplicaciones médicas (dónde se ofrece
soporte para diagnósticos y cirugías) hasta la industria cinematográfica
(captura de movimiento, \textit{post-producción}), pasando por vigilancia con
cámaras. 

En el ámbito deportivo se encuentran muchas aplicaciones para el seguimiento de
objetos. Puede resultar útil para validar o reemplazar las decisiones de los
jueces o árbitros del partido, permitir a deportistas de alto rendimiento
analizar y mejorar sus movimientos, ayudar a los entrenadores a decidir
estrategias y evaluar a los deportistas, otorgar estadísticas a fanáticos del
juego, entre otras aplicaciones.

En este trabajo se estudia el problema de realizar el seguimiento de todos los
jugadores de un partido de fútbol, utilizando una única cámara, con su posición
fija y que abarca todo el campo de juego. Se busca que el seguimiento sea en
tiempo real, es decir, que el procesamiento sea lo suficientemente rápido para
que se vean los resultados mientras se ejecuta el video.

Se plantea el uso de una técnica de seguimiento existente, \textit{contornos
activos}, para el seguimiento en tiempo real de los jugadores con una única
computadora. Se investigaron alternativas para mejorar el seguimiento
específicas a las características de un video de fútbol.

%% TODO champo como redactamos la parte de "En tal sección se habla de X, en tal de Y". Digo por lo de las 2 partes que tenemos. Tipo, esto está bien???
%% En el texto aparecerían como Sección II.1, Sección II.2, etc...
El trabajo se divide en dos partes: en la primera se presenta el estado del arte respecto a
técnicas aplicables al problema de seguimiento de múltiples jugadores de fútbol
mediante el uso de una o varias cámaras de video, así como las bases teóricas
para estos métodos, características y limitaciones de cada uno, y la
factibilidad de conocer el estado del juego completo en cada instante de
tiempo; en la segunda se detallan las diferentes secciones
del presente trabajo. En la segunda parte, la Sección \ref{sec:problems} se realiza la descripción detallada del problema, en la Sección
\ref{sec:solution} se describen las mejoras y novdedades implementadas, en la Sección \ref{sec:results} se presentan los resultados obtenidos y en
la Sección \ref{sec:iftrace} se realiza un comparación con el famoso algorimo de seguimiento de objetos IFTrace. Finalmente, en la Sección
\ref{sec:conclusion} se detallan las conclusiones del trabajo.


\newpage

\chapter{Estado del arte}

\documentclass[twocolumn,a4paper,10pt]{article}

\usepackage[utf8]{inputenc}
\usepackage{t1enc}
\usepackage[spanish]{babel}
\usepackage[pdftex,usenames,dvipsnames]{color}
\usepackage[pdftex]{graphicx}
\usepackage{enumerate}
\usepackage{url}
\usepackage{amsmath}
\usepackage{amsfonts}
\usepackage{amssymb}
\usepackage[table]{xcolor}
\usepackage[small,bf]{caption}
\usepackage{float}
\usepackage{subfig}
\usepackage{bm}
\usepackage{fancyhdr}
\usepackage{times}
\usepackage{titlesec}
\usepackage[numbers]{natbib}
\usepackage{titling}

\titleformat{\section}{\small\center\bfseries}{\thesection.}{0.5em}{\normalsize\uppercase}
\titleformat{\subsection}{\small\center\bfseries}{}{0.5em}{\small\uppercase}
\renewcommand{\bibsection}{}

% TITLE Configuration
\setlength{\droptitle}{-30pt}
\pretitle{\begin{center}\Huge\begin{rmfamily}}
\posttitle{\par\end{rmfamily}\end{center}\vskip 0.5em}
\preauthor{\begin{center}
        \large \lineskip 0.5em%
\begin{tabular}[t]{c}}
\postauthor{\end{tabular}\normalsize 
    \\[1em] Estudiantes del Instituto Tecnológico de Buenos Aires
\par\end{center}}
\predate{\begin{center}\small}
\postdate{\par\end{center}}

% Headers
\addtolength{\voffset}{-40pt}
\addtolength{\textheight}{80pt}
\renewcommand{\headrulewidth}{0pt}
\fancyhead{}
\fancyfoot{}
\lhead{\small No publicado}
\rhead{\small \thepage}
\cfoot{\small Copyright \copyright 2013 ITBA}

% Metadata
\title{}
\date{20 de Septiembre de 2013}
\author{Civile, Juan Pablo \and Crespo, Álvaro \and Ordano, Esteban }

\begin{document}

\pagestyle{fancy}
\maketitle
\thispagestyle{fancy}

% We don't want this right now, right?
%\begin{customabstract}
%\textbf{
%}
%\end{customabstract}
%
%\begin{keywords}
%\end{keywords}

\section{Estado del Arte}

% * Real time tracking
%   sarasarsara (LP, IFTrace, etc)
%   * Tracking human poses
% * Football-specific papers
%   * Players papers
%   * Ball papers
% * Paper de lo' tano'

\section*{Referencias}
\begin{thebibliography}{99}
\end{thebibliography}

\end{document}


\newpage

\chapter{Descripción del problema}
\label{sec:problems}

%%TODO acá falta un descripción general del problema de un par de parrafos, lo que
%% tenemos abajo son las dificultades y problemas particulares pero nunca introducimos
%% el problema en esta sección.

\section{Análisis en Tiempo Real}

Se planteó la meta de que el algoritmo pueda realizar el seguimiento en tiempo
real, aprovechando que contornos activos, la técnica utilizada para obtener la
posición de los jugadores es lo suficientemente eficiente para funcionar en
tiempo real.

Realizar el seguimiento en tiempo real agrega restricciones fuertes al
problema. Toda técnica tiene un costo de procesamiento, por lo tanto se debe
tener cautela al momento de seleccionar qué procesos de análisis de imagen
pueden ser realizados en los aproximadamente 40 milisegundos que separan un
cuadro de otro al mostrar un video de 24 cuadros por segundo.

\section{Supervisión Humana}

% TODO: BULLLLLSHITTTTTT los tanos no tenian 100% automatico?
% Esteban says: No, los tanos tenían fallas en su algoritmo aunque creo
% que era solo con el seguimiento de la pelota. 80% accuracy

% Ahí leí de nuevo y si lo tienen resuelto, sorry.

Respecto a la supervición necesaria, el objetivo que se planteó es igualar y
complementar la información sobre un partido que un operador (o grupo de
operadores) podría obtener de un video.

En el presente trabajo, un supervisor seleccionará inicialmente las posiciones
de los jugadores en la cancha. El algoritmo de contornos activos será
ejecutado para obtener la posición de cada jugador para los cuadros siguientes.
El supervisor corregirá eventuales incorrectas actualizaciones de contornos
activos. Se desea minimizar estas falsas detecciones.

\section{Sistema de Cámaras}

Nuestro enfoque consta de un sistema de una única cámara fija, posicionada en
la cancha de tal forma que pueda observar toda la cancha en un solo cuadro. Al
utilizar una única cámara, la resolución adquiere un rol determinante,
dificultando o impidiendo el uso de muchas de las técnicas de seguimiento.

\section{Dificultades}

\subsection{Corregir la Perspectiva de la Cámara}

La imagen capturada por la cámara es una representación en 2 dimensiones de
la realidad. Nuestro modelo de datos representa cada jugador y la pelota
como un punto en un campo de dos dimensiones, es decir, se descarta el valor
de la altura de cada objeto seguido, ya que no interesa esa información.
Para esto, se aplica una homografía para convertir las coordenadas de un punto de
la imagen a coordenadas en el plano donde se encuentra la cancha.

El calculo de una homografia involucra reconocer por lo menos 4 puntos de la cancha
en la imagen (\cite{homography-estimation}). Esto puede hacerse de manera supervisada, se selecciona en la imagen
puntos y luego se dice a que punto de la cancha corresponden. O puede hacerse de
manera automática mediante un algoritmo de deteccion de líneas que permita
comparar las líneas en la imagen con las que se encuentran en la cancha.

Ignorar el valor de altura de los objetos seguidos es una buena aproximación
para los jugadores, pero podría eventualmente causar errores en la medición de
la posición y velocidad de la pelota (\cite{Liu20061146}).

\subsection{Sistema de cámaras}

Un sistema de múltiples cámaras se ve beneficiado por una mejor resolución y
por lo tanto una mejor precisión al determinar la posición de objetos, pero
requiere un sistema de sincronización que coordine la obtención de información.

Por otro lado, un sistema constituído por una única cámara no tiene la
complejidad extra que implica la sincronización de la información de las
diferentes cámaras, pero sufre de una menor resolución y menor precisión.

Esta reducción de resolución puede tornarse prohibitiva para algunas técnicas o
algoritmos de seguimiento ya que algunos objetos (como por ejemplo la pelota)
tendrán unos pocos pixels en cada imagen, lo cual dificulta la tarea de
segmentación y seguimiento al contar con una imagen de peor calidad.  Esto
puede observarse en las imágenes de las figuras \ref{fig:barsa3} y
\ref{fig:barsa4}.

\begin{figure}[H]
    \begin{minipage}[t]{.5\textwidth}
        \centering
        \includegraphics[width=.4\linewidth]{./images/resize_barcelona2.png}
        \captionof{figure}{En la imagen se observa la falta de resolución. Los
        bordes se ven difusos.}
        \label{fig:barsa3}
    \end{minipage}%
    \begin{minipage}[t]{.5\textwidth}
        \centering
        \includegraphics[width=.4\linewidth]{./images/resize_barcelona3.png}
        \captionof{figure}{La imagen muestra varias de las dificultades del
          problema: la baja resolución y el
          diminuto tamaño de la pelota con respecto al jugador.  }
        \label{fig:barsa4}
    \end{minipage}
\end{figure}

\subsection{Complejidad del Análisis en Tiempo Real}

Al tener una resolución de \textit{1080p} (aproximadamente dos millones de
pixels por cuadro), el procesamiento de cada pixel debe tomar a lo sumo 20
nanosegundos.  Para lidiar con esta restricción se puede utilizar información
adicional de la que se disponga respecto al video con el objeto de evitar
procesar pixels de poca o nula utilidad para el seguimiento. Un ejemplo de esto
es descartar pixels que estén fuera de la cancha, ya que es probable que la
cámara encuadre más que el campo de juego, abarcando las gradas, el público
espectador, publicidades alrededor del campo de juego, entre otros.

Muchos autores han desarrollado algoritmos automáticos de seguimiento de
objetos en secuencias de imágenes\cite{IFTrace, alp, local-learning, MHT-2}.
Todos ellos están basados en soluciones de ecuaciones diferenciales en
derivadas parciales y resultan aceptablemente robustos, pero tienen severas
restricciones que impiden que se utilicen para aplicaciones en tiempo real.

Nuestra investigación utiliza el algoritmo de contornos
activos\cite{fast-level-set}, el cual no utiliza ecuaciones diferenciales
(haciéndolo apto para aplicaciones en tiempo real) y además hace un análisis
local de los objetos seguidos en la imagen, lo cual hace que el tiempo de
análisis de un cuadro sea dependiente de la resolución de los jugadores e
independiente de la resolución del video.

\subsection{Distorsión de la lente}

Al utilizar una única cámara para captar la cancha entera se corre el riesgo de
tener distorción en los puntos de la imagen más alejados al foco de la cámara.
Este es el llamado ``efecto de ojo de buey'' e introduce mucho error, por ejemplo
en la aplicación de la homografía, por lo
tanto se debe aplicar una corrección. Una lente apropiada y bien calibrada
puede reducir este error, pero nunca puede ser eliminado totalmente. Sólo
puede evitarse utilizando una mejor técnica de filmación.

\subsection{Oclusiones entre jugadores}

En un partido es muy común que ocurran oclusiones entre los jugadores. El
sistema debe poder tolerar la oclusión parcial o total de los jugadores.
Esto puede llevar a situaciones muy difíciles de automatizar. Una situación
difícilmente automatizable sucede cuando dos jugadores del mismo equipo (con
vestimenta muy similar) se encuentren alineados con respecto a la cámara.
Se pueden agregar reglas para intentar que el método no se confunda, cuando
los jugadores se separen, quién es quién, pero no hay una solución evidente.

Por ejemplo, se puede utilizar información de cuadros anteriores para
estimar la velocidad de cada uno y estimar sus nuevas posiciones, pero esto
es poco efectivo si los jugadores cambian de velocidad mientras uno ocluye
al otro, o si la velocidad era muy similar al momento de generarse la oclusión.

Otra situación problemática similar es una jugada de córner, donde las
oclusiones entre varios jugadores son muy numerosas, lo que agrega a la
restricción de tiempo real mayor complejidad, ya que la resolución de
oclusiones debe ser muy eficiente en tiempo.



\section{Solución propuesta}

En esta sección describimos las distintas tecnicas aplicadas. Primero describimos algoritmos para
ignorar los elementos del fondo de la imagen en \ref{sec:background-elimination}. Luego, explicamos
como utilizamos el algoritmo contornos activos\cite{fast-level-set} y las modificaciónes necesarias
en \ref{sec:ac}. Finalmente, detallamos la integración final entre las distintas tecnicas, que es
la que analizamos en la siguiente seccion \ref{sec:resultados}.

\subsection{Eliminación de fondo por varios métodos}
\label{sec:background-elimination}
Problemas y soluciones (tal vez como subsubsection)

\subsubsection{Tribuna y publicidades}
% Simplemente como ponemos en negro para que no interfiera

\subsubsection{Energia}
  * Método de la energía

\subsubsection{Eliminación de lineas}
  * Eliminación de lineas por Detector de borde + umbral + morfología

\subsubsection{Eliminación del pasto}
  * Eliminación de colores de cancha por histograma de colores


\subsection{Contornos activos}
\label{sec:ac}

- Características

  * Colores del jugador
  * Aprendizaje
  * Múltiples características
  * Selección de máximos en histograma con bfs

- Descriptores

\subsection{Algoritmo final}
\label{sec:alg-final}
% Que onda esta seccion? Me parece medio tirada de los pelos la idea

- Eliminación de lineas por detector de borde + umbral + morfología
- Múltiples características
- Descriptores
- Imágenes de cómo resolvemos algunos casos extremos

- Opcional para equipos difíciles de detectar: seguimiento por complemento
  - eliminación de colores de cancha por histograma + ronda de contornos

- Que lastima que no tuvimos trackeo de pelota porque hubiese dado mejores datos


\chapter{Resultados}
\label{chap-results}

Esta Sección presenta un análisis de los resultados obtenidos de la ejecución del algoritmo sobre distintos videos. A lo largo de est Sección se refiere al algoritmo como a la modificación que se le hizo a \textit{contornos activos} y que fue descripta en el Capítulo \ref{chap-solution}. Las palabras \textit{aplicación} y \textit{programa} serán utilizadas indistintamente para referirse al programa que implementa dicho algoritmo, junto con una interfaz de usuario para el operador. Se refiere a la\textit{implementación} como a la parte del programa que implementa el algoritmo.

Se presentan los resultados obtenidos de la ejecución del programa, junto con métricas utilizadas para evaluar la performance de la implementación.

\section{Aplicación}

El programa incorpora una interfaz gráfica utilizada tanto para dar instrucciones como para recibir información acerca del partido y del funcionamiento del algoritmo. La misma cuenta con una sección donde se muestra una lista de jugadores siendo seguidos, otra donde se puede visualizar el video original con un recuadro en cada jugador seguido, un mapa de calor de espacios ocupados por los jugadores, y una imágen que permite al operador evaluar el correcto funcionamiento del algoritmo.

Para comenzar la ejecución, la implementación requiere que el operador identifique la posición de los jugadores en la imagen inicial de la secuencia. En este paso, también debe agregar información acerca del número de camiseta que viste, el equipo, y nombre del jugador.

Un paso adicional que debe ejecutar el operador es determinar puntos a utilizar para el cálculo de la homografía.

% TODO: Esto no se entiende, y no lo explicamos antes????
Los datos necesarios son al menos cuatro parejas de puntos. De estas parejas, el primer punto es una coordenada la imagen en perspectiva y el otro punto es la coordenada en un plano bidimensional. Con cuatro de estas relaciones, se puede calcular la matriz que resuelve la homografía para cualquier otro punto.  

Una vez que el programa tiene estos datos, el algoritmo de contornos activos puede comenzar a correr. Existen dos modos, uno que avanza sólo ante la indicación del operador (mecánica ``cuadro por cuadro'') y otro modo que avanza automáticamente cada vez que se computa un cuadro (mecánica ``tiempo real'').  % [ TODO: algo de amor a este párrafo ]

Para cada cuadro, una ventana muestra las posiciones actualizadas de los jugadores. El usuario puede seleccionar un jugador en particular y ver en detalle información sobre su posición, velocidad promedio, actual, y máxima desde el comienzo del seguimiento, y un mapa de calor que muestra con colores próximos al rojo los lugares más visitados por el jugador. Además, en un archivo se escribe la posición de cada jugador en cada momento.

De manera opcional, en cada cuadro se guarda en disco duro una copia del estado actual del seguimiento para referencia futura.

\section{Material utilizado}

Se utilizaron principalmente dos videos, uno correspondiente a un partido entre los equipos argentinos de los clubes Boca Juniors e Independiente; y un segundo video en el cual se enfrentan los equipos Independiente y San Lorenzo. Se detalla a continuación las características de las imágenes extraídas de esos videos.

\begin{itemize}
  \item \textbf{Boca vs Independiente:} El video cuenta con una resolución de \textit{1080p} (1920 pixels de ancho y 1080 de alto). La cancha se muestra en su totalidad, y se puede observar la totalidad de las gradas del lado opuesto y cielo por encima de ellas. Luego de descartar esas regiones del video, la resolución pasa a ser de 1459 píxels de ancho por 304 de alto. % TODO: No está terminado, falta describir jugadores y eso
  \item \textbf{Independiente vs San Lorenzo:} También filmado en resolución de \textit{1080p}, las esquinas del campo de juego quedan fuera del campo visual. Luego de descartar las gradas, la resolución final del video es de 1920 pixels de ancho y 540 de alto. % TODO: No está terminado, describir jugadores
\end{itemize}

Además, se tomaron pequeños cortes de tres videos de fútbol televisado en los cuales la cámara se encuentra relativamente estática y se analizó la correctitud del seguimiento en estos casos, contando con mayor resolución pero sin poder efectuar exitosamente la eliminación de fondo o la correlación a un plano bidimencional por homografía % [ TODO: ESTO ESTÁ ESCRITO CON LA ** ]

\begin{itemize}
  \item \textbf{Manchester City vs Barcelona:}
  \item \textbf{Real Madrid vs Borussia Dortmound:}
  \item \textbf{Argentina vs Suiza:}
\end{itemize}

\section{Evaluación del método}

Se evaluó cualitativamente por un operador que los contornos de los jugadores según el algoritmo modificado de contornos activos no correspondían a los jugadores, como es el caso de trabajos citados en el estado del arte (ver \cite{papers-tanos}). Se atribuye esto a la baja calidad de los videos que se pudieron obtener, y la poca capacidad de procesamiento en comparación. (TODO: Completar esto justificando mejor).

Dado esto, se procedió a evaluar cuántas veces sucedía esta discrepancia por unidad de tiempo. Esta métrica, resumida como la cantidad de errores del algoritmo por cada cien cuadros, se intentó minimizar durante el estudio de las variantes de la aplicación.

Otra métrica que se buscó minimizar es el tiempo de demora por cuadro procesado, hasta intentar alcanzar una velocidad de $24$ cuadros por segundo (equivalente a $42$ milisegundos por cuadro), la velocidad de los videos utilizados.

% TODO chequear la forma de referirse a NUESTRO algoritmo sin decir la palabra NUESTRO :s
\section{Comparación con IFTrace}

El algoritmo de seguimiento IFTrace, propuesto por \citeauthor*{IFTrace}, es un algoritmo robusto que soporta
cambios de iluminación y forma, oclusiones y se centra en hallar características representativas de la textura
de los objetos a seguir. Es capaz de recuperarse de errores menores y permite el seguimiento de múltiples objetos
a la vez.

Un algoritmo de este tipo podría proporcionar una solución al problema. Para comprobarlo, se llevaron a cabo algunas pruebas
utilizando un video sintético creado para este fin, y un video real de un partido de fútbol. En las Figuras \ref{fig:sample-happy-occluded}
y \ref{fig:sample-boca} pueden observarse los primeros cuadros de cada video. Pueden apreciarse a simple vista las marcadas diferencias
entre ambos videos, como por ejemplo la resolución de la imagen y el tamaño y la complejidad de los objetos de inteŕes.

\begin{figure}[H]
    \centering
    \includegraphics[width=\linewidth]{./images/sample_happy_occluded.png}
    \caption{Muestra de un cuadro del video sintético de prueba.}
    \label{fig:sample-happy-occluded}
\end{figure}

\begin{figure}[H]
    \centering
    \includegraphics[width=\linewidth]{./images/sample_boca.png}
    \caption{Muestra de un cuadro del video real de un partido de fútbol.}
    \label{fig:sample-boca}
\end{figure}

Como se puede ver en la Figura
\ref{fig:happy-occluded-iftrace}, IFTrace logra un correcto seguimiento de múltiples objetos en el video sintético.
También puede observarse, en la Figura \ref{fig:boca-iftrace}, como sigue correctamente a un jugador en el video real.
Sin embargo, el seguimiento sólo es exitoso durante unos pocos cuadros, ya que, en el cuadro 17, el algoritmo cae en un error del
cual sólo una corrección manual
puede sacarlo. Este tipo de correción semi-supervisada no está contemplada en el algoritmo de IFTrace.

\begin{figure}[H]
    \centering
    \begin{tabular}{cccc}
        \subfloat[Cuadro 1]{\includegraphics[width = 1.5in]{./images/cropped_happy_occluded_00001.png}} &
        \subfloat[Cuadro 5]{\includegraphics[width = 1.5in]{./images/cropped_happy_occluded_00005.png}} &
        \subfloat[Cuadro 8]{\includegraphics[width = 1.5in]{./images/cropped_happy_occluded_00008.png}} &
        \subfloat[Cuadro 12]{\includegraphics[width = 1.5in]{./images/cropped_happy_occluded_00012.png}}
    \end{tabular}
    %% NASTY hack to make refernce work with figures and subfigures, put \label inside \caption env, little bird told me
    \caption{IFTrace funcionando en una secuencia de cuadros de video sintético.
    \label{fig:happy-occluded-iftrace}}
\end{figure}

\begin{figure}[H]
    \centering
    \begin{tabular}{cccc}
        \subfloat[Cuadro 9]{\includegraphics[width = 1.5in]{./images/cropped_boca_00009.png}} &
        \subfloat[Cuadro 12]{\includegraphics[width = 1.5in]{./images/cropped_boca_00012.png}} &
        \subfloat[Cuadro 14]{\includegraphics[width = 1.5in]{./images/cropped_boca_00014.png}} &
        \subfloat[Cuadro 17]{\includegraphics[width = 1.5in]{./images/cropped_boca_00017.png}}
    \end{tabular}
    %% NASTY hack to make refernce work with figures and subfigures, put \label inside \caption env, little bird told me
    \caption{Seguimiento de un jugador en un video real utilizando IFTrace.
    \label{fig:boca-iftrace}}
\end{figure}

\begin{figure}[H]
    \centering
    \begin{tabular}{cccc}
        \subfloat[Cuadro 1]{\includegraphics[width = 1.5in]{./images/cropped_processing2.png}} &
        \subfloat[Cuadro 5]{\includegraphics[width = 1.5in]{./images/cropped_processing5.png}} &
        \subfloat[Cuadro 8]{\includegraphics[width = 1.5in]{./images/cropped_processing14.png}} &
        \subfloat[Cuadro 12]{\includegraphics[width = 1.5in]{./images/cropped_processing25.png}}
    \end{tabular}
    %% NASTY hack to make refernce work with figures and subfigures, put \label inside \caption env, little bird told me
    \caption{Nuestro algoritmo en funcionamiento en un video sintético.
    \label{fig:happy-occluded-activeContour}}
\end{figure}

\begin{figure}[H]
    \centering
    \begin{tabular}{cccc}
        \subfloat[Cuadro 2]{\includegraphics[width = 1.5in]{./images/cropped_rendered002.png}} &
        \subfloat[Cuadro 12]{\includegraphics[width = 1.5in]{./images/cropped_rendered007.png}} &
        \subfloat[Cuadro 14]{\includegraphics[width = 1.5in]{./images/cropped_rendered012.png}} &
        \subfloat[Cuadro 17]{\includegraphics[width = 1.5in]{./images/cropped_rendered017.png}}
    \end{tabular}
    %% NASTY hack to make refernce work with figures and subfigures, put \label inside \caption env, little bird told me
    \caption{Seguimiento de los jugadores en un video real mediante nuestro algoritmo.
    \label{fig:boca-activeContour}}
\end{figure}

Como se puede observar en la Figura \ref{fig:happy-occluded-activeContour}, nuestro algoritmo logra seguir con éxito a los objetos
de interés en el video sintético. Además, también se obtiene un resultado positivo en el video real en la situación en que IFTrace
pierde al jugador, como puede observarse en la Figura \ref{fig:boca-activeContour}.

\subsection{Evaluación de comportamiento}

Otro punto importante de comparación entre los dos algoritmos es su tiempo de ejecución, es decir el tiempo que tarda en llevar a cabo su trabajo.
De acuerdo a las mediciones realizadas con un video real de un partido de fútbol, siguiendo a un solo jugador, el tiempo promedio que
tarda IFTrace por cuadro es 6.962 segundos, mientras que nuestro algoritmo tiene un tiempo promedio de 0.712 segundos. Se puede observar que
%% 6.9628571428571435 si quieren los decimales
%% TODO verificar nuestro numero!!!
se encuentra un orden magnitud por debajo de IFTrace.

Cabe destacar que ambos algoritmos podrían verse beneficiados por numerosas optimizaciones, como por ejemplo la programación
en GPU y la reducción de operaciones de I/O, \textit{Input/Output}, como lo son las escrituras a disco. Sin embargo, por razones de
practicidad se tomaron estas mediciones con versiones estándar de ambos algoritmos.

% TODO Esto creo que probablemente terminemos sacandolo right? Que métrica podemos meter?!?!
% - Comparación de nuestros resultados vs los de ellos (Métrica: cantidad de jugadores perdidos por frame -- o si se nos ocurre una mejor ambas o esa)



% TODO chequear la forma de referirse a NUESTRO algoritmo sin decir la palabra NUESTRO :s
\section{Comparación con IFTrace}

El algoritmo de seguimiento IFTrace, propuesto por \citeauthor*{IFTrace}, es un algoritmo robusto que soporta
cambios de iluminación y forma, oclusiones y se centra en hallar características representativas de la textura
de los objetos a seguir. Es capaz de recuperarse de errores menores y permite el seguimiento de múltiples objetos
a la vez.

Un algoritmo de este tipo podría proporcionar una solución al problema. Para comprobarlo, se llevaron a cabo algunas pruebas
utilizando un video sintético creado para este fin, y un video real de un partido de fútbol. En las Figuras \ref{fig:sample-happy-occluded}
y \ref{fig:sample-boca} pueden observarse los primeros cuadros de cada video. Pueden apreciarse a simple vista las marcadas diferencias
entre ambos videos, como por ejemplo la resolución de la imagen y el tamaño y la complejidad de los objetos de inteŕes.

\begin{figure}[H]
    \centering
    \includegraphics[width=\linewidth]{./images/sample_happy_occluded.png}
    \caption{Muestra de un cuadro del video sintético de prueba.}
    \label{fig:sample-happy-occluded}
\end{figure}

\begin{figure}[H]
    \centering
    \includegraphics[width=\linewidth]{./images/sample_boca.png}
    \caption{Muestra de un cuadro del video real de un partido de fútbol.}
    \label{fig:sample-boca}
\end{figure}

Como se puede ver en la Figura
\ref{fig:happy-occluded-iftrace}, IFTrace logra un correcto seguimiento de múltiples objetos en el video sintético.
También puede observarse, en la Figura \ref{fig:boca-iftrace}, como sigue correctamente a un jugador en el video real.
Sin embargo, el seguimiento sólo es exitoso durante unos pocos cuadros, ya que, en el cuadro 17, el algoritmo cae en un error del
cual sólo una corrección manual
puede sacarlo. Este tipo de correción semi-supervisada no está contemplada en el algoritmo de IFTrace.

\begin{figure}[H]
    \centering
    \begin{tabular}{cccc}
        \subfloat[Cuadro 1]{\includegraphics[width = 1.5in]{./images/cropped_happy_occluded_00001.png}} &
        \subfloat[Cuadro 5]{\includegraphics[width = 1.5in]{./images/cropped_happy_occluded_00005.png}} &
        \subfloat[Cuadro 8]{\includegraphics[width = 1.5in]{./images/cropped_happy_occluded_00008.png}} &
        \subfloat[Cuadro 12]{\includegraphics[width = 1.5in]{./images/cropped_happy_occluded_00012.png}}
    \end{tabular}
    %% NASTY hack to make refernce work with figures and subfigures, put \label inside \caption env, little bird told me
    \caption{IFTrace funcionando en una secuencia de cuadros de video sintético.
    \label{fig:happy-occluded-iftrace}}
\end{figure}

\begin{figure}[H]
    \centering
    \begin{tabular}{cccc}
        \subfloat[Cuadro 9]{\includegraphics[width = 1.5in]{./images/cropped_boca_00009.png}} &
        \subfloat[Cuadro 12]{\includegraphics[width = 1.5in]{./images/cropped_boca_00012.png}} &
        \subfloat[Cuadro 14]{\includegraphics[width = 1.5in]{./images/cropped_boca_00014.png}} &
        \subfloat[Cuadro 17]{\includegraphics[width = 1.5in]{./images/cropped_boca_00017.png}}
    \end{tabular}
    %% NASTY hack to make refernce work with figures and subfigures, put \label inside \caption env, little bird told me
    \caption{Seguimiento de un jugador en un video real utilizando IFTrace.
    \label{fig:boca-iftrace}}
\end{figure}

\begin{figure}[H]
    \centering
    \begin{tabular}{cccc}
        \subfloat[Cuadro 1]{\includegraphics[width = 1.5in]{./images/cropped_processing2.png}} &
        \subfloat[Cuadro 5]{\includegraphics[width = 1.5in]{./images/cropped_processing5.png}} &
        \subfloat[Cuadro 8]{\includegraphics[width = 1.5in]{./images/cropped_processing14.png}} &
        \subfloat[Cuadro 12]{\includegraphics[width = 1.5in]{./images/cropped_processing25.png}}
    \end{tabular}
    %% NASTY hack to make refernce work with figures and subfigures, put \label inside \caption env, little bird told me
    \caption{Nuestro algoritmo en funcionamiento en un video sintético.
    \label{fig:happy-occluded-activeContour}}
\end{figure}

\begin{figure}[H]
    \centering
    \begin{tabular}{cccc}
        \subfloat[Cuadro 2]{\includegraphics[width = 1.5in]{./images/cropped_rendered002.png}} &
        \subfloat[Cuadro 12]{\includegraphics[width = 1.5in]{./images/cropped_rendered007.png}} &
        \subfloat[Cuadro 14]{\includegraphics[width = 1.5in]{./images/cropped_rendered012.png}} &
        \subfloat[Cuadro 17]{\includegraphics[width = 1.5in]{./images/cropped_rendered017.png}}
    \end{tabular}
    %% NASTY hack to make refernce work with figures and subfigures, put \label inside \caption env, little bird told me
    \caption{Seguimiento de los jugadores en un video real mediante nuestro algoritmo.
    \label{fig:boca-activeContour}}
\end{figure}

Como se puede observar en la Figura \ref{fig:happy-occluded-activeContour}, nuestro algoritmo logra seguir con éxito a los objetos
de interés en el video sintético. Además, también se obtiene un resultado positivo en el video real en la situación en que IFTrace
pierde al jugador, como puede observarse en la Figura \ref{fig:boca-activeContour}.

\subsection{Evaluación de comportamiento}

Otro punto importante de comparación entre los dos algoritmos es su tiempo de ejecución, es decir el tiempo que tarda en llevar a cabo su trabajo.
De acuerdo a las mediciones realizadas con un video real de un partido de fútbol, siguiendo a un solo jugador, el tiempo promedio que
tarda IFTrace por cuadro es 6.962 segundos, mientras que nuestro algoritmo tiene un tiempo promedio de 0.712 segundos. Se puede observar que
%% 6.9628571428571435 si quieren los decimales
%% TODO verificar nuestro numero!!!
se encuentra un orden magnitud por debajo de IFTrace.

Cabe destacar que ambos algoritmos podrían verse beneficiados por numerosas optimizaciones, como por ejemplo la programación
en GPU y la reducción de operaciones de I/O, \textit{Input/Output}, como lo son las escrituras a disco. Sin embargo, por razones de
practicidad se tomaron estas mediciones con versiones estándar de ambos algoritmos.

% TODO Esto creo que probablemente terminemos sacandolo right? Que métrica podemos meter?!?!
% - Comparación de nuestros resultados vs los de ellos (Métrica: cantidad de jugadores perdidos por frame -- o si se nos ocurre una mejor ambas o esa)


\section{Conclusiones}

- Mal material, imposible agarrar la pelota y difícil los jugadores
- Resultados aceptables


\printbibliography

%% TODO get rid of this before final deadline
\begin{comment}

- Algoritmo de deteccion de fondo: cosas que no son la cancha en el video arruinan las mediciones
  => Poner en negro toda parte del video que no sea la cancha

- Funcion de feature de contornos activos
  + Tomar promedio de color del contorno inicial no soporta camisetas rayadas
  + Tampoco aguanta camisetas con mucha iluminacion
  => Busqueda de nuevos features
    - sigma
    - coeficiente de variacion: ???
    - distintos color-spaces: algunos color spaces funcionan mejor para un tipo de camiseta que otro. seria necesario tener varios features distintos y saber seleccionar el mejor

- Distorcion de la lente introduce error en la homografia
  => Algoritmos de correcion de la lente
     - El factor de correcion es distinto para cada video
     - Es dificil de calcular programaticamente

- Jugadores distantes se borronean mucho
  => ???

- La pelota es muy chica
  => ???

- Cuando un jugador marca a otro, suele recorrer mucha distancia oculto o semi oculto
  => ???

- Además de distorción de la lente, la RESOLUCION: al agarrar TODA la cancha en una toma los jugadores MUY chicos, y la pelota más
    => ???

- Las canchas tienen diferentes medidas (varían en un cierto rango). Esto afecta a la homografía
    => Adaptar la homografía a las dimensiones de cada cancha.

- Como relacionar posiciones relativas de los jugadores teniendo una vista 3D con perspectiva
    => Homografía
    (Muy básico?)

\end{comment}

\end{document}
