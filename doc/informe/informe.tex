\documentclass[a4paper,10pt]{article}

\usepackage[utf8]{inputenc}
\usepackage{t1enc}
\usepackage[spanish]{babel}
\usepackage[pdftex,usenames,dvipsnames]{color}
\usepackage[pdftex]{graphicx}
\usepackage{enumerate}
\usepackage{url}
\usepackage{amsmath}
\usepackage{amsfonts}
\usepackage{amssymb}
\usepackage[table]{xcolor}
\usepackage[small,bf]{caption}
\usepackage{float}
\usepackage{subfig}
\usepackage{bm}
\usepackage{fancyhdr}
\usepackage{times}
\usepackage{titlesec}
\usepackage{csquotes}
\usepackage[backend=bibtex]{biblatex}
\usepackage{titling}
% \usepackage{algorithmicx}
\usepackage{algpseudocode}
\usepackage{algorithm}


%%%%% BEGIN ALGPSEUDOCODE STUFF %%%%%%
\algdef{SxnE}[FOREACH]{ForEach}{EndFor}[1]{\algorithmicfor\ #1\ \algorithmicdo}
% LEAVES BLANK LINE AT END \algblockdefx[FOREACH]{ForEach}{EndFor}{\textbf{for each }}{}
\algdef{SxnE}[FOR]{For}{EndFor}[1]{\algorithmicfor\ #1\ \algorithmicdo}
\algdef{SxnE}[WHILE]{While}{EndWhile}[1]{\algorithmicwhile\ #1\ \algorithmicdo}
\algdef{SxnE}[IF]{If}{EndIf}[1]{\algorithmicif\ #1\ \algorithmicthen}
\algdef{cxnE}{IF}{Else}{EndIf}


\renewcommand{\algorithmicrequire}{\textbf{Input:}}
\renewcommand{\algorithmicensure}{\textbf{Output:}}
\algnewcommand\algorithmicauxiliary{\textbf{Auxiliary:}}
\algnewcommand\Auxiliary{\item[\algorithmicauxiliary]}

\floatname{algorithm}{Algoritmo}
%%%%% END ALGPSEUDOCODE STUFF %%%%%%


\DeclareMathOperator*{\argmin}{arg\,min}
\addbibresource{references}
\DeclareFieldFormat[inbook]{citetitle}{#1}

\newcommand{\norm}[1]{\left\lVert#1\right\rVert}

\titleformat{\section}{\small\center\bfseries}{\thesection.}{0.5em}{\normalsize\uppercase}
\titleformat{\subsection}{\small\center\bfseries}{}{0.5em}{\small\uppercase}

\def\customabstract{\vspace{.5em}
    {\small\center{\textbf{RESUMEN}} \\[0.5em] \relax%
    }}
\def\endkeywords{\par}

\def\keywords{\vspace{.5em}
    {\textit{Palabras clave: }
    }}
\def\endkeywords{\par}

% TITLE Configuration
\setlength{\droptitle}{-30pt}
\pretitle{\begin{center}\Huge\begin{rmfamily}}
\posttitle{\par\end{rmfamily}\end{center}\vskip 0.5em}
\preauthor{\begin{center}
        \large \lineskip 0.5em%
\begin{tabular}[t]{c}}
\postauthor{\end{tabular}\normalsize
    \\[1em] Estudiantes del Instituto Tecnológico de Buenos Aires
\par\end{center}}
\predate{\begin{center}\small}
\postdate{\par\end{center}}

% Headers
\addtolength{\voffset}{-40pt}
\addtolength{\textheight}{80pt}
\renewcommand{\headrulewidth}{0pt}
\fancyhead{}
\fancyfoot{}
\lhead{\small No publicado}
\rhead{\small \thepage}
\cfoot{\small Copyright \copyright 2013 ITBA}

% Metadata
\title{}
\date{20 de Septiembre de 2013}
\author{Civile, Juan Pablo \and Crespo, Álvaro \and Ordano, Esteban }

\begin{document}

\pagestyle{fancy}
\maketitle
\thispagestyle{fancy}


\begin{customabstract}
\textbf{
ABSTRACT
} \end{customabstract}


\section{Homografía}

Considerese dos imágenes, $f(x,y)$ y $f'(x,y)$, relacionadas por una transformación geométrica. Sean $p_{k}$ y $p'_{k}$ para $k = 0 ... N$ puntos de las imágenes
$f$ y $f'$ respectivamente, y dadas correspondencias tentativas $p_{k} \to p'_{k}$, se quiere estimar la transformación $T$, tal que

\begin{equation}
    f(x,y) = f(T(x,y))
\end{equation}

Esta transformación $T$ se puede modelar como una transformación de coordenadas lineales

\begin{equation}
    \begin{bmatrix}
        x_{0} \\
        y_{0} \\
        z_{0} \\
    \end{bmatrix}
    = H
    \begin{bmatrix}
        x_{1} \\
        y_{1} \\
        1 \\
    \end{bmatrix}
\end{equation}

en donde $H$ es una matriz de $3 \times 3$ que representa la proyección, rotación, escalamiento, sesgo y perspectiva.

Se puede observar que para aplicar $H$ se extiende el vector de dos dimensiones $[x,y]^{T}$ a tres componentes. El valor de la 
tercera componente puede variar, y define una clase equivalencia tal que

\begin{equation}
    \begin{bmatrix}
        zx \\
        zy \\
        z \\
    \end{bmatrix}
    \equiv
    \begin{bmatrix}
        x \\
        y \\
        1 \\
    \end{bmatrix}
\end{equation}

Este tipo de coordenadas se denominan homogéneas.

La forma de $H$ determina el tipo de transformación geométrica representada. Por ejemplo,

\begin{equation}
    H = 
    \begin{bmatrix}
        sa\cos(\theta) & -sb\sin(\theta) & t_{x}\\
        sa\sin(\theta) & sb\cos(\theta) & t_{y} \\
        p_{0}          & p_{1}          & 1     \\
    \end{bmatrix}
\end{equation}

Representa una rotación de ángulo $\theta$, una traslación dada por $t_{x}$ y $t_{y}$ (notesé el uso del ``1'' de coordenadas 
homogéneas), una escalamiento dado por el factor $s$, un sesgo introducido por $a$ y $b$ y un cambio de perspectiva
dado por $p_{0}$ y $p{1}$.

En total, son 8 las parámetros encodeados en la matriz $H$, y sus elementos son 

\begin{equation}
    H = 
    \begin{bmatrix}
        H_{00} & H_{01} & H_{02}\\
        H_{10} & H_{11} & H_{12}\\
        H_{20} & H_{21} & 1\\
    \end{bmatrix}
\end{equation}

\subsection{Estimación de una homografía}

Se pueden estimar los 8 parámetros desconocidos de la matriz de transformación $H$, basandose en correspondencias de puntos conocidas. 
La transformación de una coordenada, $x$ en coordenadas homogéneas ($z=1$), a una coordenada objetivo $x'= Hx$ da como resultado

\begin{eqnarray*}
    x' &=& H_{00}x + H_{01}y + H_{02}\\
    y' &=& H_{10}x + H_{11}y + H_{12}\\
    z' &=& H_{20}x + H_{21}y + H_{22}\\
\end{eqnarray*}

Diviendo las primeras 2 ecuaciones por $z'$ para convertirlas en a la forma Euclideana, y con un poco de álgebra básica se puede llegar a

\begin{eqnarray*}
    \frac{x}{z'}(H_{20}x + H_{21}y + H_{22}) - H_{00}x - H_{01}y - H_{02}\\
    \frac{x}{z'}(H_{20}x + H_{21}y + H_{22}) - H_{10}x - H_{11}y - H_{12}\\
\end{eqnarray*}

Que también puede ser escrito como 

\begin{flalign}
    Ah = 
    \begin{bmatrix}
        -x & -y & -1 & 0 & 0 & 0 & \frac{x'x}{z'} & \frac{x'y}{z'} & \frac{x'}{z'}\\
        0 & 0 & 0 & -x & -y & -1 & \frac{y'x}{z'} & \frac{y'y}{z'} & \frac{y'}{z'}\\
          &   &   &    & \vdots & &               &                &\\
    \end{bmatrix}
    \begin{bmatrix}
        H_{00} \\
        H_{01} \\
        H_{02} \\
        H_{10} \\
        H_{11} \\
        H_{12} \\
        H_{20} \\
        H_{21} \\
        H_{22} \\
    \end{bmatrix}
    = 0
\end{flalign}
Cada correspondencia de puntos agrega dos filas a la matriz $A$, por lo que $n$ correspondencias generan una matriz de $2N \times 9$.

Como consecuencia, se tiene el siguiente sistema de ecuaciones lineales para resolver

\begin{equation}
    Ah = 0 \hspace{1cm} h \neq 0
\end{equation}

Con 4 puntos de correspondencias, el sistema es \textit{compatible determinado} y la única solución es el espacio nulo de $A$. Para más correspondencias, 
el sistema pasa a ser \textit{compatible indeterminado} y de las infinitas soluciones, se busca la de menor norma norma
%TODO maybe agregar alguna mini explicación de porque?

\begin{equation}
    argmin_{\norm{h}=1} \norm{Ah} = argmin_{\norm{h}=1} h^{T}A^{T}Ah = \lambda_{min}
\end{equation}

donde $\lambda_{min}$ es el menor autovalor de $A^{T}A$. Esto se desprende de las propiedades de la norma 2 de matrices y de la propiedad de
las matrices cuadradas simétricas que dice que dado $B = A^{T}A$, $\lambda_{i}$ y $q_{i}$ sus autovalores y autovectores respectivamente, las matrices 

\begin{equation}
    Q = \begin{bmatrix}
            q_{0} q_{2} \dots q_{n-1}
        \end{bmatrix}
\end{equation}
y

\begin{equation}
    D = \begin{bmatrix}
            \lambda_{0} & & & \\
                        & \lambda_{2} & & \\
                        & & \ddots & \\
                        & & & \lambda_{n-1}\\
        \end{bmatrix}
\end{equation}

cumplen $B = QDQ^{T}$.

\begin{equation}
    argmin_{\norm{h}=1} h^{T}QDQ^{T}h = argmin_{\norm{y}=1} y^{T}Dy = argmin_{\norm{y}=1} \lambda_{1}y^{2}_{1} + ... + \lambda_{n}y^{2}_{n}
\end{equation}

Con $\lambda_{i} = \lambda_{min}$, se alcanza un mínimo cuando todas las componentes de $y$ se anulan excepto por $y_{i} = 1$. Como $y = Q^{T}h$, 
$h=Qy = q_{min}$, el autovector de $B$ correspondiente al mínimo autovalor.

El problema reduce entonces a hallar este autovector. Para ello, se recurre a la Descomposición en Valores Singulares 
%TODO footnote o referencia
\begin{equation}
    A = U\Sigma V^{T}
\end{equation}

donde las columnas de $U$ contienen los autovectores de $AA^{T}$ y las columnas de $V$ los autovectores de $A{T}A$, mientras que $\Sigma$ es una matriz
diagonal con los autovalores de $$A{T}A$$ en su diagonal. Notesé que los autovalores de $A{T}A$ son iguales a los autovalores de $A$ al cuadrado.

La Descomposición en Valores Singulares se calcula de tal forma que los autovalores en la diagonal de $\Sigma$ aparecen en orden decreciente. Por lo tanto 
se toma como solución la última columna de $V$, correspondiente al menor autovalor de $A^{T}A$.

\end{document}

