\documentclass[a4paper,10pt]{article}

\usepackage{fullpage}
\usepackage[utf8]{inputenc}
\usepackage{t1enc}
\usepackage[spanish]{babel}
\usepackage[pdftex,usenames,dvipsnames]{color}
\usepackage[pdftex]{graphicx}
\usepackage{enumerate}
\usepackage{url}
\usepackage{amsmath}
\usepackage{amsfonts}
\usepackage{amssymb}
\usepackage[table]{xcolor}
\usepackage[small,bf]{caption}
\usepackage{float}
\usepackage{subfig}
\usepackage{bm}
\usepackage{fancyhdr}
\usepackage{times}
\usepackage{titlesec}
\usepackage{csquotes}
\usepackage[backend=bibtex]{biblatex}
\usepackage{titling}
% \usepackage{algorithmicx}
\usepackage{algpseudocode}
\usepackage{algorithm}
\usepackage{letltxmacro}


%%%%% BEGIN ALGPSEUDOCODE STUFF %%%%%%
\algdef{SxnE}[FOREACH]{ForEach}{EndFor}[1]{\algorithmicfor\ #1\ \algorithmicdo}
% LEAVES BLANK LINE AT END \algblockdefx[FOREACH]{ForEach}{EndFor}{\textbf{for each }}{}
\algdef{SxnE}[FOR]{For}{EndFor}[1]{\algorithmicfor\ #1\ \algorithmicdo}
\algdef{SxnE}[WHILE]{While}{EndWhile}[1]{\algorithmicwhile\ #1\ \algorithmicdo}
\algdef{SxnE}[IF]{If}{EndIf}[1]{\algorithmicif\ #1\ \algorithmicthen}
\algdef{cxnE}{IF}{Else}{EndIf}


\renewcommand{\algorithmicrequire}{\textbf{Input:}}
\renewcommand{\algorithmicensure}{\textbf{Output:}}
\algnewcommand\algorithmicauxiliary{\textbf{Auxiliary:}}
\algnewcommand\Auxiliary{\item[\algorithmicauxiliary]}

\DefineBibliographyStrings{spanish}{andothers = {et\addabbrvspace al\adddot}}
\renewbibmacro{in:}{}

\floatname{algorithm}{Algoritmo}
%%%%% END ALGPSEUDOCODE STUFF %%%%%%


\DeclareMathOperator*{\argmin}{arg\,min}
\addbibresource{references}
\DeclareFieldFormat[inbook]{citetitle}{#1}

\newcommand{\norm}[1]{\left\lVert#1\right\rVert}

\titleformat{\section}{\small\center\bfseries}{\thesection.}{0.5em}{\normalsize\uppercase}
\titleformat{\subsection}{\small\center\bfseries}{}{0.5em}{\small\uppercase}

\def\customabstract{\vspace{.5em}
    {\small\center{\textbf{RESUMEN}} \\[0.5em] \relax%
    }}
\def\endkeywords{\par}

\def\keywords{\vspace{.5em}
    {\textit{Palabras clave: }
    }}
\def\endkeywords{\par}

% TITLE Configuration
\setlength{\droptitle}{-30pt}
\pretitle{\begin{center}\Huge\begin{rmfamily}}
\posttitle{\par\end{rmfamily}\end{center}\vskip 0.5em}
\preauthor{\begin{center}
        \large \lineskip 0.5em%
\begin{tabular}[t]{c}}
\postauthor{\end{tabular}\normalsize
    \\[1em] Instituto Tecnológico de Buenos Aires
\par\end{center}}
\predate{\begin{center}\small}
\postdate{\par\end{center}}

% Headers
\addtolength{\voffset}{-40pt}
\addtolength{\textheight}{80pt}
\renewcommand{\headrulewidth}{0pt}
\fancyhead{}
\fancyfoot{}
\lhead{\small No publicado}
\rhead{\small \thepage}
\cfoot{\small Copyright \copyright 2013 ITBA}

% Metadata
\title{Análisis Automático en Tiempo Real de Jugadas en Partidos de Futbol}
\date{20 de Septiembre de 2013}
\author{Civile, Juan Pablo \and Crespo, Álvaro \and Ordano, Esteban }

\begin{document}

\pagestyle{fancy}
\maketitle
\thispagestyle{fancy}


\begin{customabstract}
\textbf{
ABSTRACT
} \end{customabstract}

\part*{El problema}

El problema en cuestión es el del seguimiento de objetos en una secuencia de imágenes, aplicado a partidos de futbol.
La idea es poder detectar y seguir a los jugadores, obteniendo información útil para la toma de decisiones en tiempo real. 
Esto puede servir para sacar estadísticas o datos de los jugadores, que podrían ser utilizados por los entrenadores en su trabajo.
Una idea más ambiciosa es la posibilidad de detectar a un jugador en fuera de juego, analizar la existencia de gol en situaciones 
dudosas, detectar salida de la pelota en forma precisa, entre otras.

El problema tiene varias dificultades fundamentales. Una es la velocidad de los eventos que se desean analizar y la complejidad
de las situaciones en la que los objetos pueden encontrarse (oclusión, choques, etc...). Tanto
los jugadores como la pelota alcanzan velocidades muy altas, lo cual dificulta el análisis en tiempo real. Otra dificultad 
reside en la elección del sistema de cámaras a utilizar para llevar a cabo la observación. Un sistema de múltiples cámaras se
ve beneficiado por una mejor resolución y por lo tanto una mejor observación de los objetos, con la dificultad de necesitar un
sistema distribuido de sincronización. Por otro lado, un sistema constituido por una única cámara no tiene la complejidad extra
que implica la sincronización de la información las diferentes cámaras, pero sufre lógicamente la falta de resolución. Esta 
limitación puede tornarse prohibitiva para algunas técnicas o algoritmos de seguimiento ya que algunos objetos (como por ejemplo
la pelota) se observan solo como unos pocos pixeles, lo cual dificulta enormemente la tarea de segmentación y seguimiento.

Un aspecto no menor del análisis es la restricción de tiempo real. Muchos autores han desarrollado algoritmos automáticos 
de seguimiento de objetos en secuencias de imágenes. Todos ellos están basados en soluciones de ecuaciones diferenciales 
en derivadas parciales y resultan muy robustos, además permiten cambios en la topología pero tienen serias limitaciones 
en aplicaciones que requieran tiempo real.

Nuestro enfoque consta de un sistema de una única cámara fija, posicionada en la cancha de tal forma que pueda observar 
toda la cancha en un solo cuadro. El algoritmo de seguimiento está basado en contornos activos, lo que permite cumplir 
con la restricción de tiempo real. Sin embargo, al utilizar una cámara la resolución adquiere un rol determinante, 
dificultando o impidiendo el uso de muchas de las técnicas de seguimiento.




\begin{verbatim}

- Algoritmo de deteccion de fondo: cosas que no son la cancha en el video arruinan las mediciones
  => Poner en negro toda parte del video que no sea la cancha

- Funcion de feature de contornos activos
  + Tomar promedio de color del contorno inicial no soporta camisetas rayadas
  + Tampoco aguanta camisetas con mucha iluminacion
  => Busqueda de nuevos features
    - sigma
    - coeficiente de variacion: ???
    - distintos color-spaces: algunos color spaces funcionan mejor para un tipo de camiseta que otro. seria necesario tener varios features distintos y saber seleccionar el mejor

- Distorcion de la lente introduce error en la homografia
  => Algoritmos de correcion de la lente
     - El factor de correcion es distinto para cada video
     - Es dificil de calcular programaticamente

- Jugadores distantes se borronean mucho
  => ???

- La pelota es muy chica
  => ???

- Cuando un jugador marca a otro, suele recorrer mucha distancia oculto o semi oculto
  => ???

- Además de distorción de la lente, la RESOLUCION: al agarrar TODA la cancha en una toma los jugadores MUY chicos, y la pelota más
    => ???

- Las canchas tienen diferentes medidas (varían en un cierto rango). Esto afecta a la homografía
    => Adaptar la homografía a las dimensiones de cada cancha.

- Como relacionar posiciones relativas de los jugadores teniendo una vista 3D con perspectiva
    => Homografía
    (Muy básico?)

\end{verbatim}

\end{document}



