\documentclass[a4paper,10pt]{article}

\usepackage{fullpage}
\usepackage[utf8]{inputenc}
\usepackage{t1enc}
\usepackage[spanish]{babel}
\usepackage[pdftex,usenames,dvipsnames]{color}
\usepackage[pdftex]{graphicx}
\usepackage{enumerate}
\usepackage{url}
\usepackage{amsmath}
\usepackage{amsfonts}
\usepackage{amssymb}
\usepackage{comment}
\usepackage[table]{xcolor}
\usepackage[small,bf]{caption}
\usepackage{float}
\usepackage{subfig}
\usepackage{bm}
\usepackage{fancyhdr}
\usepackage{times}
\usepackage{titlesec}
\usepackage{csquotes}
\usepackage[backend=bibtex,sorting=none]{biblatex}
\usepackage{titling}
% \usepackage{algorithmicx}
\usepackage{algpseudocode}
\usepackage{algorithm}
\usepackage{letltxmacro}


%%%%% BEGIN ALGPSEUDOCODE STUFF %%%%%%
\algdef{SxnE}[FOREACH]{ForEach}{EndFor}[1]{\algorithmicfor\ #1\ \algorithmicdo}
% LEAVES BLANK LINE AT END \algblockdefx[FOREACH]{ForEach}{EndFor}{\textbf{for each }}{}
\algdef{SxnE}[FOR]{For}{EndFor}[1]{\algorithmicfor\ #1\ \algorithmicdo}
\algdef{SxnE}[WHILE]{While}{EndWhile}[1]{\algorithmicwhile\ #1\ \algorithmicdo}
\algdef{SxnE}[IF]{If}{EndIf}[1]{\algorithmicif\ #1\ \algorithmicthen}
\algdef{cxnE}{IF}{Else}{EndIf}


\renewcommand{\algorithmicrequire}{\textbf{Input:}}
\renewcommand{\algorithmicensure}{\textbf{Output:}}
\algnewcommand\algorithmicauxiliary{\textbf{Auxiliary:}}
\algnewcommand\Auxiliary{\item[\algorithmicauxiliary]}

\DefineBibliographyStrings{spanish}{andothers = {et\addabbrvspace al\adddot}}
\renewbibmacro{in:}{}

\floatname{algorithm}{Algoritmo}
%%%%% END ALGPSEUDOCODE STUFF %%%%%%


\DeclareMathOperator*{\argmin}{arg\,min}
\addbibresource{references}
\DeclareFieldFormat[inbook]{citetitle}{#1}

\newcommand{\norm}[1]{\left\lVert#1\right\rVert}

% \titleformat{\section}{\small\center\bfseries}{\thesection.}{0.5em}{\normalsize\uppercase}
% \titleformat{\subsection}{\small\center\bfseries}{}{0.5em}{\small\uppercase}

\def\customabstract{\vspace{.5em}
    {\small\center{\textbf{RESUMEN}} \\[0.5em] \relax%
    }}
\def\endkeywords{\par}

\def\keywords{\vspace{.5em}
    {\textit{Palabras clave: }
    }}
\def\endkeywords{\par}

% TITLE Configuration
\setlength{\droptitle}{-30pt}
\pretitle{\begin{center}\Huge\begin{rmfamily}}
\posttitle{\par\end{rmfamily}\end{center}\vskip 0.5em}
\preauthor{\begin{center}
        \large \lineskip 0.5em%
\begin{tabular}[t]{c}}
\postauthor{\end{tabular}\normalsize
    \\[1em] Instituto Tecnológico de Buenos Aires
\par\end{center}}
\predate{\begin{center}\small}
\postdate{\par\end{center}}

% Headers
\addtolength{\voffset}{-40pt}
\addtolength{\textheight}{80pt}
\renewcommand{\headrulewidth}{0pt}
\fancyhead{}
\fancyfoot{}
\lhead{\small No publicado}
\rhead{\small \thepage}
\cfoot{\small Copyright \copyright 2013 ITBA}

% Metadata
\title{Análisis Automático en Tiempo Real de Jugadas en Partidos de Futbol}
\date{20 de Septiembre de 2013}
\author{Civile, Juan Pablo \and Crespo, Álvaro \and Ordano, Esteban }

\begin{document}

\pagestyle{fancy}
\maketitle
\thispagestyle{fancy}

\begin{comment}
\begin{customabstract}
\textbf{
ABSTRACT
} \end{customabstract}
\end{comment}

\section{Introducción}

El seguimiento de jugadores en una secuencia de imágenes (video) es un caso
particular del seguimiento de objetos, aplicado a partidos de fútbol. El
objetivo es detectar y seguir la posición de los jugadores y la pelota a medida
que se desarrolla el juego, obteniendo información útil para
distintas aplicaciones, como puede ser soporte informático para los árbitros
(detección automática de pases, goles, posiciones adelantadas,
etc...), adaptación del entrenamiento de los jugadores, análisis
y estudio de las tácticas de un contricante, entre otros.

%TODO: Maybe poner algo del estado del arte? Onda los tanos y el paper de suecia y otros que ya
% atacaron el problema particular aplicado al futbol?
% Tebex: Mepa que el estado del arte se nombra después
En el presente trabajo, se asumen las siguientes restricciones: el seguimiento de jugadores se
realiza en tiempo real, con reducida supervisión humana, y a partir de una sola
fuente de video, consistente en una cámara de alta resolución (HD) fija capaz
de encuadrar todo el campo de juego.

\subsection{Tiempo real}

Realizar el seguimiento en tiempo real agrega restricciones fuertes al
problema. Toda técnica tiene un costo de procesamiento, por lo tanto se debe
tener cautela al momento de seleccionar qué procesos de análisis de imagen
pueden ser realizados en los aproximadamente 40 milisegundos que separan un
cuadro de otro al mostrar un video de 24 cuadros por segundo.

\subsection{Supervisión}

% TODO: BULLLLLSHITTTTTT los tanos no tenian 100% automatico?
El objetivo final es igualar y complementar la información sobre un partido
que un operador (o grupo de operadores) podría obtener de un video. El estado
del arte no llega a alcanzar este objetivo sin supervisión para la corrección
de errores o establecer \textit{ground truths}, es decir, se requiere que un
humano intervenga para agregar información que permita o mejore el
funcionamiento del sistema.

En el presente trabajo, un supervisor seleccionará inicialmente las posiciones de los
jugadores en la cancha, y corregirá eventuales detecciones incorrectas. Se desea minimizar
estas falsas detecciones.

\subsection{Sistema de Cámaras}

Nuestro enfoque consta de un sistema de una única cámara fija, posicionada en
la cancha de tal forma que pueda observar toda la cancha en un solo cuadro. Al
utilizar una única cámara, la resolución adquiere un rol determinante,
dificultando o impidiendo el uso de muchas de las técnicas de seguimiento.

\subsection{Dificultades}

\subsubsection{Sistema de cámaras}

Un sistema de múltiples cámaras se ve beneficiado por una mejor resolución y
por lo tanto una mejor precisión al determinar la posición de objetos, pero
requiere un sistema de sincronización que coordine la obtención de información.

Por otro lado, un sistema constituído por una única cámara no tiene la
complejidad extra que implica la sincronización de la información de las
diferentes cámaras, pero sufre de una menor resolución y menor precisión.

Esta reducción de resolución puede tornarse prohibitiva para algunas técnicas o
algoritmos de seguimiento ya que algunos objetos (como por ejemplo la pelota)
tendrán unos pocos pixels en cada imagen, lo cual dificulta la
tarea de segmentación y seguimiento al contar con una imagen de peor calidad.

\subsubsection{Real time}

Al tener una resolución de \textit{1080p} (aproximadamente dos millones de
pixels por cuadro), el procesamiento de cada pixel debe tomar a lo sumo 20 nanosegundos.
Para lidiar con esta restricción se puede utilizar información adicional de la
que se disponga respecto al video con el objeto de evitar procesar pixels de
poca o nula utilidad para el seguimiento. Un ejemplo de esto es descartar pixels que
estén fuera de la cancha, ya que es probable que la cámara encuadre más que el
campo de juego, abarcando las gradas, el público espectador, publicidades
alrededor del campo de juego, entre otros.

Muchos autores han desarrollado algoritmos automáticos de seguimiento de
objetos en secuencias de imágenes\cite{IFTrace, alp, local-learning, MHT-2}.
Todos ellos están basados en soluciones de
ecuaciones diferenciales en derivadas parciales y resultan aceptablemente
robustos, pero tienen severas restricciones que impiden que se utilicen para
aplicaciones en tiempo real.

Nuestra investigación utiliza el algoritmo de contornos activos\cite{fast-level-set}, el cual no
utiliza ecuaciones diferenciales (haciéndolo apto para aplicaciones en tiempo
real) y además hace un análisis local de los objetos seguidos en la imagen, lo
cual hace que el tiempo de análisis de un cuadro sea dependiente de la
resolución de los jugadores e independiente de la resolución del video.

\subsubsection{Corregir la Perspectiva de la Cámara}

La imagen capturada por la cámara es una representación en 2 dimensiones de
la realidad. Nuestro modelo de datos representa cada jugador y la pelota
como un punto en un campo de dos dimensiones, es decir, se descarta el valor
de la altura de cada objeto seguido, ya que no interesa esa información.
Para esto, se aplica una homografía para convertir las coordenadas de un punto de
la imagen a coordenadas en el plano donde se encuentra la cancha.

El calculo de una homografia involucra reconocer por lo menos 4 puntos de la cancha
en la imagen. Esto puede hacerse de manera supervisada, se selecciona en la imagen 
puntos y luego se dice a que punto de la cancha corresponden. O puede hacerse de
manera automatica mediante un algoritmo de deteccion de lineas que permita
comparar las lineas en la imagen con las que se encuentran en la cancha.

%TODO would be awesome un dibujo de la transformación que hace la homografía.

Ignorar el valor de altura de los objetos seguidos es una buena aproximación
para los jugadores, pero podría eventualmente causar errores en la medición de
la posición y velocidad de la pelota.

\subsubsection{Distorsión de la lente}

Al utilizar una única cámara para captar la cancha entera se corre el riesgo de
tener distorción en los puntos de la imagen más alejados al foco de la cámara.
Este es el llamado ``efecto de ojo de buey'' e introduce mucho error, por ejemplo
en la aplicación de la homografía, por lo
tanto se debe aplicar una corrección. Una lente apropiada y bien calibrada
puede reducir este error, pero nunca puede ser eliminado totalmente. Sólo
puede evitarse utilizando una mejor técnica de filmación.

\subsubsection{Oclusiones entre jugadores}

En un partido es muy común que ocurran oclusiones entre los jugadores. El
sistema debe poder tolerar la oclusión parcial o total de los jugadores.
Esto puede llevar a situaciones muy difíciles de automatizar. Una situación
difícilmente automatizable sucede cuando dos jugadores del mismo equipo (con
vestimenta muy similar) se encuentren alineados con respecto a la cámara.
Se pueden agregar reglas para intentar desambiguar, cuando los jugadores se
separen, quién es quién, pero no hay una solución evidente.

Por ejemplo, se puede utilizar información previa de cuadros anteriores para
estimar la velocidad de cada uno y determinar quién es quién, pero esto
es poco efectivo si los jugadores cambian de velocidad mientras uno ocluye
al otro, o si la velocidad era muy similar al momento de generarse la oclusión.

Otra situación problemática similar es una jugada de córner, donde las
oclusiones entre varios jugadores son muy numerosas, lo que agrega a la
restricción de tiempo real mayor complejidad, ya que la resolución de
oclusiones debe ser muy eficiente en tiempo.

\subsubsection{Contornos Activos}

El correcto funcionamiento del algoritmo de contornos
activos\cite{fast-level-set} depende de una buena selección de la función
característica, para poder distinguir claramente a un jugador respecto a otros
objetos o respecto del fondo.

%TODO meter alguna referencia?
%TODO que onda las otras imagenes?
\begin{figure}
    \centering
    \begin{minipage}{.5\textwidth}
        \centering
        \includegraphics[width=.4\linewidth]{./images/rect2995.png}
        \captionof{figure}{Camiseta de color lisa. El color de la camiseta es claramente distinguido del fondo.}
        \label{fig:camiseta}
    \end{minipage}%
    \begin{minipage}{.5\textwidth}
        \centering
        \includegraphics[width=.4\linewidth]{./images/rect2996.png}
        \captionof{figure}{Camiseta de 2 colores rayada. Las franjas de la camiseta con claramente distinguibles entre sí y el fondo.}
        \label{fig:camiseta-rayada}
    \end{minipage}
\end{figure}

Este es un desafío grande debido principalmente a dos problemas:
\begin{itemize}

\item Selección de colores: si la cancha tiene un color muy similar a la
  camiseta de un equipo, ¿cómo será posible distinguirlos? Se deben explorar
  distintas alternativas, como podría ser utilizar otra codificación de color.

\item Textura de los jugadores: esto representa un gran problema por varias
  razones. La primera es que la técnica de contornos activos está atada a la
  selección de una o varias características del objeto a seguir. Como la
  característica más distintiva es el color, la técnica termina basándose
  mayormente en ella. Pero para camisetas con más de un color, se vuelve
  absurda la idea de encontrar un color característico, dado que no existe. Por
  otro lado, dada la escasa cantidad de pixels que representan a un jugador,
  debido a nuestro enfoque de única cámara, cualquier tipo de análisis se torna
  más complejo cuando los pixels no tienen todos las mismas características o
  sus características no están lo suficientemente definidas como para
  diferenciarlas (ya sea de otros jugadores o del fondo).

\end{itemize}

%TODO aca mepa que pueden ir varias imagenes.
% 1 de un tracking de de un jugador de remera blanca (podría ser PRE y POST, osea sin pintar y pintado)
% 1 de un tracking de un jugador de boca (again PRE y POST)
% 1 de la pelota? Para mostrar la cantidad infima de pixels?

\printbibliography

\begin{comment}

- Algoritmo de deteccion de fondo: cosas que no son la cancha en el video arruinan las mediciones
  => Poner en negro toda parte del video que no sea la cancha

- Funcion de feature de contornos activos
  + Tomar promedio de color del contorno inicial no soporta camisetas rayadas
  + Tampoco aguanta camisetas con mucha iluminacion
  => Busqueda de nuevos features
    - sigma
    - coeficiente de variacion: ???
    - distintos color-spaces: algunos color spaces funcionan mejor para un tipo de camiseta que otro. seria necesario tener varios features distintos y saber seleccionar el mejor

- Distorcion de la lente introduce error en la homografia
  => Algoritmos de correcion de la lente
     - El factor de correcion es distinto para cada video
     - Es dificil de calcular programaticamente

- Jugadores distantes se borronean mucho
  => ???

- La pelota es muy chica
  => ???

- Cuando un jugador marca a otro, suele recorrer mucha distancia oculto o semi oculto
  => ???

- Además de distorción de la lente, la RESOLUCION: al agarrar TODA la cancha en una toma los jugadores MUY chicos, y la pelota más
    => ???

- Las canchas tienen diferentes medidas (varían en un cierto rango). Esto afecta a la homografía
    => Adaptar la homografía a las dimensiones de cada cancha.

- Como relacionar posiciones relativas de los jugadores teniendo una vista 3D con perspectiva
    => Homografía
    (Muy básico?)

\end{comment}

\end{document}



