\documentclass[a4paper,10pt]{article}

\usepackage{fullpage}
\usepackage[utf8]{inputenc}
\usepackage{t1enc}
\usepackage[spanish]{babel}
\usepackage[pdftex,usenames,dvipsnames]{color}
\usepackage[pdftex]{graphicx}
\usepackage{enumerate}
\usepackage{url}
\usepackage{amsmath}
\usepackage{amsfonts}
\usepackage{amssymb}
\usepackage[table]{xcolor}
\usepackage[small,bf]{caption}
\usepackage{float}
\usepackage{subfig}
\usepackage{bm}
\usepackage{fancyhdr}
\usepackage{times}
\usepackage{titlesec}
\usepackage{csquotes}
\usepackage[backend=bibtex]{biblatex}
\usepackage{titling}
% \usepackage{algorithmicx}
\usepackage{algpseudocode}
\usepackage{algorithm}
\usepackage{letltxmacro}


%%%%% BEGIN ALGPSEUDOCODE STUFF %%%%%%
\algdef{SxnE}[FOREACH]{ForEach}{EndFor}[1]{\algorithmicfor\ #1\ \algorithmicdo}
% LEAVES BLANK LINE AT END \algblockdefx[FOREACH]{ForEach}{EndFor}{\textbf{for each }}{}
\algdef{SxnE}[FOR]{For}{EndFor}[1]{\algorithmicfor\ #1\ \algorithmicdo}
\algdef{SxnE}[WHILE]{While}{EndWhile}[1]{\algorithmicwhile\ #1\ \algorithmicdo}
\algdef{SxnE}[IF]{If}{EndIf}[1]{\algorithmicif\ #1\ \algorithmicthen}
\algdef{cxnE}{IF}{Else}{EndIf}


\renewcommand{\algorithmicrequire}{\textbf{Input:}}
\renewcommand{\algorithmicensure}{\textbf{Output:}}
\algnewcommand\algorithmicauxiliary{\textbf{Auxiliary:}}
\algnewcommand\Auxiliary{\item[\algorithmicauxiliary]}

\DefineBibliographyStrings{spanish}{andothers = {et\addabbrvspace al\adddot}}
\renewbibmacro{in:}{}

\floatname{algorithm}{Algoritmo}
%%%%% END ALGPSEUDOCODE STUFF %%%%%%


\DeclareMathOperator*{\argmin}{arg\,min}
\addbibresource{references}
\DeclareFieldFormat[inbook]{citetitle}{#1}

\newcommand{\norm}[1]{\left\lVert#1\right\rVert}

\titleformat{\section}{\small\center\bfseries}{\thesection.}{0.5em}{\normalsize\uppercase}
\titleformat{\subsection}{\small\center\bfseries}{}{0.5em}{\small\uppercase}

\def\customabstract{\vspace{.5em}
    {\small\center{\textbf{RESUMEN}} \\[0.5em] \relax%
    }}
\def\endkeywords{\par}

\def\keywords{\vspace{.5em}
    {\textit{Palabras clave: }
    }}
\def\endkeywords{\par}

% TITLE Configuration
\setlength{\droptitle}{-30pt}
\pretitle{\begin{center}\Huge\begin{rmfamily}}
\posttitle{\par\end{rmfamily}\end{center}\vskip 0.5em}
\preauthor{\begin{center}
        \large \lineskip 0.5em%
\begin{tabular}[t]{c}}
\postauthor{\end{tabular}\normalsize
    \\[1em] Instituto Tecnológico de Buenos Aires
\par\end{center}}
\predate{\begin{center}\small}
\postdate{\par\end{center}}

% Headers
\addtolength{\voffset}{-40pt}
\addtolength{\textheight}{80pt}
\renewcommand{\headrulewidth}{0pt}
\fancyhead{}
\fancyfoot{}
\lhead{\small No publicado}
\rhead{\small \thepage}
\cfoot{\small Copyright \copyright 2013 ITBA}

% Metadata
\title{Análisis Automático en Tiempo Real de Jugadas en Partidos de Futbol}
\date{20 de Septiembre de 2013}
\author{Civile, Juan Pablo \and Crespo, Álvaro \and Ordano, Esteban }

\begin{document}

\pagestyle{fancy}
\maketitle
\thispagestyle{fancy}


\begin{customabstract}
\textbf{
ABSTRACT
} \end{customabstract}

\section{Introducción}

El seguimiento de jugadores en una secuencia de imágenes (video) es un caso
particular del seguimiento de objetos, aplicado a partidos de fútbol. El
objetivo es detectar y seguir la posición de los jugadores y la pelota a medida
que se desarrolla el juego, obteniendo información de mayor nivel para
distintos usos, como puede ser soporte informático para los árbitros
(detección automática de pases, goles, o cuando un jugador está fuera de
juego), datos útiles para el entrenamiento de los jugadores, obtener
información acerca de las tácticas de un contricante, entre otros.

% TODO: TIEMPO VERBAL EN ESTE PÁRRAFO
Se asumen las siguientes restricciones: el seguimiento de jugadores se
realiza en tiempo real, con reducida supervisión humana, y a partir de una sola
fuente de video, consistente en una cámara de alta resolución (HD) fija capaz
de encuadrar todo el campo de juego.

\subsubsection{Tiempo real}

Realizar el seguimiento en tiempo real agrega restricciones fuertes al
problema. Toda técnica tiene un costo de procesamiento, por lo tanto se debe
tener cautela al momento de seleccionar qué procesos de análisis de imágen
pueden ser realizados en los aproximadamente 40 milisegundos que separan un
cuadro de otro al mostrar un video de 24 cuadros por segundo.

\subsubsection{Supervisión}

El objetivo último es igualar y complementar la información sobre un partido
que un operador (o grupo de operadores) podría obtener de un video. El estado
del arte no llega a alcanzar este objetivo sin supervisión para la corrección
de errores o establecer \textit{ground truths}, es decir, se requiere que un
humano intervenga para agregar información que permita o mejore el
funcionamiento del sistema.

% TODO: TIEMPO VERBAL
Se plantea que un supervisor seleccionará inicialmente las posiciones de los
jugadores en la cancha, y corregirá detecciones incorrectas. Se desea que estas
falsas detecciones sean minimizadas.

\subsubsection{Sistema de Cámaras}

Nuestro enfoque consta de un sistema de una única cámara fija, posicionada en
la cancha de tal forma que pueda observar toda la cancha en un solo cuadro. Al
utilizar una única cámara la resolución adquiere un rol determinante,
dificultando o impidiendo el uso de muchas de las técnicas de seguimiento.

\subsection{Dificultades}

Detectamos las siguientes dificultades:

\subsubsection{Sistema de cámaras}

Un sistema de múltiples cámaras se ve beneficiado por una mejor resolución y
por lo tanto una mejor precisión al determinar la posición de objetos, pero
requiere un sistema de sincronización que coordine la obtención de información.

Por otro lado, un sistema constituído por una única cámara no tiene la
complejidad extra que implica la sincronización de la información las
diferentes cámaras, pero sufre de una menor resolución y menor precisión.

Esta reducción de resolución puede tornarse prohibitiva para algunas técnicas o
algoritmos de seguimiento ya que algunos objetos (como por ejemplo la pelota)
tendrán unos pocos pixels en cada imágen, lo cual dificulta enormemente la
tarea de segmentación y seguimiento.

\subsubsection{Real time}

Al tener una resolución de \textit{1080p} (aproximadamente dos millones de
pixels), el procesamiento de cada píxel debe tomar a lo sumo 20 nanosegundos.
Para lidiar con esta restricción se puede utilizar información adicional de la
que se disponga respecto al video con el objeto de evitar procesar pixels de
poca utilidad para el seguimiento. Un ejemplo de esto es descartar pixels que
estén fuera de la cancha, ya que es probable que la cámara encuadre más que el
campo de juego, abarcando las gradas, el público espectador, publicidades
alrededor del campo de juego, entre otros.
% TODO: Agregar algo acerca de técnicas y píxels vecinos que se vuelve
% prohibitivo

Muchos autores han desarrollado algoritmos automáticos de seguimiento de
objetos en secuencias de imágenes. Todos ellos están basados en soluciones de
ecuaciones diferenciales en derivadas parciales y resultan aceptablemente
robustos, pero tienen severas restricciones que impiden que se utilicen para
aplicaciones en tiempo real.

% TODO: QUOTE NEEDED
Nuestro enfoque consta en utilizar el algoritmo de contornos activos,
el cual no utiliza ecuaciones diferenciales (haciéndolo apto para
aplicaciones en tiempo real).

\subsubsection{Modelo de datos de posiciones en tres dimensiones o dos
dimensiones}

Para simplificar el análisis planteamos un modelo de datos donde las posiciones
de los jugadores no tienen una componente de altura. Si bien la cámara es una
fuente de datos que trabaja en dos dimensiones, se corrige la inclinación del
campo respecto al eje perpendicular a la lente asumiendo que tanto los
jugadores como la pelota tienen nula componente en altura. Esto es una buena
aproximación para los jugadores, pero podría eventualmente causar errores en la
medición de la posición y velocidad de la pelota.

\subsubsection{Distorsión de la lente}

Al utilizar una única cámara para captar la cancha entera se corre el riesgo de
tener distorción en los puntos de la imagen más alejados al foco de la cámara.
Este es el llamado ``efecto de ojo de buey'' e introduce mucho error, por lo
tanto se debe aplicar una corrección. Una lente apropiada y bien calibrada
puede reducir este error, pero nunca puede ser eliminado totalmente sólo con
una mejor técnica de filmación.

\subsubsection{Oclusiones entre jugadores}

En un partido es muy común que ocurren oclusiones entre los jugadores. El
sistema debe poder tolerar la oclusión parcial o total de los jugadores.

Esto puede llevar a situaciones muy difíciles de automatizar. Una situación
difícilmente automatizable es cuando dos jugadores del mismo equipo (con
vestimenta muy similar) se encuentren alineados con respecto a la cámara.
Se pueden agregar reglas para intentar desambiguar, cuando los jugadores se
separen, quién es quién, pero no hay una solución evidente.

Por ejemplo, se puede utilizar información previa de cuadros anteriores para
estimar la velocidad de cada uno y desambiguar quién es quién, pero esto
será poco efectivo si los jugadores cambian de velocidad mientras uno ocluye
al otro, o la velocidad era muy similar al momento de generar la oclusión.

Otra situación problemática similar es una jugada de córner, donde las
oclusiones entre varios jugadores serán muy numerosas, lo que agrega a la
restricción de tiempo real mayor complejidad, ya que la resolución de
oclusiones deberá ser muy eficiente en tiempo.

\subsection{Contornos Activos}

* Definición de la función característica
  * Selección de colores
  * Textura de los jugadores
* Solución de oclusiones

\begin{enumerate}
    \item Equipos con camisetas de varios colores: esto representa un gran problema por varias razones. La primera es que la técnica de contornos activos
        está atada a la selección de una o varias características del objeto a seguir. Como la característica más distintiva es el color, la técnica termina
        basándose mayormente en ella. Pero para camisetas con más de un color, se vuelve absurdo la idea de encontrar un color característico,
        dado que no existe. Por otro lado, dada la cantidad de pixels que representan a un jugador, dado nuestra enfoque de única cámara, cualquier
        tipo de análisis se torna más complejo cuando los pixels son tienen todos las mismas características.
    \item Definición de función característica para los objetos: relacionado con el punto anterior, establecer cuales son las características distintivas para
        poder indentificar los objetos a través del método de contornos activos se vuelve el problema principal por resolver.
\end{enumerate}




\begin{verbatim}

- Algoritmo de deteccion de fondo: cosas que no son la cancha en el video arruinan las mediciones
  => Poner en negro toda parte del video que no sea la cancha

- Funcion de feature de contornos activos
  + Tomar promedio de color del contorno inicial no soporta camisetas rayadas
  + Tampoco aguanta camisetas con mucha iluminacion
  => Busqueda de nuevos features
    - sigma
    - coeficiente de variacion: ???
    - distintos color-spaces: algunos color spaces funcionan mejor para un tipo de camiseta que otro. seria necesario tener varios features distintos y saber seleccionar el mejor

- Distorcion de la lente introduce error en la homografia
  => Algoritmos de correcion de la lente
     - El factor de correcion es distinto para cada video
     - Es dificil de calcular programaticamente

- Jugadores distantes se borronean mucho
  => ???

- La pelota es muy chica
  => ???

- Cuando un jugador marca a otro, suele recorrer mucha distancia oculto o semi oculto
  => ???

- Además de distorción de la lente, la RESOLUCION: al agarrar TODA la cancha en una toma los jugadores MUY chicos, y la pelota más
    => ???

- Las canchas tienen diferentes medidas (varían en un cierto rango). Esto afecta a la homografía
    => Adaptar la homografía a las dimensiones de cada cancha.

- Como relacionar posiciones relativas de los jugadores teniendo una vista 3D con perspectiva
    => Homografía
    (Muy básico?)

\end{verbatim}

\end{document}



