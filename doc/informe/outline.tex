\documentclass[a4paper,10pt]{article}

\usepackage{fullpage}
\usepackage[utf8]{inputenc}
\usepackage{t1enc}
\usepackage[spanish]{babel}
\usepackage[pdftex,usenames,dvipsnames]{color}
\usepackage[pdftex]{graphicx}
\usepackage{enumerate}
\usepackage{url}
\usepackage{amsmath}
\usepackage{amsfonts}
\usepackage{amssymb}
\usepackage{comment}
\usepackage[table]{xcolor}
\usepackage[small,bf]{caption}
\usepackage{float}
\usepackage{subfig}
\usepackage{bm}
\usepackage{fancyhdr}
\usepackage{times}
\usepackage{titlesec}
\usepackage{csquotes}
\usepackage[backend=bibtex,sorting=none]{biblatex}
\usepackage{titling}
% \usepackage{algorithmicx}
\usepackage{algpseudocode}
\usepackage{algorithm}
\usepackage{letltxmacro}
\usepackage[margin=1cm]{caption}

\DeclareMathOperator*{\argmin}{arg\,min}
\addbibresource{references}
\DeclareFieldFormat[inbook]{citetitle}{#1}

\newcommand{\norm}[1]{\left\lVert#1\right\rVert}

% \titleformat{\section}{\small\center\bfseries}{\thesection.}{0.5em}{\normalsize\uppercase}
% \titleformat{\subsection}{\small\center\bfseries}{}{0.5em}{\small\uppercase}

\def\customabstract{\vspace{.5em}
    {\small\center{\textbf{RESUMEN}} \\[0.5em] \relax%
    }}
\def\endkeywords{\par}

\def\keywords{\vspace{.5em}
    {\textit{Palabras clave: }
    }}
\def\endkeywords{\par}

% TITLE Configuration
\setlength{\droptitle}{-30pt}
\pretitle{\begin{center}\Huge\begin{rmfamily}}
\posttitle{\par\end{rmfamily}\end{center}\vskip 0.5em}
\preauthor{\begin{center}
        \large \lineskip 0.5em%
\begin{tabular}[t]{c}}
\postauthor{\end{tabular}\normalsize
    \\[1em] Instituto Tecnológico de Buenos Aires
\par\end{center}}
\predate{\begin{center}\small}
\postdate{\par\end{center}}

% Headers
\addtolength{\voffset}{-40pt}
\addtolength{\textheight}{80pt}
\renewcommand{\headrulewidth}{0pt}
\fancyhead{}
\fancyfoot{}
\lhead{\small } % No publicado
\rhead{\small \thepage}
\cfoot{\small Copyright \copyright 2013 ITBA}

% Metadata
\title{Problemas para el Análisis Automático en Tiempo Real de Jugadas en Partidos de Futbol}
\date{20 de Septiembre de 2013}
\author{Civile, Juan Pablo \and Crespo, Álvaro \and Ordano, Esteban }


\begin{document}

\customabstract{
Las técnicas de seguimiento de objetos en videos tienen numerosas aplicaciones
en las actividades cotidianas. En el ámbito deportivo pueden ser útiles para
soportar (o hasta reemplazar) las decisiones de los jueces o árbitros del
juego, permitir a los deportistas mejorar su juego mediante el análisis de sus
movimientos, otorgar estadísticas y métricas a los fanáticos del deporte, entre
otras aplicaciones.

Se analiza en la primer parte de este trabajo el estado del arte respecto a
técnicas aplicables al problema de seguimiento de múltiples jugadores de fútbol
mediante el uso de una o varias cámaras de video, así como las bases teóricas
para estos métodos, características, limitaciones de cada uno, y la
factibilidad de conocer el estado del juego completo en cada instante de
tiempo.

Se plantea el uso de una de estas técnicas, \textit{contornos activos}, para
el seguimiento en tiempo real de los jugadores en base a una filmación
con una cámara fija abarcando todo el campo de juego. Se plantean mejoras
al algoritmo específicas a los fines del seguimiento de jugadores de fútbol.
}

\section{Introducción}

(Sacar pedazos de informe y de state-of-the-art }

\section{Estado del arte}

(Adaptar state-of-the-art.tex)

\section{Descripción del problema}

(Adaptar informe.tex)

\section{Solución propuesta}

\subsection{Ideas}

- Eliminación de fondo por varios métodos:

  * Método de la energía
  * Eliminación de lineas por Detector de borde + umbral + morfología
  * Eliminación de colores de cancha por histograma de colores

- Características

  * Colores del jugador
  * Aprendizaje
  * Múltiples características
  * Selección de máximos en histograma con bfs

- Descriptores

  * Champo?

\subsection{Algoritmo final}

- Eliminación de lineas por detector de borde + umbral + morfología
- Múltiples características
- Descriptores

- Opcional para equipos difíciles de detectar: seguimiento por complemento
  - eliminación de colores de cancha por histograma + ronda de contornos

- Que lastima que no tuvimos 

\section{Resultados}

- Distintos videos en los que se probó
  * Dificultad de cámara quieta
- Un par de imagenes para mostrar
- Medición de cantidad de pérdidas de jugadores

\subsection{Comparación con IFTrace}

- Podremos? NO PUDIMOS!

\section{Conclusiones}

- Mal material
- Resultados aceptables
- Lástima que no tenemos la pelota

\end{document}
