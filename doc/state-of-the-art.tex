\documentclass[twocolumn,a4paper,10pt]{article}

\usepackage[utf8]{inputenc}
\usepackage{t1enc}
\usepackage[spanish]{babel}
\usepackage[pdftex,usenames,dvipsnames]{color}
\usepackage[pdftex]{graphicx}
\usepackage{enumerate}
\usepackage{url}
\usepackage{amsmath}
\usepackage{amsfonts}
\usepackage{amssymb}
\usepackage[table]{xcolor}
\usepackage[small,bf]{caption}
\usepackage{float}
\usepackage{subfig}
\usepackage{bm}
\usepackage{fancyhdr}
\usepackage{times}
\usepackage{titlesec}
\usepackage[numbers]{natbib}
\usepackage{titling}

\titleformat{\section}{\small\center\bfseries}{\thesection.}{0.5em}{\normalsize\uppercase}
\titleformat{\subsection}{\small\center\bfseries}{}{0.5em}{\small\uppercase}
\renewcommand{\bibsection}{}

% TITLE Configuration
\setlength{\droptitle}{-30pt}
\pretitle{\begin{center}\Huge\begin{rmfamily}}
\posttitle{\par\end{rmfamily}\end{center}\vskip 0.5em}
\preauthor{\begin{center}
        \large \lineskip 0.5em%
\begin{tabular}[t]{c}}
\postauthor{\end{tabular}\normalsize 
    \\[1em] Estudiantes del Instituto Tecnológico de Buenos Aires
\par\end{center}}
\predate{\begin{center}\small}
\postdate{\par\end{center}}

% Headers
\addtolength{\voffset}{-40pt}
\addtolength{\textheight}{80pt}
\renewcommand{\headrulewidth}{0pt}
\fancyhead{}
\fancyfoot{}
\lhead{\small No publicado}
\rhead{\small \thepage}
\cfoot{\small Copyright \copyright 2013 ITBA}

% Metadata
\title{}
\date{20 de Septiembre de 2013}
\author{Civile, Juan Pablo \and Crespo, Álvaro \and Ordano, Esteban }

\begin{document}

\pagestyle{fancy}
\maketitle
\thispagestyle{fancy}

% We don't want this right now, right?
%\begin{customabstract}
%\textbf{
%}
%\end{customabstract}
%
%\begin{keywords}
%\end{keywords}

\section{Estado del Arte}

En el campo del tracking y la segmentación de objetos en secuencias de imágenes y videos en tiempo real, se pueden diferenciar distintas familias de 
algoritmos que atacan este tipo de problemas de diversas formas. Una de ellas está conformada por los algoritmos basados en lo que se conoce como \textit{local learning}.
\cite{local-learning}
Estos tipos de algoritmos se caracterizan por definir una función objetivo, y un método de aprendizaje para, a través de un entrenamiento basado en 
muestras, intentar minimizar dicha función. Al este proceso de entrenamiento le sigue el proceso de tracking. Estos algoritmos suelen ser computacionalmente costosos, y 
por lo tanto no les es fácil satisfacer la restricción del tracking en tiempo real.\\

Otra familia de algoritmos se base en lo que se llama \textit{template tracking}. En esta familia se destacan los algoritmos basados en predictores lineales o 
\textit{linear predictors} \cite{alp} \cite{original-linear-predictors}. 
% TODO 
\cite{CITATION NEEDED}. 
Estos algoritmos utilizan templates de tamaño fijo que describir los objetos a trackear. El problema de este tipo de algoritmos radica en que es costoso aprender 
totalmente un template, y que, en principio no resulta sencillo adaptarlos. Es decir, para actualizar un template (por ejemplo cuando el objeto cambia de tamaño, dirección, etc...)
se debe computar o construir un template nuevo desde cero. En \cite{alp}, Holzer, Ilic y Navab, proponen un complejo algoritmo basado en el uso de linear predictors y varias capas
de templates de distintos tamaños, lo que les permiten manejar oclusiones parciales y totales con resultados muy positivos. Además, proponen un atajo computacional para 
modificar un template, calculando eficientemente inversas de matrices. Si bien los resultados obtenidos no son óptimos desde el punto de vista de la restricción del tiempo real, 
se acercan bastante, sin haber realizado mayores optimizaciones, como ser paralelización de ciertas partes del algoritmo, programación GPU, etc...

Por último se encuentran los algoritmos de basados en grafos, o \textit{graph based}. Entre ellos se destaca IFTrace \cite{IFTrace}, el cual se basa en la técnica conocida como 
\textit{Image Foresting Transform} \cite{IFT} y en la técnica de tracking de features, KLT \cite{KLT}. Este algoritmo es muy versátil, ya que las únicas suposiciones que hace son que el objeto trackeado en la imagen consista de un
solo set de pixeles conectados, que puede haber un cambio aguda del color u otras propiedades a lo largo de los bordes del objeto, y que estas propiedades pueden cambiar en el 
tiempo. Este algoritmo se basa en una pequeña segmentación base realizada de forma interactiva por el usuario en el primer frame, pero luego lograr manejar satisfactoriamente
oclusiones parciales y en la mayoría de los casos recuperarse de oclusiones totales. Para modelar la imagen y los objetos se utiliza un grafo, en el cual los pixeles son los nodos, y
estos están conectados mediantes aristas si son adyacentes y tiene un peso o costo. Este costo debe ser una medida de la probabilidad de que los puntos estén separados por el borde del
objeto trackeado. En problemas sencillos puede ser tan simple como el valor absoluto de la diferencia de color entre los pixeles. IFTrace combina el gradiente del color con el 
gradiente de una función de classificación de pixeles difusos, algo así como la similaridad entre el pixel y el set de pixeles del objeto segmentado en el frame anterior.


% * Real time tracking
%   sarasarsara (LP, IFTrace, etc)
%   * Tracking human poses
% * Football-specific papers
%   * Players papers
%   * Ball papers
% * Paper de lo' tano'

\section*{Referencias}
\begin{thebibliography}{99}
    \bibitem{local-learning} X. Li, H. Lu, ``Object tracking based on local learning''. Int. Conf. Image Process. (ICIP) 2012
    \bibitem{original-linear-predictors} F. Jurie and M. Dhome, “Hyperplane Approximation for Template Matching,” IEEE Trans. Pattern Analysis and Machine Intelligence, vol. 24, no. 7, pp. 996-100, July 2002.
    \bibitem{alp} S. Holzer, S. Ilic, N. Navab, ``Multilayer Adaptive Linear Predictors for Real-Time Tracking''. 
                  IEEE Trans. Pattern Analysis and Machine Intelligence, vol. 35, no. 1. Jan. 2013.
    \bibitem{IFTrace} R. Minetto, A.X. Falcão, et al. ``IFTrace: Video segmentation of deformable objects using the Foresting Transform''. 
                      Computer Vision and Image Understanding 116 (2012) 274–291.
    \bibitem{IFT} A. Falcão, J. Stolfi, R. Lotufo, ``The image foresting transform: theory, algorithms, and applications'', IEEE Trans. Pattern Anal. Mach. Intel. (TPAMI) 26 (1) (2004) 19–29.
    \bibitem{Falcao-seg-optimum-path} P. Miranda, A. Falcão, Links between image segmentation based on optimum-path forest and minimum cut in graph, J. Math. Imag. Vis. 35 (2009) 128–142.
    \bibitem{KLT} C. Tomasi, T. Kanade, Detection and tracking of point features, Tech. Rep. CMU-CS-91-132, Carnegie Mellon University, 1991.

 



\end{thebibliography}

\end{document}
