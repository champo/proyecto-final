\documentclass[a4paper,10pt]{article}

\usepackage{fullpage}
\usepackage[utf8]{inputenc}
\usepackage{t1enc}
\usepackage[spanish]{babel}
\usepackage[pdftex,usenames,dvipsnames]{color}
\usepackage[pdftex]{graphicx}
\usepackage{enumerate}
\usepackage{url}
\usepackage{amsmath}
\usepackage{amsfonts}
\usepackage{amssymb}
\usepackage[table]{xcolor}
\usepackage[small,bf]{caption}
\usepackage{float}
\usepackage{subfig}
\usepackage{bm}
\usepackage{fancyhdr}
\usepackage{times}
\usepackage{titlesec}
\usepackage{csquotes}
\usepackage[backend=bibtex,sorting=none]{biblatex}
\usepackage{titling}
% \usepackage{algorithmicx}
\usepackage{algpseudocode}
\usepackage{algorithm}
\usepackage{letltxmacro}

%%%%% BEGIN ALGPSEUDOCODE STUFF %%%%%%
\algdef{SxnE}[FOREACH]{ForEach}{EndFor}[1]{\algorithmicfor\ #1\ \algorithmicdo}
% LEAVES BLANK LINE AT END \algblockdefx[FOREACH]{ForEach}{EndFor}{\textbf{for each }}{}
\algdef{SxnE}[FOR]{For}{EndFor}[1]{\algorithmicfor\ #1\ \algorithmicdo}
\algdef{SxnE}[WHILE]{While}{EndWhile}[1]{\algorithmicwhile\ #1\ \algorithmicdo}
\algdef{SxnE}[IF]{If}{EndIf}[1]{\algorithmicif\ #1\ \algorithmicthen}
\algdef{cxnE}{IF}{Else}{EndIf}


\renewcommand{\algorithmicrequire}{\textbf{Input:}}
\renewcommand{\algorithmicensure}{\textbf{Output:}}
\algnewcommand\algorithmicauxiliary{\textbf{Auxiliary:}}
\algnewcommand\Auxiliary{\item[\algorithmicauxiliary]}

\DefineBibliographyStrings{spanish}{andothers = {et\addabbrvspace al\adddot}}
\renewbibmacro{in:}{}

\floatname{algorithm}{Algoritmo}
%%%%% END ALGPSEUDOCODE STUFF %%%%%%


\DeclareMathOperator*{\argmin}{arg\,min}
\addbibresource{references}
\DeclareFieldFormat[inbook]{citetitle}{#1}

\newcommand{\norm}[1]{\left\lVert#1\right\rVert}

\titleformat{\section}{\small\center\bfseries}{\thesection.}{0.5em}{\normalsize\uppercase}
\titleformat{\subsection}{\small\center\bfseries}{}{0.5em}{\small\uppercase}

\def\customabstract{\vspace{.5em}
    {\small\center{\textbf{RESUMEN}} \\[0.5em] \relax%
    }}
\def\endkeywords{\par}

\def\keywords{\vspace{.5em}
    {\textit{Palabras clave: }
    }}
\def\endkeywords{\par}

% TITLE Configuration
\setlength{\droptitle}{-30pt}
\pretitle{\begin{center}\Huge\begin{rmfamily}}
\posttitle{\par\end{rmfamily}\end{center}\vskip 0.5em}
\preauthor{\begin{center}
        \large \lineskip 0.5em%
\begin{tabular}[t]{c}}
\postauthor{\end{tabular}\normalsize
    \\[1em] Instituto Tecnológico de Buenos Aires
\par\end{center}}
\predate{\begin{center}\small}
\postdate{\par\end{center}}

% Headers
\addtolength{\voffset}{-40pt}
\addtolength{\textheight}{80pt}
\renewcommand{\headrulewidth}{0pt}
\fancyhead{}
\fancyfoot{}
\lhead{\small No publicado}
\rhead{\small \thepage}
\cfoot{\small Copyright \copyright 2013 ITBA}

% Metadata
\title{Estado del Arte: Análisis de Partidos de Fútbol mediante Tratamiento de Imágenes}
\date{20 de Septiembre de 2013}
\author{Civile, Juan Pablo \and Crespo, Álvaro \and Ordano, Esteban }

\begin{document}

\pagestyle{fancy}
\maketitle
\thispagestyle{fancy}

\begin{customabstract}
\textbf{
Las técnicas de seguimiento de objetos en videos tienen numerosas aplicaciones
en las actividades cotidianas. En el ámbito deportivo pueden ser útiles para
soportar (o hasta reemplazar) las decisiones de los jueces o árbitros del
juego, permitir a los deportistas mejorar su juego mediante el análisis de sus
movimientos, otorgar estadísticas y métricas a los fanáticos del deporte, entre
otras aplicaciones. Se describen a lo largo de este trabajo técnicas aplicables
al problema de seguimiento de múltiples jugadores de fútbol mediante el uso de
una o varias cámaras de video. Se presentan las bases teóricas para estos
métodos y se analizan características y limitaciones de cada uno, así como la
factibilidad de conocer el estado del juego completo en cada instante de tiempo.
} \end{customabstract}

\part*{Introducción}

Dentro del área del análisis de imágenes, uno de los problemas más relevantes y
estudiados es el problema de reconocer objetos en una secuencia de
imágenes para seguir su posición, detectar cambios de orientación o en su
forma, reconocerlo aunque esté parcialmente (o totalmente) ocluído, y otras
variaciones del problema. Esto tiene aplicaciones en muchos campos, como ser
medicina, control automatizado de procesos industriales, reconocimiento facial
y gestual, etcétera.

Existe una tendencia en los deportes en las últimas décadas por pasar a ser
controlados automáticamente. Un ejemplo de esto son las cámaras de alta
velocidad en tenis, y las repeticiones y análisis de los árbitros en fútbol
americano o rugby. Para la industria del entretenimiento, estos
mecanismos automatizados de análisis del estado del juego proveen datos
estadísticos, como en el caso del fútbol pueden ser posesión de pelota, cantidad
de pases exitosos, tiros al arco, fracción del tiempo que cada jugador trotó,
corrió o caminó, entre otros.

Para que el análisis sea automático y poder extraer información útil,
se debe plantear un sistema que automatice la recolección
y procesamiento de imágenes. En este trabajo se estudia el estado
actual de las distintas tecnologías usadas para este fin.

Este trabajo está compuesto de la siguiente manera:
En la sección \ref{sec:tracking} se describen los algoritmos de seguimiento
de objetos categorizados en cuatro familias:
\textit{Predictores Lineales},
\textit{Aprendizaje Local},
\textit{Corte de Grafos},
y \textit{Contornos Activos}.

En la sección \ref{sec:homography} se describe el concepto teórico de homografía y
su utilidad y aplicación en el análisis de imágenes, así como su cálculo numérico.

En la sección \ref{sec:futbol} se discuten los avances de las técnicas modernas en
análisis automático de video de eventos deportivos.
Se presentan sistemas implementados que resuelven el problema de seguimiento de
jugadores, pelota, y estado del juego. También se describen técnicas
que detectan movimientos de jugadores, pases de pelota,
tiros al arco, entre otros. Se dividen por los
sistemas de cámaras utilizados (una sola cámara fija, video televizado, o
múltiples cámaras fijas).

\newpage

\part*{}
\documentclass[twocolumn,a4paper,10pt]{article}

\usepackage[utf8]{inputenc}
\usepackage{t1enc}
\usepackage[spanish]{babel}
\usepackage[pdftex,usenames,dvipsnames]{color}
\usepackage[pdftex]{graphicx}
\usepackage{enumerate}
\usepackage{url}
\usepackage{amsmath}
\usepackage{amsfonts}
\usepackage{amssymb}
\usepackage[table]{xcolor}
\usepackage[small,bf]{caption}
\usepackage{float}
\usepackage{subfig}
\usepackage{bm}
\usepackage{fancyhdr}
\usepackage{times}
\usepackage{titlesec}
\usepackage[numbers]{natbib}
\usepackage{titling}

\titleformat{\section}{\small\center\bfseries}{\thesection.}{0.5em}{\normalsize\uppercase}
\titleformat{\subsection}{\small\center\bfseries}{}{0.5em}{\small\uppercase}
\renewcommand{\bibsection}{}

% TITLE Configuration
\setlength{\droptitle}{-30pt}
\pretitle{\begin{center}\Huge\begin{rmfamily}}
\posttitle{\par\end{rmfamily}\end{center}\vskip 0.5em}
\preauthor{\begin{center}
        \large \lineskip 0.5em%
\begin{tabular}[t]{c}}
\postauthor{\end{tabular}\normalsize 
    \\[1em] Estudiantes del Instituto Tecnológico de Buenos Aires
\par\end{center}}
\predate{\begin{center}\small}
\postdate{\par\end{center}}

% Headers
\addtolength{\voffset}{-40pt}
\addtolength{\textheight}{80pt}
\renewcommand{\headrulewidth}{0pt}
\fancyhead{}
\fancyfoot{}
\lhead{\small No publicado}
\rhead{\small \thepage}
\cfoot{\small Copyright \copyright 2013 ITBA}

% Metadata
\title{}
\date{20 de Septiembre de 2013}
\author{Civile, Juan Pablo \and Crespo, Álvaro \and Ordano, Esteban }

\begin{document}

\pagestyle{fancy}
\maketitle
\thispagestyle{fancy}

% We don't want this right now, right?
%\begin{customabstract}
%\textbf{
%}
%\end{customabstract}
%
%\begin{keywords}
%\end{keywords}

\section{Estado del Arte}

% * Real time tracking
%   sarasarsara (LP, IFTrace, etc)
%   * Tracking human poses
% * Football-specific papers
%   * Players papers
%   * Ball papers
% * Paper de lo' tano'

\section*{Referencias}
\begin{thebibliography}{99}
\end{thebibliography}

\end{document}


\printbibliography

\end{document}
