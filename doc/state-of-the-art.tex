\documentclass[a4paper,10pt]{article}

\usepackage{fullpage}
\usepackage[utf8]{inputenc}
\usepackage{t1enc}
\usepackage[spanish]{babel}
\usepackage[pdftex,usenames,dvipsnames]{color}
\usepackage[pdftex]{graphicx}
\usepackage{enumerate}
\usepackage{url}
\usepackage{amsmath}
\usepackage{amsfonts}
\usepackage{amssymb}
\usepackage[table]{xcolor}
\usepackage[small,bf]{caption}
\usepackage{float}
\usepackage{subfig}
\usepackage{bm}
\usepackage{fancyhdr}
\usepackage{times}
\usepackage{titlesec}
\usepackage{csquotes}
\usepackage[backend=bibtex]{biblatex}
\usepackage{titling}
% \usepackage{algorithmicx}
\usepackage{algpseudocode}
\usepackage{algorithm}
\usepackage{letltxmacro}


%%%%% BEGIN ALGPSEUDOCODE STUFF %%%%%%
\algdef{SxnE}[FOREACH]{ForEach}{EndFor}[1]{\algorithmicfor\ #1\ \algorithmicdo}
% LEAVES BLANK LINE AT END \algblockdefx[FOREACH]{ForEach}{EndFor}{\textbf{for each }}{}
\algdef{SxnE}[FOR]{For}{EndFor}[1]{\algorithmicfor\ #1\ \algorithmicdo}
\algdef{SxnE}[WHILE]{While}{EndWhile}[1]{\algorithmicwhile\ #1\ \algorithmicdo}
\algdef{SxnE}[IF]{If}{EndIf}[1]{\algorithmicif\ #1\ \algorithmicthen}
\algdef{cxnE}{IF}{Else}{EndIf}


\renewcommand{\algorithmicrequire}{\textbf{Input:}}
\renewcommand{\algorithmicensure}{\textbf{Output:}}
\algnewcommand\algorithmicauxiliary{\textbf{Auxiliary:}}
\algnewcommand\Auxiliary{\item[\algorithmicauxiliary]}

\floatname{algorithm}{Algoritmo}
%%%%% END ALGPSEUDOCODE STUFF %%%%%%


\DeclareMathOperator*{\argmin}{arg\,min}
\addbibresource{references}
\DeclareFieldFormat[inbook]{citetitle}{#1}

\newcommand{\norm}[1]{\left\lVert#1\right\rVert}

\titleformat{\section}{\small\center\bfseries}{\thesection.}{0.5em}{\normalsize\uppercase}
\titleformat{\subsection}{\small\center\bfseries}{}{0.5em}{\small\uppercase}

\def\customabstract{\vspace{.5em}
    {\small\center{\textbf{RESUMEN}} \\[0.5em] \relax%
    }}
\def\endkeywords{\par}

\def\keywords{\vspace{.5em}
    {\textit{Palabras clave: }
    }}
\def\endkeywords{\par}

% TITLE Configuration
\setlength{\droptitle}{-30pt}
\pretitle{\begin{center}\Huge\begin{rmfamily}}
\posttitle{\par\end{rmfamily}\end{center}\vskip 0.5em}
\preauthor{\begin{center}
        \large \lineskip 0.5em%
\begin{tabular}[t]{c}}
\postauthor{\end{tabular}\normalsize
    \\[1em] Instituto Tecnológico de Buenos Aires
\par\end{center}}
\predate{\begin{center}\small}
\postdate{\par\end{center}}

% Headers
\addtolength{\voffset}{-40pt}
\addtolength{\textheight}{80pt}
\renewcommand{\headrulewidth}{0pt}
\fancyhead{}
\fancyfoot{}
\lhead{\small No publicado}
\rhead{\small \thepage}
\cfoot{\small Copyright \copyright 2013 ITBA}

% Metadata
\title{Estado del Arte: Análisis de Partidos de Fútbol mediante Tratamiento de Imágenes}
\date{20 de Septiembre de 2013}
\author{Civile, Juan Pablo \and Crespo, Álvaro \and Ordano, Esteban }

\begin{document}

\pagestyle{fancy}
\maketitle
\thispagestyle{fancy}

\begin{customabstract}
\textbf{
Las técnicas de seguimiento de objetos en videos tienen numerosas aplicaciones
en las actividades cotidianas. En el ámbito deportivo pueden ser útiles para
soportar (o hasta reemplazar) las decisiones de los jueces o árbitros del
juego, permitir a los deportistas mejorar su juego mediante el análisis de sus
movimientos, otorgar estadísticas y métricas a los fanáticos del deporte, entre
otras aplicaciones. Se describen a lo largo de este trabajo técnicas aplicables
al problema de seguimiento de múltiples jugadores de fútbol mediante el uso de
una o varias cámaras de video. Se presentan las bases teóricas para estos
métodos y se analizan características y limitaciones de cada uno, asi como la
factibilidad de conocer el estado del juego completo en cada instante de tiempo.
} \end{customabstract}

\part*{Introducción}

Dentro del área del análisis de imágenes, uno de los problemas más relevantes y
estudiados es el problema de reconocer objetos en una secuencia de
imágenes para seguir su posición, detectar cambios de orientación o en su
forma, reconocerlo aunque esté parcialmente (o totalmente) ocluído, y otras
variaciones del problema. Tiene aplicaciones en muchos campos, como ser
medicina, control automatizado de procesos industriales, reconocimiento facial
y gestual, etcétera.

Existe una tendencia en los deportes en las últimas décadas por pasar a ser
controlados automáticamente. Un ejemplo de esto son las cámaras de alta
velocidad en tenis, y las repeticiones y análisis de los árbitros en fútbol
americano o rugby. Para la industria del entretenimiento, estos
mecanismos automatizados de análisis del estado del juego proveen datos
estadísticos, como en el caso del fútbol pueden ser posesión de pelota, cantidad
de pases exitosos, tiros al arco, fracción del tiempo que cada jugador trotó,
corrió o caminó, entre otros.

Para permitir estos análisis de manera automática y recuperar datos con alto
nivel semántico, se debe plantear un sistema que automatice la recolección
y procesamiento de imágenes. En este trabajo se estudia el estado
actual de las distintas tecnologías usadas para este fin.

En la sección \ref{sec:tracking} se describen los algoritmos de seguimiento
de objetos categorizados en cuatro familias:
\textit{Predictores Lineales},
\textit{Aprendizaje Local},
\textit{Corte de Grafos},
y \textit{Contornos Activos}.

En la sección \ref{sec:homography} se describe el concepto teórico de homografía y
su utilidad y aplicación en el analisis de imagenes, así como su cálculo numérico.

En la sección \ref{sec:futbol} se discuten los avances modernos en el fútbol.
Se presentan sistemas implementados que resuelven el problema de seguimiento de
jugadores, pelota, y estado del juego. También se describen técnicas con
mayor nivel semántico que detectan movimientos de jugadores, pases de pelota,
tiros al arco, entre otros. Se dividen por los
sistemas de cámaras utilizados (una sola cámara fija, video televizado, o
múltiples cámaras fijas).

\newpage

\section{Algoritmos de seguimiento}
\label{sec:tracking}

A continuación se describen los algoritmos de cuatro familias dentro de la rama
de análisis de imágenes para seguimiento de objetos.

\subsection{Predictores Lineales}

Un \textit{Predictor Lineal} es una función que toma los valores de una serie
de variables aleatorias y predice el valor de una variable dependiente.  La
función tiene la forma $f(x_1, ..., x_n) = \beta_0 + \beta_1 x_1 + \dots +
\beta_n x_n$, donde $x_i$ es una variable aleatoria y $\beta_i$ una constante.
Se dice que el predictor es lineal debido a la condición $\beta_i \in \mathbb{R}$
\footnote{Cualquier $x_i$ no lineal, puede expresarse como $x'_i = g(x_i)$ y tratarse como lineal}.

En los artículos \cite{alp, original-linear-predictors} los autores utilizan
predictores para hacer seguimiento de objetos en video.  El objetivo es
relacionar el cambio de valor de varios píxeles cuadro a cuadro con el
movimiento del objeto entre dichos cuadros.  Para soportar cualquier clase de
movimiento, este se representa mediante una homografía (ver
\cite{homography-estimation}), $H \in R^{3\times3}$.

Los predictores se representan mediante una matriz $A \in \mathbb{R}^{9xn}$
donde cada fila representa una función de predicción.  De dos cuadros
consecutivos se obtiene el vector de cambios de los valores de píxeles $X = (x_1,
\dots, x_n)$ y se computa:

\begin{eqnarray*}
    AX &=& B \\
    B &=& (h_{1,1}, h_{1,2}, h_{1,3}, h_{2,1}, h_{2,2}, h_{2,3}, h_{3,1}, h_{3,2}, h_{3,3})
\end{eqnarray*}

Donde $B$ contiene los valores de la homografia $H$ de movimiento entre los cuadros.

\subsubsection{Cálculo de los predictores}
% TODO: Aclarar que es perturbar
% TODO: Aclarar que el calculo de la homografia de la perturbacion es directa
Para obtener la relación $A$ se utiliza un paso inicial de entrenamiento.
Primero se determina la posición del objeto en un cuadro de manera supervisada.
Una vez conocida la posición, se toman $n$ muestras aleatorias del valor de
píxeles en ese cuadro.

Luego se crean $m$ perturbaciones de la posición del objeto, y se computa la
homografía $H_i$ (para $i = 1, 2, \dots m$) que representa la perturbación.
Se denomina $B_i$ al vector que representa cada homografía. Además, se toman
muestras de valores de píxeles ($X_i$) con sus posiciones alteradas por la
homografía $H_i$. Se plantea:

\begin{equation}
    A \left( X_1 \lvert \dots \lvert X_m \right) = \left( B_1 \lvert \dots \lvert B_m \right)
\end{equation}
El cual es un sistema de ecuaciones lineales que puede ser resuelto de manera sencilla.

\subsubsection{Mejoras}

Para mejorar la efectividad de los algoritmos, se calculan varios predictores y se disponen en capas.
Cada capa se construye para buscar cambios de posición cada vez más finos o pequeños.
Es decir, la primer capa busca cambios bruscos y la última pequeños cambios.

\citeauthor*{alp} (ver \cite{alp}) introducen una nueva técnica para manejo de oclusiones
parciales de manera eficiente. Para esto se divide el objeto en pequeñas
secciones cuadradas, llamadas \textit{templates}, y se computa la matriz $A$ de
la siguiente manera:

\begin{eqnarray*}
    Y &=& \left( B_1 \lvert \dots \lvert B_m \right) \\
    H &=& \left( X_1 \lvert \dots \lvert X_m \right) \\
    A &=& Y H^T(HH^T)^{-1}
\end{eqnarray*}

Esto permite actualizar $A$
rápidamente para remover y agregar \textit{templates}. Es decir, permite ignorar
secciones ocultas del objeto hasta que éstas vuelven a ser visibles.

\subsection{Aprendizaje Local}
% Local learning

El aprendizaje local es otra técnica que permite resolver el problema del
seguimiento de objetos en secuencias de imágenes, que ha sido también un tema
de investigación en el área del aprendizaje automático (ver
\cite{local-learning-machine-learning}) y usado principalmente en la
clasificación de imágenes, recuperación y reconocimiento de objetos.

A diferencia del aprendizaje global, que entrena un modelo basado en
todos los datos de entrenamiento, en \textit{local learning} se consideran
varios modelos locales, cada uno extraído de solamente un subconjunto de los
datos de entrenamiento. Este tipo de enfoque enfatiza la idea de que un modelo
local puede caracterizar mejor las propiedades intrínsecas y discriminativas de
su correspondiente subconjunto de datos, que un modelo global para el conjunto
completo de datos. De esto se desprende que utilizar múltiples modelos locales
puede ofrecer mejores resultados cuando los datos están distribuidos de manera
complicada o poco clara.

Si bien se puede pensar al seguimiento de objetos como una tarea de clasificación binaria
que apunta a separar el objeto del resto de la imagen o fondo, en realidad, el principal
problema reside en relacionar las apariciones del objeto entre dos cuadros
consecutivos. Esto se logra mediante una función de distancia que
establece el grado la similaridad entre los puntos característicos \footnote{Los puntos
  característicos son puntos que se diferencian del resto por poseer o mostrar
ciertas características específicas y representativas. En el caso del
seguimiento de objetos, son los puntos que mejor representan las
características del objeto, y que tienen mayor resistencia a variar con los
diferentes cambios de escala, rotación, iluminación, etc...} de cada cuadro.

La elección de la medida de distancia es de suma importancia. A modo de
ejemplo, la distancia Euclideana puede llevar a algoritmos de seguimiento
inestables a la hora de diferenciar el objeto del fondo de la imagen. Por lo
tanto, se requiere una mejor y más compleja medida de distancia. Esta es la
idea princial que proponen Li y Lu (ver \cite{local-learning}).

La función de distancia que usan es una combinación lineal de distancias
elementales, como se presenta en \cite{malisiewicz-cvpr08}. Definen a la
función de distancia entre dos puntos característicos, a los que llaman
\textit{ejemplares}, $e$ y $z$ como:

\begin{equation}
    \label{eq:distance-exemplar}
    D_{e}(z) = w_{e} \cdot d_{ez}
\end{equation}

donde $w_{e}$ es el vector de pesos de $e$, y $d_{ez}$ es el vector de
distancia n-dimensional entre $e$ y $z$, cuya n-ésima componente es la
distancia $L_{2}$ entre la característica n-ésima de $e$ y $z$. Cabe destacar
que la función del vector de pesos, $w_{e}$, es asignar una importancia relativa a
la distancia de cada componente de $d_{ez}$. De esta forma, se ponderan las distancias
entra ciertas características por sobre otras.

Cada ejemplar está también asociado con un vector binario $B_{e}$, cuyos elementos
no nulos implican que los ejemplares correspondientes son similares a $e$. La
longitud de $B_{e}$ es igual al número de ejemplares con el mismo rótulo de $e$
\footnote{Se dice que un conjunto de ejemplares tiene un mismo rótulo si pertenecen a un mismo objeto. En el caso de seguimiento de objetos
más sencillo, exitirían 2 objetos: el objeto a seguir, y el fondo}.
Asumiendo que el aprendizaje de cada una de las funciones de distancia es
independiente del resto, se puede aprender $w_{e}$ y $B_{e}$, en un problema
de aprendizaje formulado de la siguiente manera:

\begin{equation}
    \label{eq:learning-problem}
    f_{1}(w,B) = \sum_{i \in C} B_{i}L(-w \cdot d_{i}) + \sum_{i\notin C}L(w \cdot d_{i})
\end{equation}
\begin{equation}
    {w^{*}, B^{*} = \argmin_{w,b} f_{1} (w,b) }
\end{equation}
\begin{equation}
   w \geq 0, B_{i} \in {0,1}, \sum_{i} B_{i} = M
\end{equation}

Nótese que se descarta el subíndice $e$ para tener una mayor claridad. El
conjunto $C$ es el conjunto de todos los ejemplares con el mismo rótulo que $e$
y $M$ es el mínimo número de ejemplares similares a $e$ (predefinido). La
función $L$ puede ser cualquier función de costo o pérdida \footnote{Una
función de pérdida o función de costo es simplemente una función que mapea un
evento o los valores de una o más variables a un número real que representa
algún ``costo'' asociado al evento.} , estrictamente positiva. El vector de
pesos $w$ se requiere que sea positivo para asegurar que una mayor distancia
elemental (entre alguna de las características) no pueda nunca llevar a una
menor distancia total, lo que implicaría una mayor similaridad.

Previo al seguimiento se deben especificar manualmente algunos parámetros iniciales,
como ser: punto central, ancho, altura y rotación del objeto. En el proceso de
entrenamiento, se aplica un Análisis de la Componente Principal
\footnote{El Análisis de la Componente Principal, es una técnica utilizada para reducir la dimensionalidad de un conjunto de datos.
Sirve para hallar las causas de la variabilidad de los datos y ordenarlas por importancia.
Técnicamente, busca la proyección según la cual los datos queden mejor representados en términos de cuadrados mínimos. Involucra el cálculo de la descomposición en
autovalores de la matriz de covarianza, normalmente tras centrar los datos en la media de cada atributo.}
de forma incremental en los primeros $F$ cuadros. En su trabajo, Li y Lu usan
$F = 5$ (ver \cite{local-learning}).

Los mejores resultados se seleccionan como muestras de entrenamiento positivas, mientras que
los peores resultados, se toman como muestras negativas. Se obtienen las características
RGB y LBP, como se presentan en \cite{tracking-bag-of-features}, de todas las muestras,
tanto positivas como negativas. Para cada muestra de entrenamiento, se calculan las
distancias elementales RGB y LBP con respecto a todas las restantes muestras.

Para obtener el vector de pesos óptimo, $w^{*}$, se calcula iterativamente $B$,
en función de $w$, y $w$ en función de $B$, asegurando que el valor de $f_{1}$ nunca
crezca, para encontrar un mínimo local. Este proceso iterativo se puede modelar de la
siguiente forma:

\begin{equation}
   \label{eq:local-learning-B-k}
   B^{k} = \argmin_{B} \sum_{i \in C} B_{i}L(-w^{k} \cdot d_{i})
\end{equation}
\begin{equation}
    \label{eq:local-learning-w-k}
    w^{k+1} = \argmin_{w} \sum_{i:B_{i}^{k}=1} L(-w \cdot d_{i}) + \sum_{i\notin C}L(w \cdot d_{i})
\end{equation}

Dado un $w^{k}$, que inicialmente podría ser aleatorio, se minimiza la ecuación
\ref{eq:local-learning-B-k} fijando $B_{i} = 1$ para los $M$ valores más
pequeños de $L(-w \cdot d_{i})$ y $0$ para el resto. Dado $B^{k}$, se puede
obtener fácilmente $w^{k}$, resolviendo la ecuación \ref{eq:local-learning-w-k}.
El aprendizaje termina cuando $B^{k+1}=B^{k}$.\\

En el proceso de seguimiento, para cada cuadro nuevo, se determinan cuáles son los candidatos
a ser ejemplares. En el área del aprendizaje automático, siempre se seleccionan
candidatos para luego verificar si cumplen las condiciones requeridas. En este
caso, se quiere ver si los candidatos cumplen con las condiciones necesarias
% TODO (jpcivile): Verificar si la frase "los candidatos..." es correcta, no se entendía nada
para ser puntos característicos. Los candidatos a puntos característicos se
toman aleatoriamente utilizando un filtro de partículas
\footnote{El filtro de partículas es un método empleado para estimar el estado
  de un sistema que cambia a lo largo del tiempo. Más concretamente es un
  método de Montecarlo(secuencial). Se compone de un conjunto de muestras (las
  partículas) y valores, o pesos, asociados a cada una de esas muestras.
  Las partículas son estados posibles del proceso, que se pueden representar
  como puntos en el espacio de estados de dicho proceso.}
sobre el resultado del cuadro anterior. Luego, se extraen las características
RGB y LBP como muestras de testeo. Después de calcular las distancias
elementales entre las muestras de testeo y de entrenamiento, se forma una
matriz de distancia, $D \in R^{N_{test} \times N_{train}}$ a través de las
funciones de distancia entrenadas. Es decir, el elemento $(i,j)$ de la matriz
$D$, contiene la distancia entre la muestra de testeo $i$, y la muestra de
entrenamiento $j$.

Finalmente, se utiliza la matriz $D$ para localizar al objeto como sigue:

\begin{equation}
    T = \argmin_{t} c \sum_{i} D_{i}(t) + (1 - c) \sum_{j} D_{j}(t)
\end{equation}

\begin{equation}
    \forall i \in \{i : i \in S^{+}, D_{i}(t) \leq Thr_{D} \}
\end{equation}
\begin{equation}
    \forall j \in \{j : j \in S^{+}, D_{j}(t) > Thr_{D} \}
\end{equation}

Donde $t$ es un candidato y $T$ es el objeto. $S^{+}$ es el
conjunto de muestras de entrenamiento positivas. La idea consiste en
que las muestras cuyas distancias sean menor que el umbral $Thr_{D}$
contribuyan más a localizar el objeto, entonces la constante $c$ debería tener
un valor acorde, $c > 0.5$. Los autores Li y Lu utilizaron $c = 0.7$ (ver
\cite{local-learning}).

El algoritmo de seguimiento se complementa con una idea más, que
contribuye a atacar el problema de los cambios de poses y las oclusiones.
Para esto es necesario actualizar el modelo que se tiene para
que efectivamente pueda manejar estas dificultades.
Haciendo uso del umbral de distancia $Thr_{D}$ para prevenir
malas actualizaciones, se realizan correciones al modelo,
en este caso, las funciones de distancia. Esto es, una distancia
menor a $Thr_{D}$ indica que el candidato pertenece a la misma clase,
mientras que una distancia mayor indica lo contrario. Utilizando
la ecuación \ref{eq:learning-model-update-1}, se agregan candidatos,
como muestras positivas de entrenamiento y cada 5 cuadros,
se utiliza el conjunto actualizado de entrenamiento para
reentrenar todas las funciones de distancia.

\begin{equation}
    \label{eq:learning-model-update-1}
    t_{label} = \left\{
                \begin{array}{l l}
                  % TODO: LA corrección dice que Tht_{D} no está definida
                    1, & D_{S^{+}}(t) \leq Tht_{D}\\
                    0, & D_{S^{+}}(t) >  Tht_{D}
                \end{array} \right.
\end{equation}

\begin{equation}
    \label{eq:learning-model-update-2}
    D_{S^{+}}(t) = \dfrac{1}{N_{S^{+}}} \sum_{i \in S^{+}} D_{i}(t)
\end{equation}

En la ecuación \ref{eq:learning-model-update-2}, $N_{S^{+}}$ es el
número de muestras positivas de entrenamiento y $D_{S^{+}}(t)$
es el promedio de distancia del candidato $t$ a las muestras
de entrenamiento positivas.

\subsection{Graph based}
% IFTrace

El algoritmo conocido como \textit{IFTrace} (ver \cite{IFTrace}) tiene como objetivo
seguir un objeto en una secuencia de imágenes. Este algoritmo realiza algunas
asunciones sobre el objeto a seguir:

\begin{itemize}
    \item consiste de una o más regiones conexas,
    \item tiene un borde bien definido
    \item sus propiedades intrínsecas pueden variar con el tiempo.
\end{itemize}

Los objetos a seguir son inicialmente marcados de forma interactiva por el
usuario en el primer cuadro, y luego se seleccionan automáticamente varios
marcadores en el interior y los alrededores del objeto. Estos marcadores se
localizan en el siguiente cuadro utilizando en forma conjunta el algoritmo de
seguimiento de características KLT (ver \cite{KLT}) y extrapolación de
movimiento. Los bordes del objeto son entonces identificados a partir de estos
marcadores por la IFT, \textit{Image Foresting Transform} (ver \cite{IFT}).
Otra característica central del algoritmo es el operador de detección de borde
que se adapta gradualmente a cambios en el color y la textura del objeto.

El método IFT trata a la imagen como un grafo, en el cual los píxeles son los
nodos, y estos están conectados mediantes aristas si son adyacentes y tiene un
peso o costo. Este costo debe ser una medida de la probabilidad de que los
puntos estén separados por el borde del objeto de interés. En problemas
sencillos puede ser tan simple como el valor absoluto de la diferencia de color
entre los píxeles. El método combina el gradiente del color con el gradiente de
una función de clasificación difusa (\textit{fuzzy matching}), que consiste en
cuantificar la similitud entre el píxel y el conjunto de píxeles del objeto
segmentado en el frame anterior.

La IFT encuentra caminos de costo mínimo desde los marcadores, tanto internos
como externos, a cada píxel de la imagen. Los píxeles son agrupados en
particiones, que resultan ser árboles de camino óptimo disjuntos,
\footnote{Dado un grafo simple, no dirigido y conexo G, un árbol de camino
  óptimo, o árbol de camino más corto, es un árbol recubridor de $G$, tal que
  la distancia o costo del camino desde la raíz $v$ a cualquier otro vértice
$u$ es la distancia de camino mínima de $v$ a $u$ en $G$.} donde cada árbol
tiene a uno de los marcadores como raíz. La proyección del objeto es entonces
tomada como la unión de los árboles cuyas raíces son marcadores internos.

\subsubsection{El algoritmo IFTrace}
\begin{algorithm}
    \caption{IFTrace}
    \label{alg:IFTrace-algorithm1-IFTrace}
    \begin{algorithmic}
        \Require\hspace{\algorithmicindent}\hspace{\algorithmicindent}Video I : D x \{1..$n_{f}$\} $\to V$
        \State\hspace{\algorithmicindent}\hspace{\algorithmicindent}\hspace{\algorithmicindent}\hspace{0.3cm}donde D = \{1..$n_{x}$\} x \{1..$n_{y}$\}
        \State\hspace{\algorithmicindent}\hspace{\algorithmicindent}\hspace{\algorithmicindent}\hspace{0.3cm}Mascara Binaria $O^{(1)}$ : D $\to$ \{0,1\}

        \Ensure \hspace{\algorithmicindent}\hspace{0.23cm} Mascaras Binarias $O^{(t)}$ : D $\to$ \{0,1\} para $t \in $ \{2..$n_f$\}
        \State

        \For{$t = 2,3, ..., n_f$}
            \State $(R^{(t-1)}_{o}$,$R^{(t-1)}_{x}) \gets $  SelectMarkers($I^{(t-1)}$,$O^{(t-1)}$)
            \State $(S^{(t)}_{o}$,$S^{(t)}_{x}) \gets$  TrackMarkers($R^{(t-1)}_{o}$,$R^{(t-1)}_{x}$,$I^{(t-1)}$,$I^{(t)}$)
            \If{$S^{(t)}_{o}$ $\neq \emptyset$}
                \State $O^{(t)} \gets$  IFTSegment($I^{(t)}, C^{(t-1)}, S^{(t)}_{o}, S^{(t)}_{x}$)
                \State$C^{(t)} \gets$ BuildClassifier($I^{(t)}$,$O^{(t)}$)
            \Else
                \State $C^{(t)} \gets C^{(t-1)}$
                \State $(O^{(t)}$,$C^{(t)}) \gets$  RecoverObj(I,t,O,C)
            \EndIf\EndFor
        \State \Return $O$
    \end{algorithmic}
\end{algorithm}

El algoritmo \ref{alg:IFTrace-algorithm1-IFTrace} muestra el algoritmo de IFTrace
desde el más alto nivel. IFTrace toma como input un video $I$, modelado como un
array 3D de píxeles, siendo las dimensiones el alto y ancho de la imagen, y
número de cuadro de la secuencia de video, y una máscara binaria $O^{(1)}$,
correspondiente al objeto marcado en el primer cuadro, y devuelve un array de
máscaras binarias $O^{(t)}$, con las proyecciones del objeto para cada cuadro
del video.

Para cada cuadro de la secuencia, el algoritmo elije dos conjuntos de puntos o píxeles: los que pertenecen a la máscara del objeto $R_{o}$, y los que pertenecen a
sus alrededores $R_{x}$. Luego se procede
a localizarlos en el siguiente cuadro a través del procedimiento \textit{trackMarkers}.

Si el conjunto de píxeles que representan al objeto $S_{0}^{(t)}$ no resulta
vacío para este nuevo cuadro, entonces el seguimiento fue exitoso. A
continuación, se utiliza el procedimiento
\textit{IFTSegment} para obtener la máscara del objeto para el cuadro actual,
denominado $O{(t)}$. Este procedimiento toma los conjuntos de pixeles pertenecientes al objeto $S_{0}^{(t)}$ y
a sus alrededores $S_{x}^{(t)}$, el clasificador de color del cuadro anterior $C^{(t-1)}$ y el
cuadro actual $I{(t)}$.

Como paso final, se obtiene el clasificador de color del cuadro actual $C{(t)}$, a partir de la máscara del objeto $O{(t)}$ y el cuadro actual $I{(t)}$.

Por otro lado, si el seguimiento del objeto falla para el nuevo cuadro, es decir si el conjunto de píxeles resultantes luego de aplicar \textit{trackMarkers} $S_{0}^{(t)}$
resulta vacío, entonces se mantiene el mismo clasificador de color del cuadro anterior $C{(t-1)}$, y se procede
a intentar recuperar el objeto en el cuadro actual mediante \textit{recoverObj}. Si la recuperación
es exitosa, se obtiene un máscara del objeto $O{(t)}$, y un clasificador de color para el cuadro actual $C{(t)}$,
sino, todos los conjuntos se dejan vacíos y el clasificador se mantiene inalterado.\

\subsubsection{Algoritmo RecoverObj}

\begin{algorithm}
    \caption{RecoverObj - Intento de recuperar un objeto perdido de vista}
    \label{alg:IFTrace-algorithm2-recoverObj}
    \begin{algorithmic}
        \Require\hspace{\algorithmicindent}\hspace{\algorithmicindent}Video $I$, indice de cuadro $t$, tope cuenta hacia atras $m_{f}$,
        \State\hspace{\algorithmicindent}\hspace{\algorithmicindent}\hspace{\algorithmicindent}\hspace{0.3cm}mascaras del objeto $O^{k}$ y  clasificadores de colores $C^{k}$ para
        \State\hspace{\algorithmicindent}\hspace{\algorithmicindent}\hspace{\algorithmicindent}\hspace{0.3cm}los $k$ cuadros previos.

        \Ensure \hspace{\algorithmicindent}\hspace{0.23cm} Mascara del objeto recuperada $O^{t}$ (puede estar vacia) y su
        \State\hspace{\algorithmicindent}\hspace{\algorithmicindent}\hspace{\algorithmicindent}\hspace{0.3cm} correspondiente clasificador de color $C^{t}$
        \State

        \For{$k = t-1, t-2, ..., max{1, t-m_{f}}$}
            \If{$O^{k}$ no esta vacio}
                \State $M \gets $ CandidateMask($I^{(t)}, C^{(k)}$)
                \State Sea $\kappa$ la lista de regiones conexas en $M$
                \ForEach{region $K$ en $\kappa$}
                    \State ($S_{o}$,$S_{x} \gets$ SelectMarkers($I^{(k)},K$)
                    \State $K' \gets$ IFTSegment($I^{(t)}, C^{(k)}, S _{o}, S_{x}$)
                    \If{$K'$ es suficientemente similar a $O^{k}$}
                        \State $C' \gets$ BuildClassifier($I^{(1)}, O^{(1)}, D \ O^{(1)}$)
                        \State \Return ($K', C'$)
                    \EndIf
                \EndFor
            \EndIf
        \EndFor
        \State \Return ($\emptyset, \emptyset$)
    \end{algorithmic}
\end{algorithm}

Si en algún paso del algoritmo IFTrace, el seguimiento del objeto falla, se
procede a utilizar la heurística \textit{recoverObj} (Algoritmo
\ref{alg:IFTrace-algorithm2-recoverObj}), que sirve para localizar una región
conexa cuya forma y colores sea similar a la del objeto en alguno de los
cuadros anteriores.

La heurística se aplica utilizando un cuadro de referencia, e iterando hacia atrás
hasta encontrar un cuadro en el que se pueda recuperar el objeto, evitando
utilizar cuadros previos en los que no se pudo recuperar el objeto anteriormente.

Una vez ubicado el cuadro de referencia en el cual el seguimiento fue exitoso,
se construye una ``máscara candidata'' $M$ que indica las posibles ubicaciones
del objeto en el cuadro actual $t$. Esto es lo que hace el procedimiento \textit{CandidateMask}.
En este paso se
utiliza el clasificador de color del cuadro de referencia $k$, $C^{(k)}$. La máscara
etiqueta los píxeles del cuadro como ``posiblemente objeto''(1) o
``probablemente fondo''(0).

Esta máscara $M$ estará compuesta de cero
o más regiones conexas. Si el objeto está visible en el cuadro actual,
su proyección debería coincidir con alguna de estas regiones. Entonces
se procede a analizar cada una de estas regiones, para las cuales se obtienen
los dos conjuntos de marcadores $S_{0}^{(t)}$ y $S_{x}^{(t)}$, utilizando la misma técnica que en el
algoritmo \ref{alg:IFTrace-algorithm1-IFTrace}.

% TODO Lo parafraseé por la correción. Se entiende? (Seen by: Alvaneitor)
% Respuesta (eordano): Saqué ", utilizado como referencia" porque me pareció sonaba mal y minor edits
Luego, para cada región $K$, se utiliza nuevamente el algoritmo IFT para
conseguir una proyección $K'$ de la región y compararla con la máscara del
objeto en el cuadro $k$. En caso de ser similares, se construye el clasificador
de color $C'$ y se retorna como resultado, junto a la proyección $K'$ que es la
máscara representativa del objeto para el actual cuadro.

Cabe destacar que, por simplicidad, para la comparación de formas se utilizan
las Invariantes de Momento de Maitra (ver \cite{MaitraMomentInvariants}), y las
formas se consideran similares si la distancia Euclideana entre sus vectores de
invariantes es menor a 2. Sin embargo se podrían utilizar otros descriptores de
formas.

Si ninguna de las regiones resulta similar a la máscara del objeto,
\textit{recoverObj} intenta el mismo procedimiento utilizando como cuadro de
referencia el cuadro anterior. Luego de un número especificado de intentos, o
de agotar todos los cuadros, la heurística termina. En este caso, se retorna
una máscara vacía y un clasificador nulo, señalando que el objeto se perdió en
el cuadro actual. \textit{IFTrace} intentará entonces recuperar en el siguiente
cuadro.

\subsubsection{Selección de Marcadores}
El procedimiento \textit{selectMarkers} es el encargado de elegir dos conjuntos
de marcadores $R_{0}^{(t)}$ y $R_{x}^{(t)}$, los que están dentro de la máscara del objeto, y los que están a
su alrededor, respectivamente.

Los marcadores internos se eligen aplicando una erosión morfológica
\footnote{La erosión morfológica es una de las dos operaciones fundamentales de
la matemática morfológica. Consiste en reducir o ``erosionar'' los bordes de un
determinado objeto, reduciendo su tamaño. }
%TODO Mismo caso que morfological dilation, valdría la pena la refrencia a la wiki? O alguna imagen o ejemplo en un apendice?
% Respuesta (eordano): Creo que como son muy conocidas en tratamiento de imágenes no hace falta
 a la máscara del objeto, con radio $\delta_{o}$, y luego seleccionando los
 píxeles de la región resultante que tengan mayor probabilidad de ser seguidos
 por el algoritmo KLT. Específicamente, se construye la matriz:

\begin{equation}
    H[p] = \left[\begin{array}{cc}
                \sum_{q}(\frac{\partial L}{\partial x}[q])^2 & \sum_{q}(\frac{\partial L}{\partial x}[q])(\frac{\partial L}{\partial y}[q]) \\
                & \\
                \sum_{q}(\frac{\partial L}{\partial x}[q])(\frac{\partial L}{\partial y}[q]) & \sum_{q}(\frac{\partial L}{\partial y}[q])^2 \end{array}\right]
    \label{IFTrace-matrix-H}
\end{equation}


\noindent donde $L[p]$ es la luminancia del píxel $p$, y las sumatorias incluyen a todos los
píxeles $q$ en una ventana de $9x9$ centrada en $p$.\\
Se define el grado de ``similaridad'' $\lambda[p]$ como el valor mínimo singular de $H[p]$.
Solo los píxeles con $\lambda > 1$ son escogidos para utilizarlos en la
propagación KLT. Esto coincide con los autores Tomasi y Karade (ver \cite{Tomasi91detectionand})
quienes afirman que los píxeles con mayor valor de $\lambda$ tienen más posibilidades
de ser localizados por el algoritmo KLT.\\
Los marcadores externos son escogidos aplicando una dilatación morfológica
\footnote{La dilatación morfológica es otra de las operaciones  básicas de la matématica morfológica. Generalmente utiliza un elemento
estructurador para expandir alguna forma contenida en la imagen.}
% TODO maybe poner refrencia a la wiki o la imagen del ejemplo en algun apendice?
%  Por ejemplo, si se utiliza un disco para dilatar un cuadrado, se logra un cuadrado mayor con los bordes redondeados
% (eordano) Misma respuesta que TODO anterior
a la máscara del objeto $0$
,con radio $\delta_{x}$, y tomando los píxeles a lo largo del borde del objeto.

\subsubsection{Seguimiento de Marcadores}

El procedimiento \textit{trackMarkers} se utiliza para localizar los marcadores
internos en el cuadro actual, que se corresponden con los puntos internos del objeto
en el cuadro anterior. Utiliza el algoritmo de seguimiento KLT, descrito por Tomasi y
Karade (ver \cite{Tomasi91detectionand}).

El algoritmo KLT recibe un punto $p$ en un cuadro $I^{t-1}$, una posición estimada
$q_{0}$ en el próximo cuadro $I^{t}$, y busca un punto $q$ tal que los vecinos
de $q$ en $I^{t}$ sean similares a los de $p$ en $I^{t-1}$. Este algoritmo tiene
varios parámetros de configuración que alteran su comportamiento: $\ell$, la cantidad
escalas consideradas; $\kappa$, el factor de reducción entre las sucesivas escalas; y
$\omega$, el ancho de la ventana usada para comparar vecinos en cada escala.
La implementación de IFTrace está configurada para usar $\ell=2$,$\kappa=4$ y
$\omega=9$, como en la implementación de KLT de Birchfield (ver \cite{Birchfield-KLT-implementation}).

Los marcadores exteriores no son seguidos con KLT, ya que no tiene sentido debido
a que la mayoría se perdería por oclusión con el objeto o se seguiría el fondo, en
vez del objeto. En cambio, lo que se hace es trasladarlos de acuerdo a los
desplazamientos medios de los marcadores internos más cercanos. Esto tiende
a mantener estos marcadores fuera del objeto, pero cerca de su borde, aún cuando
hay rápidos cambios de tamaño o de forma (por ejemplo, rotaciones o movimiento de extremidades).

\subsubsection{Segmentación de objetos - IFT}

La IFT interpreta a la imagen $I$ como un grafo $G$, cuyos nodos son los
píxeles y cuyas aristas unen dos nodos si los píxeles que representan son
adyacentes. IFT toma como parámetros un conjunto de marcadores (píxeles) $S$,
una función de costo de arista $w$, y una función de conectividad $f$ que
asigna un costo de camino $f(\pi)$ a todo camino $\pi$ en $G$, dependiendo del
costo de sus aristas.

Para cada píxel $p$, IFT encuentra un camino directo óptimo (de costo mínimo),
que lo conecta con su raíz $R(p)$, perteneciente al conjunto $S$ de marcadores.
Estos caminos forman un bosque de caminos óptimos. Cada árbol en este bosque de
caminos óptimos tiene como raíz a algún marcador $r$, y agrupa a todos los
píxeles de la imagen para los cuales existe un camino a $r$ de costo menor
a cualquier otro camino a otro marcador en $S$. IFT también le asigna a cada
píxel $p$ un costo $V(p) = \pi(p)$, de valor igual al costo del camino mínimo
para llegar a $p$ partiendo desde su raíz, y un marcador raíz $R(p)$, el
marcador que se encuentra en la raíz del árbol al cual pertenece.

IFTrace depende de la IFT para segmentar los objetos a seguir. Usa la función
de conectividad $f_{max}$, que asigna a un camino $\pi$ el máximo costo para
todas las aristas en $\pi$. Los píxeles marcadores utilizados son tanto los
marcadores interiores, obtenidos por el algoritmo KLT, como los exteriores que
se ubican alrededor del objeto.

La segmentación obtenida al usar $f_{max}$ tiene una importante propiedad
\textit{minimax}. Sea la \textit{frontera} del objeto de interés el conjunto de
todas las aristas cuyos nodos pertenecen a distintos segmentos, dicha
segmentación maximiza el costo mínimo de todas las aristas en la frontera de la
segmentación. Esto significa que la segmentación IFT/$f_{max}$ es esencialmente
la misma que la famosa segmentación \textit{divisoria} (ver
\cite{watershed-segmentation}). Esto hace que sea particularmente apropiada
para casos en los que el objeto está mejor caracterizado por su conectividad
que por su forma o tamaño. Es por esto que se elige una función de costo de
arista $w(p,q)$ que se basa en la probabilidad de que $p$ y $q$ estén en
diferentes lados del borde del objeto.

\subsubsection{Algoritmo IFT}

Para algunas funciones de conectividad, como $f_{max}$, el bosque de caminos
óptimos se puede computar eficientemente mediante una variante del famoso
algoritmo de Dijkstra (ver \cite{watershed-segmentation}). Por cuestiones de
eficiencia, el algoritmo retorna un mapa de raíces $R$, que almacena, para
cada nodo $p$, su raíz $R(p)$.\\

\begin{algorithm}
    \caption{Algoritmo IFT con $f_{max}$}
    \label{fig:IFTrace-IFT-algorithm}
    \begin{algorithmic}
        \Require\hspace{\algorithmicindent}\hspace{\algorithmicindent}Grafo $G_{1}$, conjunto de semillas $S = S_{o} U S_{x}$

        \Ensure \hspace{\algorithmicindent}\hspace{0.23cm} Bosque de camino optimo $P$, mapa de conectividad $V$
        \State\hspace{\algorithmicindent}\hspace{\algorithmicindent}\hspace{\algorithmicindent}\hspace{0.3cm} y mapa de raiz $R$.

        \Auxiliary\hspace{\algorithmicindent} Cola de prioridades Q, variable $tmp$

        \State

        \ForEach{$p \in G_{1}$}
            \State $P(p) \gets$ nil, $R(p) \gets$ p, $V(p) \gets + \infty$
            \If{ $p \in S$} insertar $p$ en $Q$ y setear $V(p) \gets 0$ \EndIf
        \EndFor
        \While{$Q \neq \emptyset$}
            \State Remover $p$ de $Q$ tal que $V(p)$ sea minimo.
            \ForEach{$q$ 4-vecino de $p$, tal que $V(q) > V(p)$}
                \State Computar $tmp \gets max\{V(p), w(p,q)\}$
                \If{$tmp < V(q)$}
                    \If{$V(q) \neq + \infty$} remover $q$ de $Q$ \EndIf
                    \State $P(q) \gets p$, $R(q) \gets R(p)$, $V(q) \gets tmp$
                    \State Insertar $q$ en $Q$
                \EndIf
            \EndFor
        \EndWhile
    \end{algorithmic}
\end{algorithm}



Se agregan
los nodos que representan marcadores a la cola $Q$. El ciclo \textit{while}
principal computa un camino óptimo desde las raíces a cada nodo $p$. En
cada iteración, un camino de valor $V(p)$ mínimo se encuentra cuando
removemos el último píxel $p$ de la cola. Luego se evalua si el camino que
alcanza el píxel adyacente $q$ a través de $p$ es más barato que el actual
camino que termina en $q$ y se actualizan $Q$, $V(q)$, $R(q)$ y $P(q)$.

Este algoritmo se puede modificar para recomputar el bosque de camino óptimo
en forma incremental, a medida que se van agregando o quitando marcadores,
generalmente en tiempo sub-lineal.

\subsubsection{Comparación con Corte de Grafos}
% Comparación con otros algoritmos de Graph-Cut

Boykov and Funka-Lea [5,12] han propuesto otra alternativa para la segmentación
de imágenes basada en grafos, conocida como Corte de Grafos.
En este enfoque, el grafo de píxeles se modela como una red de flujo,
donde el objeto es la fuente y el fondo el sumidero. Se puede probar que el máximo flujo total de todas las fuentes
hacia todos los sumideros es igual a la mínima capacidad total de todo corte
(conjunto de aristas) que separa fuentes de sumideros. Este flujo máximo y su
corte mínimo asociado se puede encontrar con el clásico algoritmo
Ford Fulkerson (ver \cite{Cormen:2009:IAT:1614191}).

% En comparación con la IFT, \textit{Graph-Cut} tiene 2 grandes desventajas: es
En comparación con la IFT, \textit{Corte de Grafos} tiene 2 grandes desventajas:
tiene un mayor costo computacional (IFT es $O(n)$ mientras que
% el mejor algoritmo \textit{Graph-Cut} es $O(n^{2.5})$), y además tiende a
el mejor algoritmo \textit{Corte de Grafos} es $O(n^{2.5})$), y además tiende a
minimizar el número de aristas en el corte, en vez de sus capacidades. Esto
último quiere decir que muestra una preferencia a segmentar basada en la
longitud del borde, en lugar de basarse en la importancia de la conexión
entre el objeto y el fondo (ver \cite{journals/jmiv/MirandaF09}).

Se puede introducir una corrección al algoritmo de \textit{Corte de Grafos} para
remediar este segundo problema, pero lo único que se obtiene es el mismo
resultado de la IFT, con un costo computacional mucho mayor (ver \cite{journals/jmiv/MirandaF09}).

\subsubsection{Costo de arista de IFT}

Como se explicó anteriormente, IFT opera con una función de costo de arista
$w$, la cual resulta crítica para la segmentación. Idealmente, el costo
para aristas que cruzan el borde del objeto debe ser alto, y bajo para todo
el resto. En otras palabras, la función de costo debe ser un detector
de bordes del objeto.

Un ejemplo puede ser la distancia Euclideana entre los colores RGB de los píxeles
en ambos extremos de una arista. Desafortunadamente, este detector resulta demasiado
simplista para casos prácticos, en los que un objeto puede tener diferentes colores y
texturas. Por lo tanto, se requieren detectores de bordes más sofisticados.

En la implementación de \textit{IFTrace}, se utiliza un detector de bordes basado en
una combinación lineal

\begin{equation}
   \label{eq:IFTrace-edge-detector}
   w(p,q) = \gamma w_{f}(p,q) + (1 - \gamma)w_{0}(p,q) = \gamma |\nabla I| + (1 - \gamma)|\nabla M|
\end{equation}

donde $\gamma$ es un parámetro definido por el usuario, $\nabla I$ es el gradiente
del color de la imagen y $M$ es un mapa de clasificación de color.

La primer componente $w_{f}(p,q)$ es la distancia Euclideana entre los colores
de los extremos de la arista, como se menciona anteriormente pero con una
particularidad: se mide en el espacio de colores $Lab$ de $CIE$ \footnote{ Un
  espacio de colores \textit{Lab} es un espacio de colores de 3 dimensiones,
  $L$ para la luminosidad o claridad, $a$ para la posición entre rojo y verde y
  $b$ para su posición entre amarillo y azul. Existen dos espacios muy similares
  \textit{Hunter Lab} y \textit{CIE Lab}, que se diferencian en la forma de
  calcular las coordenadas de color a partir de los datos: el primero utiliza
  raíces cuadradas y el segundo raíces cúbicas.} en lugar del tradicional RGB.
La justificación para esto es que las distancias entre colores se aprecian
más de esta forma. %TODO VER TEMA DE CIELAB y references

La segunda componente $w_{o}(p,q)$, es el gradiente del mapa de clasificación
de colores $M$, una imagen en escala de grises donde cada píxel $M[p]$ es
la probabilidad de que un píxel con color $v=I[p]$ pertenezca a la proyección
del objeto. Los colores que ocurran solo dentro del objeto deberían estar
asociados al valor 1, los que solo ocurran en el fondo deberían estar
asociados al 0, y los que puedan ocurrir en ambos, deberían mapearse a
valores intermedios. El propósito de incluir este término es restarle
importancia a los bordes entre colores que son internos del objeto, o
totalmente ajenos al objeto, y enfatizar los bordes que efectivamente
delimitan el objeto del fondo de la imagen.

El mapa de clasificación de colores $M$ se obtiene de la imagen $I$ por medio de una
función de clasificación difusa $C$, una función no lineal $C : \mathbb{V} \to [0,1]$.
En \textit{IFTrace}, la función $C$ se implementa como una variante del clasificador
del vecino más cercano, \textit{nearest neighbor} (NN) y determina dos conjuntos de
colores $\mathbb{U}_{0}$ y $\mathbb{U}_{x}$, que se asumen son representativos del objeto de interés y del fondo, respectivamente.
Para evaluar $C(v)$ para
cierto color $v$, se debe buscar 2 colores representantes $u_{0} \in \mathbb{U}_{0}$
y $u_{x} \in \mathbb{U}_{x}$ que son los más cercanos a $v$.
La probabilidad de que un píxel de color $v$ sea parte del objeto es entonces estimada
por la fórmula

\begin{equation}
   \label{eq:IFTrace-color-classifier}
   C(v) = \frac{|v - u_{0}|}{|v - u_{0}| + |v - u_{x}|}
\end{equation}

En teoría, se podría usar todos los píxeles para construir los conjuntos
$\mathbb{U}_{0}$ y $\mathbb{U}_{x}$. Pero en la práctica, se deben usar muestras
de menor tamaño por razones eficiencia, de otra forma, el costo de encontrar los
representantes $u_{0},u_{x}$ aumenta demasiado. De hecho, la construcción del mapa
$M$ resulta ser el paso que tiene mayor costo computacional de \textit{IFTrace}, más
aún que el seguimiento de marcadores y que la segmentación de la IFT.

Es por esto que el procedimiento para construir el clasificador,
\textit{buildClassifier}, comienza seleccionando dos conjuntos de píxeles
$R_{0},R_{x}$, respectivamente los que están dentro y fuera de la actual máscara
del objeto. Pero si alguno de estos conjuntos tiene más de 200 elementos,
entonces se realiza un muestreo aleatorio para reducirlos a ese tamaño.

\subsection{Contornos Activos}

La segmentación basada en contornos activos se basa en definir una región en base a su contorno.
El contorno o borde de una región $\Omega_i$ se representa por una curva paramétrica dada por:

\begin{equation}
    C_i(s) = (x_i(s), y_i(s))
\end{equation}

donde $0 \leq s \leq S_i$, y $C_i(0) = C_i(S_i)$, siendo $S_i$ el total de puntos que conforman la curva.
Es decir, $C_i$ es una curva cerrada.

Se definen tantos contornos como objetos de interés haya en la secuencia ($i$ es un entero entre 1 y el número de objetos).
Adicionalmente, se considera una región $\Omega_0$ que contiene a todo punto que no forma parte de ninguna otra región.
Se la denomina \textit{región de fondo}.

Se declara una función de probabilidad $p$ que permite saber que tan probable es que un píxel forme parte de una dada región.
Para esto, es necesaria una función $v$ que dado un píxel, devuelve un vector $v(x)$ de caracteristicas, por ejemplo los valores RGB del pixel.
Esto permite calcular $p(v(x) \vert \Omega_i)$, la probabilidad de que un píxel $x$ forme parte de una región $\Omega_i$ .
Estas funciones dependen de la secuencia particular a ser analizada, por lo que varían dependiendo del caso de estudio.
Como ejemplo, se toman los valores RGB como vector de características y
$p(v(x) \vert \Omega_i) = \| v(x) - v_i \| $, donde $v_i$ es el valor promedio RGB de todo pixel en la región $\Omega_i$.

Se define la función de energía de los contornos:

\begin{equation}
    \label{eq:ac-energy}
    E = - \sum_{m=0}^{M}{\int_{\Omega_m}{\log{p(v(x) \vert \Omega_m)} dx} + \lambda \int_{C_m}{ds}}
\end{equation}

Los algoritmos basados en contornos activos, buscan minimizar esta ecuación. Si el valor de $E$ es mínimo para un cuadro, se
debería haber encontrado la mejor aproximación de la región de los objetos interés. De \ref{eq:ac-energy} se
deriva la ecuación de evolución de cualquier curva $C_m$:

\begin{eqnarray}
    \frac{dC_m}{dt} &=& (F_d + F_s) \overrightarrow{N}_{C_m} \\
    F_d &=& \log{p(v(x) \vert \Omega_m) / p(v(x) \vert \Omega_0)} \\
    F_s &=& \lambda \kappa_m \label{eq:ac-formal}
\end{eqnarray}

$F_d$ y $F_s$ son fuerzas derivadas de la ecuación de energía. $F_d$ representa la competencia entre regiones y $F_s$ un
suavizado. $\kappa_m$ es la curvatura de la curva $C_m$.

Existe otra formulación del problema de evolución de curvas, que consiste en
representar la curva como el nivel cero de una superficie $\phi$ (ver \cite{Osher88}).

\subsection{Implementación Numérica}

Se toma de referencia la implementación segun \citeauthor{fast-level-set} (ver \cite{fast-level-set}).
En la implementación, el borde de un contorno $C_m$ se representa usando dos conjuntos de píxeles
correspondientes a los bordes interno y externo, $L_{in}$ y $L_{out}$
respectivamente. Entonces, la evolución se realiza intercambiando píxeles entre
estos dos conjuntos.

A continuación se detalla la técnica para la segmentación de un solo objeto de
interés que se puede fácilmente extrapolar y utilizar para la
segmentación de múltiples objetos
\footnote{Para la segmentación de múltiples objetos basta con marcar las distintas regiones con un identificador
distinto y seguirlas por separado.}.

El objeto de interés $\Omega_{1}$ y el fondo $\Omega_{0}$ cumplen
$\Omega_{1}\cup\Omega_{0} = I_{k}$ donde $I_{k}$ es la imagen del cuadro $k$ de
la secuencia de imágenes, y $\Omega_{1}\cap\Omega_{0} = \emptyset$. Cada una de
las regiones está caracterizada por su vector característico $v_{m}, m =
\{0,1\}$.

Se define una función $\phi(x)$ que indica si un píxel $x$ pertenece a una región o al fondo de la siguiente manera:

\begin{equation}
\phi(x) =
\left\{
    \begin{array}{ll}
        3  & \mbox{si } x \in \Omega_{0} \mbox{  y  } x \notin L_{out} \\
        1  & \mbox{si } x \in L_{out}\\
        -1  & \mbox{si } x \in L_{in}\\
        -3 & \mbox{si } x \in \Omega_{1} \mbox{  y  } x \notin L_{in} \\
    \end{array}
\right.
\end{equation}

Los conjuntos $L_{in}$ y $L_{out}$ se definen como

\begin{equation}
    L_{in} = \{ x \mbox{ es un píxel } \vert \mbox{    }  \phi(x) < 0 \mbox{ y } \exists y \in N_{4}(x) \mbox{ de modo que } \phi(y) > 0 \}
\end{equation}

\begin{equation}
    L_{out} = \{ x \mbox{ es un píxel } \vert \mbox{    } \phi(x) > 0 \mbox{ y } \exists y \in N_{4}(x) \mbox{ de modo que } \phi(y) < 0 \}
\end{equation}

donde $N_{4}(x) = \{ y \mbox{ es un pixel } \vert \mbox{   } |x-y| = 1 \}$, son los píxeles
vecinos del píxel $x$.

El algoritmo de segmentación es un algoritmo de dos ciclos, ya que luego de la
especificación inicial de la curva en forma supervisada \footnote{Podría no ser
supervisada. Existen variantes con determinaciones semi-supervisadas del objeto
de interés, así como también detección automática basada en ciertas
características predefinidas.} se intercambian los píxeles de $L_{in}$ y
$L_{out}$ en dos ciclos. En el primero, se aplica la fuerza $F_{d}(x)$, y en el
segundo se aplica $F_{s}(x)$ para la regularización.

En el primer ciclo, se ejecutan los siguientes pasos $N_{a}$ veces, donde $ 0 <
N_{a} < max(filas, columnas)$.

\begin{enumerate}

    \item Para cada $x \in L_{out}$, si $F_{d}(x) > 0$ entonces borrar $x$ de $L_{out}$ y agregarlo a $L_{in}$. \\
    Luego, $\forall y \in N_{4}(x)$, con $\phi(y) = 3$, agregar $y$ to $L_{out}$ y hacer $\phi(y) = 1$.

    \item Después del paso 1 algunos de los píxeles $x$ en $L_{in}$ pasan a ser píxeles internos. \\
    Por lo tanto, se sacan de $L_{in}$ y se hace $\phi(x) = -3$.

    \item Para cada $x \in L_{in}$ , si $F_{d}(x) < 0$ entonces, borrar $x$ de $L_{in}$ y agregarlo a $L_{out}$. \\
    Luego, $\forall y \in N_{4}(x)$, con $\phi(y) = -3$, agregar $y$ a $L_{in}$ y hacer $\phi(y) = -1$.


    \item Después del paso 3 algunos de los píxeles $x$ en $L_{out}$ pasan a ser píxeles externos. \\
    Por lo tanto, se sacan de $L_{out}$ y se hace $\phi(x) = 3$.

\end{enumerate}

%TODO Maybe poner footnote o algo sobre filtros Gaussianos? Apendice? (Seen by: Alvaneitor)
En el segundo ciclo, la curva se suaviza utilizando un filtro Gaussiano, de tal
forma que la fuerza de evolución es $F_{s}(x) = G \otimes \phi(x)$. Para aplicar $F_s$ se usan los mismos
pasos que para $F_d$. El resultado final es análogo a modificar la curva de acuerdo
a la definición formal dada anteriormente (Ecuación \ref{eq:ac-formal}).

En cada cuadro, el borde del contorno del objeto es actualizado de acuerdo al
resultado obtenido por el algoritmo en el cuadro anterior. En el caso de la
primera imagen, se puede dar de forma supervisada o semi-supervisada por el
usuario, o bien puede ser obtenida de forma automática mediante algoritmos de
aprendizaje complejos basados en características predefinidas.

El algoritmo termina cuando se alcanza la condición de corte, dada por las
ecuaciones \ref{eq:active-contours-stoppingCondition} o cuando se alcanza el numero de iteraciones $N_a$

\begin{equation}
\label{eq:active-contours-stoppingCondition}
    \begin{array}{ll}
        F_{d}(x) \leq 0 & \forall x \in L_{out}\\
        F_{d}(x) \leq 0 & \forall x \in L_{in}
    \end{array}
\end{equation}


\section{Homografía}
\label{sec:homography}

\subsection{Aplicación y utilización}

La técnica de la homografía se utiliza para poder transformar la vista
tridimensional de la secuencia de imágenes (la imagen del estadio de futbol), en
una vista de dos dimensiones (el campo de futbol visto desde arriba). Esto
facilita el análisis y los cálculos de las velocidades y posiciones de los
jugadores, lo cual permite sacar conclusiones y efectuar juicios complejos como
en el caso de la regla del fuera de juego. Además, visualmente elimina
ambigüedades que pueden suceder en el caso de la vista tridimensional por
razones de perspectiva, lo cual facilita el entendimiento al ser humano.

Considérese dos imágenes, $f(x,y)$ y $f'(x,y)$, relacionadas por una transformación geométrica. Sean $p_{k}$ y $p'_{k}$ para $k = 0 ... N$ puntos de las imágenes
$f$ y $f'$ respectivamente, y dadas correspondencias tentativas $p_{k} \to p'_{k}$, se quiere estimar la transformación $T$, tal que

\begin{equation}
    f(x,y) = f(T(x,y))
\end{equation}

Esta transformación $T$ se puede modelar como una transformación de coordenadas lineales

\begin{equation}
    \begin{bmatrix}
        x_{0} \\
        y_{0} \\
        z_{0} \\
    \end{bmatrix}
    = H
    \begin{bmatrix}
        x_{1} \\
        y_{1} \\
        1 \\
    \end{bmatrix}
\end{equation}

en donde $H$ es una matriz de $3 \times 3$ que representa la proyección, rotación, escalamiento, sesgo y perspectiva.

Se puede observar que para aplicar $H$ se extiende el vector de dos dimensiones $[x,y]^{T}$ a tres componentes. El valor de la
tercera componente puede variar, y define una clase equivalencia tal que

\begin{equation}
    \begin{bmatrix}
        zx \\
        zy \\
        z \\
    \end{bmatrix}
    \equiv
    \begin{bmatrix}
        x \\
        y \\
        1 \\
    \end{bmatrix}
\end{equation}

Este tipo de coordenadas se denominan homogéneas.

La forma de $H$ determina el tipo de transformación geométrica representada. Por ejemplo,

\begin{equation}
    H =
    \begin{bmatrix}
        sa\cos(\theta) & -sb\sin(\theta) & t_{x}\\
        sa\sin(\theta) & sb\cos(\theta) & t_{y} \\
        p_{0}          & p_{1}          & 1     \\
    \end{bmatrix}
\end{equation}

Representa una rotación de ángulo $\theta$, una traslación dada por $t_{x}$ y $t_{y}$ (notesé el uso del ``1'' de coordenadas
homogéneas), una escalamiento dado por el factor $s$, un sesgo introducido por $a$ y $b$ y un cambio de perspectiva
dado por $p_{0}$ y $p_{1}$.

En total, son 8 los parámetros que definen la matriz $H$, y sus elementos son

\begin{equation}
    H =
    \begin{bmatrix}
        H_{00} & H_{01} & H_{02}\\
        H_{10} & H_{11} & H_{12}\\
        H_{20} & H_{21} & 1\\
    \end{bmatrix}
\end{equation}

\subsection{Estimación de una homografía}

Se pueden estimar los 8 parámetros desconocidos de la matriz de transformación $H$, basándose en correspondencias de puntos conocidas.
La transformación de una coordenada $x$ en coordenadas homogéneas ($z=1$), a una coordenada objetivo $x'= Hx$ da como resultado

\begin{eqnarray*}
    x' &=& H_{00}x + H_{01}y + H_{02}\\
    y' &=& H_{10}x + H_{11}y + H_{12}\\
    z' &=& H_{20}x + H_{21}y + H_{22}\\
\end{eqnarray*}

Diviendo las primeras 2 ecuaciones por $z'$ para convertirlas en coordenadas Euclideanas, se puede llegar a

\begin{eqnarray*}
    \frac{x}{z'}(H_{20}x + H_{21}y + H_{22}) - H_{00}x - H_{01}y - H_{02}\\
    \frac{x}{z'}(H_{20}x + H_{21}y + H_{22}) - H_{10}x - H_{11}y - H_{12}\\
\end{eqnarray*}

Teniendo varias correspondecias como la anterior, se las puede escribir en forma matricial:

\begin{flalign}
    Ah =
    \begin{bmatrix}
        -x & -y & -1 & 0 & 0 & 0 & \frac{x'x}{z'} & \frac{x'y}{z'} & \frac{x'}{z'}\\
        0 & 0 & 0 & -x & -y & -1 & \frac{y'x}{z'} & \frac{y'y}{z'} & \frac{y'}{z'}\\
          &   &   &    & \vdots & &               &                &\\
    \end{bmatrix}
    \begin{bmatrix}
        H_{00} \\
        H_{01} \\
        H_{02} \\
        H_{10} \\
        H_{11} \\
        H_{12} \\
        H_{20} \\
        H_{21} \\
        H_{22} \\
    \end{bmatrix}
    = 0
\end{flalign}
Cada correspondencia de puntos agrega dos filas a la matriz $A$, por lo que $n$ correspondencias generan una matriz de $2N \times 9$.

Como consecuencia, se tiene el siguiente sistema de ecuaciones lineales para resolver

\begin{equation}
    Ah = 0 \hspace{1cm} h \neq 0
\end{equation}

Con 4 puntos de correspondencias, el sistema es \textit{compatible determinado} y la única solución es el espacio nulo de $A$. Para más correspondencias,
el sistema pasa a ser \textit{compatible indeterminado} y de las infinitas soluciones, se busca la de menor norma 2.
%TODO maybe agregar alguna mini explicación de porque? (Seen by: Alvaneitor).

% TODO: Apenas entiendo que garcha pasa despues de esto.
\begin{equation}
    argmin_{\norm{h}=1} \norm{Ah} = argmin_{\norm{h}=1} h^{T}A^{T}Ah = \lambda_{min}
\end{equation}

donde $\lambda_{min}$ es el menor autovalor de $A^{T}A$. Esto se desprende de las propiedades de la norma 2 de matrices y de la propiedad de
las matrices cuadradas simétricas que dice que dado $B = A^{T}A$, $\lambda_{i}$ y $q_{i}$ sus autovalores y autovectores respectivamente, las matrices

\begin{equation}
    Q = \begin{bmatrix}
            q_{0} q_{2} \dots q_{n-1}
        \end{bmatrix}
\end{equation}
y

\begin{equation}
    D = \begin{bmatrix}
            \lambda_{0} & & & \\
                        & \lambda_{2} & & \\
                        & & \ddots & \\
                        & & & \lambda_{n-1}\\
        \end{bmatrix}
\end{equation}

cumplen $B = QDQ^{T}$.

\begin{equation}
    argmin_{\norm{h}=1} h^{T}QDQ^{T}h = argmin_{\norm{y}=1} y^{T}Dy = argmin_{\norm{y}=1} \lambda_{1}y^{2}_{1} + ... + \lambda_{n}y^{2}_{n}
\end{equation}

Con $\lambda_{i} = \lambda_{min}$, se alcanza un mínimo cuando todas las componentes de $y$ se anulan excepto por $y_{i} = 1$. Como $y = Q^{T}h$,
$h=Qy = q_{min}$, el autovector de $B$ correspondiente al mínimo autovalor.

El problema se reduce entonces a hallar este autovector. Para ello, se recurre a la Descomposición en Valores Singulares
%TODO footnote o referencia? (Seen by: alvaneitor)
\begin{equation}
    A = U\Sigma V^{T}
\end{equation}

donde las columnas de $U$ contienen los autovectores de $AA^{T}$ y las columnas de $V$ los autovectores de $A^{T}A$, mientras que $\Sigma$ es una matriz
diagonal con los autovalores de $$A^{T}A$$ en su diagonal. Nótese que los autovalores de $A^{T}A$ son iguales a los autovalores de $A$ al cuadrado.

La Descomposición en Valores Singulares se calcula de tal forma que los autovalores en la diagonal de $\Sigma$ aparecen en orden decreciente. Por lo tanto
se toma como solución la última columna de $V$, correspondiente al menor autovalor de $A^{T}A$.

\subsection{Distorsión de la lente de la cámara}

El material utilizado fue obtenido con una lente con una baja distancia focal,
lo que genera una distorsión de tipo \textit{barril}, donde las líneas rectas
se curvan hacia el exterior. El modelo de \textit{Brown} plantea una solución
general para múltiples niveles de distorsión, tanto de tipo radial como
tangencial.

Para obtener los valores de la imagen sin distorsión se utilizó la siguiente
corrección:

\begin{eqnarray*}
    x_u &=& (x_d - x_c) (1+K r^2) \\
    y_u &=& (x_d - x_c) (1+K r^2)
\end{eqnarray*}

Donde $x_u$ y $y_u$ son los puntos en la imágen sin distorsión, $x_d$ y $y_d$ son
puntos en la imagen original, $x_c$ y $y_c$ son los puntos centrales de la
imagen (se puede asumir que el punto central de la cámara es el punto central
de la imagen), $K$ es un coeficiente de distorsión radial, y $r =
\sqrt{(x_d-x_c)^2) + (y_d-y_c)^2}$, la distancia del punto al centro de la
imagen.


\section{Análisis de partidas de Fútbol}
\label{sec:futbol}

\subsection{Análisis utilizando 6 cámaras}
\label{sec:6-camaras}

\citeauthor*{papers-tanos} estudian la viabilidad de juzgar si en un partido de fútbol
se dió una posición adelantada, utilizando seis cámaras dispuestas en ambos
laterales de la cancha para reducir errores de medición y perspectiva. Las
imágenes obtenidas por las cámaras son sincronizadas y procesadas para obtener
la posición de cada jugador y de la pelota en todo momento, y a partir de estos
datos detectar pases de la pelota para finalmente detectar si se cometió una
falta de posición adelantada.

El sistema consiste de seis cámaras de alta resolución, tres en cada lateral de
la cancha, con sus ejes ópticos paralelos a las líneas de llegada de la cancha
con el objetivo de reducir errores de perspectiva. Cáda cámara envía sus
imágenes a una de seis computadoras dispuestas con el objetivo de analizar las
imágenes y detectar la posición de los jugadores, la pelota, y posibles pases
entre jugadores de un mismo equipo. Un computador central recibe estos datos y
valida que las mediciones de distintas cámaras sean congruentes.

A continuación se presentan las técnicas de análisis de imágen aplicadas.
% TODO (eordano): escribir "nivel local de energía en ventana temporal de mejor manera
% TODO (eordano): buscar cita para esta técnica (21 en el paper de los tanos)

\subsubsection{Extracción del fondo}
Un algoritmo de eliminación de fondo basado en una técnica de nivel local de
energía en ventana temporal detecta aquellas áreas candidatas a contener a
alguno de los jugadores en la cancha, la pelota, al referí o jueces de línea.
Según los autores, este algoritmo es lo suficientemente robusto y confiable y
detecta correctamente objetos en primer plano, tanto en movimiento como quieto
durante un largo tiempo. Se analiza la conexidad de las áreas candidatas de
acuerdo a su topología para descartar sombras que puedan llegar a generar
errores en análisis posteriores.

Este análisis utiliza la técnica de ``ventana deslizante'', ya que el análisis
se basa en una cantidad de imágenes consecutivas de la secuencia (se referirá a
esta subsecuencia como $W$ y se la denominá ``ventana temporal'' o simplemente
``ventana''). Los autores del trabajo utilizaron una ventana de 60 cuadros,
equivalente a $2.5$ segundos, valor obtenido empíricamente y suficiente, ya que
se determinó que los objetos en primer plano (principalmente los jugadores) no
se mantenían durante tanto tiempo en posición completamente estática.

Inicialmente, la energía de un punto $(x, y)$ en una imágen es definida por la
ecuación \ref{eq:tanos-energy}:

\begin{equation}
    \label{eq:tanos-energy}
    E(x, y) = \sum_{t \in W} \| I^t(x, y) - B_C (x, y) \| ^2
\end{equation}

Donde  $I^t(x, y)$ es la intensidad del punto en coordenadas $(x, y)$ para la
imágen $t$, y $B_C(x, y)$ es una estimación de la intensidad del fondo para el
punto $(x, y)$. Para determinar $B_C$ se aplican sucesivos filtros gaussianos
al primer cuadro de $W$.

Para la siguientes ventanas se refina el modelo del fondo de la imágen. Se
calcula la función $B_F$, definida para todo punto de un cuadro, cuya imágen es
$R^2$ y corresponde a la media y la desviación estándar de $E(x, y)$.

Se define $B_F$ inicialmente mediante la ecuación \ref{eq:tanos-bf1}:

\begin{equation}
  \label{eq:tanos-bf1}
  B_F(x, y) =
  \begin{cases}
    \mu(x, y), \sigma(x, y) & if E(x, y) < th(W) \\
    \phi & if E(x, y) > th(W)
  \end{cases}
\end{equation}

Donde $th(W)$ es un valor de umbral obtenido empíricamente, proporcional al
tamaño de $W$. Un bajo nivel de energía corresponde a un punto estático,
por lo tanto se considera que ese punto pertenece al fondo y $B_F$
contendrá la media y la desviación estándar para ese punto. Si la imágen
tiene un alto valor de energía, se considera al punto un candidato a
contener un objeto en movimiento y por lo tanto el valor de $B_F$ es
indefinido.

En sucesivas ventanas se utiliza un parámetro de actualización $\beta$ (los
autores utilizaron un valor de $0.1$) y $B_F$ es recalculado en base a su valor
anterior y nuevos valores de $E(x, y)$, utilizando la ecuación
\ref{eq:tanos-bf2}:

\begin{equation} \label{eq:tanos-bf2}
  B_F(x, y) = \begin{cases}
    \mu(x, y), \sigma(x, y) & if E(x, y) < th(W) \wedge B_F(x, y) = \phi \\
    \beta B_F(x, y) + (1-\beta) \mu\left[mu(x, y), \sigma(x, y)\right] & if E(x, y) < th(W) \wedge B_F(x, y) \neq \phi \\
    \phi & if E(x, y) > th(W)
  \end{cases}
\end{equation}

Por último, se realiza un análisis de conectividad de los puntos de primer
plano y los conjuntos de puntos conexos son denominados sectores candidatos
a ser personas en la cancha o la pelota. Luego, se descartan sectores
cuyo tamaño no supere un umbral (determinado experimentalmente con el
objetivo de descartar sectores muy chicos) y se realiza un análisis de la
morfología de los conjuntos conexos, para eliminar la sombra proyectada por los
jugadores (el artículo publicado no entra en detalles acerca de este proceso).

\subsubsection{Clasificación de Jugadores y Referies}

Luego de identificar sectores candidatos, son clasificados para determinar a
que clase pertenece. Dado que los colores de las camisetas de los jugadores
no se conocen previamente, es necesario utilizar un mecanismo de clasificación
no supervisado que consiste de dos pasos: En primer lugar, se crean
las clases de objetos con un algoritmo de \textit{clustering} basado en una
versión modificada del algoritmo BSAS (ver \cite[BSAS]).
Segundo, el clasificador obtenido es utilizado para determinar la clase
de cada objeto candidato obtenido del análisis anterior.

El algoritmo BSAS sólo requiere para funcionar una medida de semejanza $d(x, C)$,
y un valor de umbral denominado $th$ para generar las clases de objetos.
Se realizan varias iteraciones, unificando clases de objetos si no difieren
por más de $th$.

Para clasificar a los objetos en su clase correspondiente se minimiza la
distancia determinada por:
\[
  C_k = \frac{1}{w_k+1}(w_k C_k + V)
\]

donde $C_k$ es el prototipo de la clase $k$, $V$ es el vector de
características asociadas a la clase $k$, y $w_k$ es el número de
objetos clasificados dentro de la clase $k$ de acuerdo a las últimas dos
ventanas temporales $W$.

\subsubsection{Seguimiento de Jugadores}

Se utiliza una técnica de seguimiento representando al área donde un jugador se
encuentra mediante una \textit{bounding box}.       % TODO traducir?
Dos \textit{bounding boxes} pueden estar en estado de colisión si sus lados se
intersectan. El vector de seguimiento de un jugador está definido por la tupla
$x_{ti} = (p_i, v_i, d_i, l_i, c_i, s_i)$, donde:

\begin{itemize}
  \item $p_i$, $v_i$, $d_i$ son la posición, velocidad, y dimensiones
    (ancho y alto del \textit{bounding box}) para el jugador $i$.
  \item $s_i$ define el estado del seguimiento, para resolver
    colisiones entre cuadros y la aparición o desaparición de
    sectores candidatos en el área.
  \item $c_i$ es la clase a la que pertenece el jugador.
  \item $l_i$ es una etiqueta identificadora del jugador, o un conjunto de
etiquetas si el cuadro está en estado de colisión.
\end{itemize}

Para cada nuevo cuadro, se actualiza la posición de la caja que contiene a
los sectores candidatos analizando el movimiento de sectores dentro del
\textit{bounding box} para actualizar la velocidad, posición y tamaño,
y respecto a otras cajas para actualizar su estado (si entró en contacto).
Para resolver situaciones en la que se esté terminando la colisión, se utilizan
datos correspondientes a la clase de los objetos detectados y la velocidad de
cada jugador previo a la colisión de las \textit{bounding boxes}.

Para la detección de la posición de la pelota, se utiliza un clasificador
entranado con un conjunto de entrenamiento de pelotas en distintas posiciones y
tamaños. Se elige el candidato a ser la pelota cuyo coeficiente de correlación
con el clasificador sea mayor en caso de que haya varios candidatos.

La velocidad de la pelota es determinada a partir de las observaciones de la
cámara por la ecuación:

\[
  V_x = \frac{(P_{x_t} - P_{x_{t-n}})}{n}, V_y = \frac{(P_{y_t} - P_{y_{t-n}})}{n}
\]

donde $P_{x_t}, P_{y_t}$ es la posición de la pelota en el cuadro $t$, y $n$ es
la cantidad de cuadros entre la imágen actual y la última ubicación correcta de
la pelota.

Sin embargo, para que una posición candidata para la pelota sea seleccionada
como un verdadero positivo, debe recaudarse información sobre más de un cuadro.
Para esto, se genera un mapa de probabilidad de que la pelota se encuentre en
cada punto de la imágen, basado en observaciones de cuadros anteriores. Esta
posibilidad está definida como sigue:
\[
  P(x, y) = \exp \left[ (- \vert ( x - \vert \tilde{x} + V_x sign(\cos \theta) \vert )
  + ( y - \vert \tilde{y} + V_y sign(\sin \theta) \vert)\vert ^ 2 / 2\theta^2) /\sigma \sqrt{2\pi} \right]
\]

donde $\tilde{x},\tilde{y}$ es la última posición conocida de la pelota,
\[
  \sigma = \frac{R_p V_{max} n}{R_{cm}T}
\]
donde $V_x$ y $V_y$ representan la velocidad de la pelota en las coordenadas
$x$ e $y$, $\theta$ es $\arctan(\tfrac{V_y}{V_x}$, $R_p$ es el radio de la pelota
en píxeles, $V_{max}$ es el máximo valor admitido para la velocidad de la
pelota, $R_{cm}$ es el radio de la pelota en centímetros, $T$ es la cantidad de
cuadros por segundo, y $n$ la cantidad de cuadros desde que la pelota fue
encontrada en la posición $\tilde{x},\tilde{y}$.

\subsubsection{Detección de Pases y Posición Adelantada}

Los datos analizados en una computadora por cada cámara son entonces transmitidos
a un servidor (llamado ``supervisor'') que unifica la información recibida de
cada cámara, considera sus posiciones relativas, y arma el estado del juego,
compuesto por el tiempo desde el inicio del partido, la posición de todos los
jugadores, el referí, jueces de línea, sus respectivas velocidades y estimación
de la aceleración, si la pelota fue detectada o no, la posición y la confianza
en la posición de la pelota. Los jugadores son posicionados a través de una
transformación homográfica y el supervisor correlaciona los datos provenientes
de distintas cámaras para formar un estado consolidado.

Para la detección de una jugada de posición adelantada, se requiere:

\begin{itemize}

  \item \textit{Determinar qué jugador pateó la pelota}: esta es la tarea más
    compleja del análisis. Se requiere determinar la posición de la pelota en
    tres dimensiones. Para lograr esto, se detecta la posición de puntos
    conocidos del campo de juego en la cámara y se utiliza la homografía para
    calcular la posición estimada de la pelota. Se hace uso también de la
    posición relativa de las cámaras en la cancha: dado que están enfrentadas,
    las mediciones de la posición de la pelota de dos cámaras enfrentadas
    debería ser similar. Se estima que el jugador más cercano a la pelota es
    aquél que la patea al detectar un cambio repentino en la velocidad de la
    pelota.

  \item \textit{Determinar si es una posición adelantada activa}: Se analiza
    la posición de jugadores en estado de ``posición adelantada pasiva''. De
    las reglas del juego, si un jugador está adelantado pero no recibe la
    pelota, la jugada puede seguir y no se ha cometido una falta. Durante los
    tres segundos posteriores a un pase largo (tres segundos es una estimación
    del tiempo que la pelota está en el aire desde que es pateada hasta que es
    recibida) el supervisor evalúa si los jugadores en posición adelantada
    interceptan la pelota, en cuyo caso, se anuncia que el jugador está en
    falta.

\end{itemize}

\subsection{Análisis utilizando 8 camaras}
\label{sec:8-camaras}

\citeauthor*{xu-8cams} proponen un sistema que consta de 8 módulos, conectados
a cámaras, que recolectan y envían información
a un módulo supervisor, encargado del mantenimiento de la posición de los
jugadores. El sistema supervisor no tiene acceso a los datos crudos de la
imagen obtenida, sino que reciben información procesada de cada nodo.

\subsubsection{Pre-procesamiento del video}

Cada nodo realiza un preprocesamiento de la imagen para estimar la posición
de la cámara y detectar áreas de interés (candidatas a ser los jugadores en
pantalla). El primer paso es la eliminación del fondo. Para ello, se utilizan dos máscaras,
una geométrica, que aproxima la geometría de la cancha, y otra que extrae
información acerca de los píxeles y genera un histograma para detectar el color
verde del pasto de la cancha. Para la primera máscara, se pasa la imagen del
espacio de colores
\textit{RGB} a \textit{HSI} y se analiza el histograma de \textit{hue} para los
valores de intensidad. Luego, los píxeles que serán considerados son
aquellos que no pertenezcan a un determinado rango de \textit{hue}.
Luego, la máscara que elimina valores de acuerdo a su color queda definida por:

\[
  M_c = ({(u, v) | H(u, v) \in [H_l, H_h]} \oplus B ) \ominus B
\]

Donde $H(u, v)$ es el valor de \textit{hue} del píxel en la posición $(u, v)$,
$B$ es un elemento estructurador cuadrado y $\oplus$ y $\ominus$ son los
operadores morfológicos de erosión y dilatación, aplicados para separar a los
jugadores y eliminar las líneas de campo blancas. La máscara geométrica
comprende los siguientes puntos:

\[
  M_g = { (u, v) | E(u, v)  \in P }
\]

Donde $P$ es el rango de coordenadas en el mundo real correspondiente a la cancha
y $E(u, v)$ es la correspondencia del punto $(u, v)$ en el mundo real, de
acuerdo a una corrección utilizando ángulos de Euler. Por último, la máscara que
determina aquellos píxeles que corresponden al pasto de la cancha son los que
pertenecen a la intersección de ambas máscaras:

\[
  M = M_c \cap M_g
\]

El segundo paso consiste en un proceso de seguimiento local. Cada nodo
determina pequeñas "cajas delimitadoras" alrededor de las áreas donde
es posible que haya jugadores. Estas cajas son
representadas por la posición en el plano de la imagen $\mathbf{x}_l$ y su
error en la medición $\mathbf{z}_l$ en un filtro de Kalman (ver \cite{funk2003study}):

\[
\mathbf{x}_l = [r_c \; c_c \;  \mathbf{\dot r}_c  \; \mathbf{\dot c}_c \;  \Delta r_1  \; \Delta c_1 \;  \Delta r_2 \;  \Delta c_2]^T
\]

\[
\mathbf{z}_l = [r_c \;  c_c  \; r_1  \; c_1  \; r_2  \; c_2]^T
\]

Donde $r_1 < r_2$ y $c_1 < c_2$ son los límites de la caja y
$r_c$ y $c_c$ sus centroides. Estos valores son
actualizados cuadro por cuadro, asumiendo que la variación en alto y ancho es
baja. Se utilizan ángulos de Euler para traducir estos valores a la posición
esperada dentro del plano de la cancha. La varianza es estimada utilizando el
jacobiano de la matriz de Euler.

Por último, se analiza la categoría de los
jugadores de acuerdo al histograma de colores, clasificando de acuerdo a los
cinco posibles uniformes (uno para cada equipo, uno para cada arquero, y los
árbitros).

\subsubsection{Seguimiento multi-cámara}

%TODO esta sección quizas podría tener un poco más de contenido. Pasa que el
% paper es fucking- disgusting! No se entiende un soto. Tebex? (by:Alvanator)
El módulo supervisor recibe los datos pre-procesados de los demás módulos.
Se unifican los valores obtenidos
de todos los módulos y se los asocia utilizando una matriz que es actualizada
de acuerdo a la distancia de Mahalanobis \footnote {La distancia de Mahalanobis
es una medida de distancia que sirve para determinar la similitud entre dos
variables aleatorias multidimensionales. Se diferencia de la distancia
euclídea en que tiene en cuenta la correlación entre
las variables aleatorias y posee invariancia de la escala.}. Si esta distancia queda por debajo de
un determinado umbral, se empieza a considerar que las mediciones de dos
cámaras distintas para dos "cajas delimitadoras" pasan a ser el mismo
jugador.

\subsection{Análisis de deportes con múltiples camaras}
\label{sec:var-camaras}

% SIFT Base
\subsubsection{Camera-based Observation of Football Games for Analyzing Multi-agent Activities}

\citeauthor*{beetz-05} analizaron, desde el punto de vista de la inteligencia
artificial, el problema de detectar el comportamiento de los jugadores dentro
de la cancha. Para eso, propusieron un sistema con dos módulos, uno que extrae
características e información de videos provenientes de mùltiples cámaras, y
un módulo de análisis y seguimiento del comportamiento de cada objeto.
Los objetos de interes son los jugadores, la pelota y los arbitros.

Encontraron una gran dificultad en detectar la posición de jugadores más alejados de la cámara,
dado que el material con el que trabajaron en general eran cámaras situadas a no más de 18
metros de altura. Esto genera una imagen borrosa de los jugadores que estan
en el otro extremo de la cancha, dificultando el analisis.

El primer modulo se encuentra encargado de procesar los cuadros de video y producir informacion
de posición de las regiones que potencialmente contienen los objetos de interes.
Como herramienta principal de detección, se utiliza la segmentacíon por colores.
Se conoce de manera supervizada las clases de colores que representan el pasto, los jugadores de un equipo,
los jugadores del otro equipo, los arbitros, las lineas de la cancha y la pelota.
Con esta informacion, es posible armar filtros complejos a la hora de buscar una clase de objeto particular.
Por ejemplo, para encontrar jugadores, tomo la region que queda de eliminar arbitros, lineas, pasto y la pelota,
y de esto, tomo las parte que cumple con la representación por colores de los jugadores.
Para evitar problemas de iluminación, cada un cierto numero de cuadros se re-aprende la clase de color
que corresponde al verde del pasto.

Para detectar las regiones de interes, se utilizan las clases de color para aislar potenciales regiones.
A partir de estas, se utiliza caracteristicas morfologicas de cada clase de objeto para eliminar regiones
invalidas. Se asume que los jugadores siempre van a estar parados, y por lo tanto tienen regiones rectangulares.
De la misma manera, se espera que la region que contiene a la pelota sea circular.
Notese que no se espera que los resultados obtenidos en este paso sean finales.
La información producida por este modulo es procesada por el 2do modulo para producir los resultados finales de
seguimiento.

Finalmente, la ultima tarea asignada al primer modulo es la de estimar los parametros de la camara.
Esto es necesario para poder asignar una ubicacion en el plano de la cancha a un objeto representado
por una region en el plano de la imagen.
De manera supervisada y previa a la ejecucion del algoritmo, se genera un modelo de la cancha.
Este modelo incluye lineas de la cancha, arcos y los carteles de propaganda.
Luego, durante la ejecución se aplica un algoritmo iterativo de 3 pasos:
\begin{enumerate}
\item Proyectar el modelo sobre la cancha
\item Utilizar algoritmos de deteccion de lineas para calcular el error de la proyección
\item Ajustar los parametros para minimizar el error de proyeccion
\end{enumerate}

Como existe la posibilidad de que no haya suficiente información en un cuadro de imagen para estimar
utilizando el metodo anterior, se utiliza ademas un algoritmo de extracción de caracteristicas y seguimiento.
En particular, la implementación de \citeauthor{shi-tomasi-tracking} (ver \cite{shi-tomasi-tracking}).

La información de regiones y sus posiciones es entregada al 2do modulo.
Este utiliza el algoritmo \textbf{Multiple Hypothesis Tracking(MHT)} (ver \cite{MHT-1, MHT-2}) mejorado por
\citeauthor*{Schmitt-1} (ver \cite{Schmitt-2}). Este algoritmo utiliza un modelo de cámara y movimiento
probabilístico para estimar la posición de los jugadores y actualizarlas en cada observación.
El algoritmo fue adapatado para tener en cuenta algunas restricciones encontradas en el futbol:
\begin{itemize}
\item Todos los objetos de interes se encuentran en el mismo plano
\item Los objetos de interes solo pueden desaparecer de la imagen por los bordes
\end{itemize}

Ya obtenida la posición de cada objeto de interes en la imagen, podemos hacer un analisis del estado del partido.
Los autores solo observan al jugador que tiene la pelota en un dado cuadro.
Para este jugador construyen eventos de movimiento. Un evento de movimiento es un cambio de posición del jugador
tal que puede ser representado por una function lineal. Por lo tanto, se termina describiendo el movimiento de un
jugador a lo largo de una jugada, de manera aproximada, como una serie de pequeñas lineas rectas.

Con estos eventos de movimiento, se construye un episodio. Esto es la serie de eventos de movimiento que realiza
un jugador desde que recibe la pelota en un cuadro $t_i$, hasta que ya no la tiene en su contral en un cuadro $t_f$.
Como no siempre se conoce con precisión la posición de la pelota, se ingresa de manera supervisada el momento en el
que un jugador toma contacto con la pelota, y cuando lo pierde.

Finalmente, se intenta clasificar los episodios segun su resultado: un pase, un tiro al arco, una evasión o perdida de pelota.
Para esto se utiliza un arbol de clasificaciones, con reglas simples para los primeros 3 casos.
En el caso de la perdida de pelota, se utiliza un arbol de decisión mas complejo.

\subsection{Análisis de deportes con video televizado}
\label{sec:tv-video}

\subsubsection{\citetitle{LIU20061146}}

En \cite{LIU20061146} se plantea una técnica para obtener las posiciones de los
jugadores y la pelota utilizando la transmision televizada de un partido de
futbol. El algoritmo se puede dividir en las siguientes etapas:

\begin{itemize}
  \item Estimación de la relación entre puntos en la imagen y coordenadas en la cancha
  \item Estimación de la posición de la camara en las coordenadas del mundo
  \item Detección y seguimiento de la pelota
  \item Detección de la cancha
  \item Detección de los jugadores
\end{itemize}

Para estimar la relación entre los puntos en la imagen y las coordenadas en la cancha se calcula una homografía.
Como la perspectiva de la cámara cambia cuadro a cuadro, esta debe ser calculada en cada cuadro nuevamente.
Si en la imagen actual se encuentran 4 puntos conocidos (esquinas de la cancha o de las areas), la homografía puede ser calculada directamente.
De no ser asi, se estima considerando que la homografía del cuadro actual $H_t$ se puede relacionar con $H_{t-1}$ mediante un \textit{Global Motion Parameter} $P$.
Este parametro $P_{t-1,t}$ se calcula tomando una serie de puntos en la imagen y relacionando su posición entre el cuadro $t-1$ y $t$.
Luego $ H_t = H_{t-1} P_{t-1,t}$.

Para obtener la posición de la cámara se descompone la homografía en 2 matrices:
\begin{eqnarray*}
H = K \begin{bmatrix} r_1 & r_2 & t \end{bmatrix} \\
K = \begin{pmatrix}
    \alpha & \gamma & u_0 \\
    0 & \beta & v_0 \\
    0 & 0 & 1
    \end{pmatrix}
\end{eqnarray*}

Donde $\alpha$, $\beta$ representan la amplitud focal de la cámara, $\gamma$ representa el \textit{skew} y $(u_0, v_0)$ es la coordenada del punto principal.
Se asume que la amplitud focal se mantiene constante, $\gamma = 0$ y el punto principal es el centro de la imagen.
$r_1$, $r_2$ son parte de una rotación $R = (r_1, r_2, r_1 \times r_2)$, y $t$ son las coordenadas del origen de la cancha en coordenadas de la imagen.
Conocidas $H$ y $K$ se puede calcular la posición de la cámara como $R^{-1} t$.
Para el cálculo de $K$ se necesitan $\alpha$ y $\beta$ que se calcula usando $P$ de la siguiente manera:

\begin{equation}
\begin{bmatrix}
    p_{1 1} p_{2 1} & p_{1 2} p_{2 2} \\
    p_{1 1} p_{3 1} & p_{1 2} p_{3 2} \\
    p_{1 1} p_{3 1} & p_{2 2} p_{3 2}
\end{bmatrix}
\begin{bmatrix}
    \alpha^2 \\
    \beta^2
\end{bmatrix}
 =
\begin{bmatrix}
    - p_{1 3} p_{2 3} \\
    - p_{1 3} p_{3 3} \\
    - p_{2 3} p_{3 3} \\
\end{bmatrix}
\end{equation}

Se usa el método \textit{KLT} (ver \cite{KLT}) para encontrar los puntos de referencia a seguir para el cálculo de $P$.

\subsubsection*{Seguimiento de la pelota}

El algoritmo planteado para seguimiento de la pelota tiene 2 etapas.
La primer etapa es de detección y utiliza un algoritmo basado en \textit{Viterbi}.
La segunda etapa es de seguimiento y utiliza un filtro \textit{Kalman} (ver \cite{funk2003study}).

El algoritmo de la primer etapa, crea una imagen binaria, segmentando según la característica blanca de la pelota.
Una vez segmentada, se eliminan los candidatos teniendo en cuenta las distintas características morfológicas de la pelota.
Esta operación se realiza en $T$ cuadros distintos, y a partir de esto se construye un grafo pesado como se explica a continuación.
En este grafo los nodos representan posibles candidatos de la posición de la pelota en los $T$ cuadros.
Cada nodo tiene asignado un peso según el grado de similaridad entre el candidato y una pelota ideal.
Luego, se coloca un vértice entre un nodo del cuadro $t$ y otro del cuadro $t + 1$ si son parecidos en caracteristícas y cercanos en posición.
Sobre este grafo se hace una búsqueda de camino óptimo. Los nodos que forman parte del camino óptimo se consideran entonces
como la pelota real.

%TODO cuando es ese "ya no se confia en el resultado del seguimiento"?? (Seen by:_alvaneitor)
% better? (champo)
Una vez detectada la pelota, se utiliza un filtro de \textit{Kalman}.
Cuando el filtro devuelve una posición con un margen de error mayor a un cierto umbral, se considera perdida.
En esa instancia se vuelve a detectar la pelota con el algoritmo de detección detallado anteriormente.

\subsubsection*{Detección de la cancha}

Para detectar la cancha y aislar el resto de los objetos, se computa un histograma de colores de la imagen.
Los autores consideran que el color de la cancha representa el mayor pico en el histograma.
Por lo tanto, buscan este pico en el histograma y toman todos los colores adyacentes en el histograma que esten dentro de un rango de ocurrencia relativo al pico principal.
Todo píxel cuyo valor caiga dentro de ese rango se considera parte del fondo.

\subsubsection*{Detección de jugadores}

Una vez que se determinó el rango de colores de la cancha, se construye una imagen binaria separando la cancha de todo otro objeto.
Construida esta imagen, se aplica la misma técnica que se usa para seguir la pelota, adaptada a la morfología de los jugadores.
Una vez que se conoce la posición en el cuadro $t$ de los jugadores, utilizando la homografía $H_t$ se puede calcular la posición
en la cancha de los jugadores.

\printbibliography

\end{document}
